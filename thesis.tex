% Autor der Vorlage: Dominik Auracher
\documentclass[11pt, bibliography=numbered, headsepline, numbers=withenddot]{scrartcl}

% Paket für Abkürzungsverzeichnis
\usepackage[acronym, automake=delayed, nopostdot]{glossaries}
% Neue Abkürzung definieren (Sortierung egal)
% Aufbau: \newacronym{label-for-the-text}{short-version}{long-version}
\newacronym{waf}{WAF}{Web Application Firewall}
\newacronym{crs}{CRS}{CoreRuleSet}
\newacronym{faq}{FAQ}{Frequently asked questions}
\newacronym{xss}{XSS}{Cross-Site Scripting}
\newacronym{http}{HTTP}{Hypertext Transfer Protocol}
\newacronym{https}{HTTPS}{Hypertext Transfer Protocol Secure}
\newacronym{html}{HTML}{Hypertext Markup Language}
\newacronym{cwe}{CWE}{Common Weakness Enumeration}
\newacronym{nist}{NIST}{National Institute of Standards and Technology}
\newacronym{ip}{IP}{Internet Protocol}
\newacronym{tcp}{TCP}{Transmission Control Protocol}
\newacronym{uri}{URI}{Uniform Resource Identifier}

\makeglossaries

% Allgemeine Informationen
\renewcommand{\author}{\color{red}Vorname Nachname} % Autor
\newcommand{\jury}{\color{red}Prüfungskomitee} % Prüfungskomitee
\newcommand{\location}{\color{red}München} % Ort, an welchem die Arbeit geschrieben wurde
\newcommand{\matriculationnumber}{\color{red}123456} % Matrikelnummer
\newcommand{\submissiondate}{\color{red}\today} % Datum der Abgabe
\newcommand{\supervisor}{\color{red}Betreuer} % Betreuung
\renewcommand{\title}{\color{red}Titel der Arbeit} % Titel der Arbeit

% Macro für Anführungsstriche
% Verwendung: \quotes{Text} -> Resultat: "Text"
\newcommand{\quotes}[1]{``#1''}

% WICHTIG:
% Bei erstmaliger Einrichtung TeXstudio -> TeXstudio konfigurieren -> Optionen -> Erzeugen -> "Standard Bibliographieprogramm" = "Biber" setzen!
\usepackage[a4paper, left=2.5cm, top=3cm, right=3cm, bottom=2.75cm]{geometry}

% Paket zur eigenen Darstellung der Kopf-/Fußzeilen
\usepackage[automark]{scrlayer-scrpage}
\clearscrheadfoot
\setlength{\headheight}{1.27cm}
\setlength{\footheight}{1.27cm}

% Paket zur Ermittlung der letzten Seite
\usepackage{lastpage}

% Paket für Deutschsprachige Inhalte
\usepackage[ngerman]{babel}
\usepackage[utf8]{inputenc}
\usepackage[T1]{fontenc}
\usepackage[babel, german=quotes]{csquotes}

% Paket für Quellcode-Anzeige (auch Inline!)
\usepackage{listings}

% Paket für Auflistungen
\usepackage[inline]{enumitem}

% Paket für Grafiken
\usepackage{graphicx}

% Paket für Farbgebungen
\usepackage{xcolor}

% Quellenverwaltung
\usepackage[backend=biber, style=alphabetic, isbn=false, hyperref=true]{biblatex}
\addbibresource{literature.bib}
%\DeclareFieldFormat*{title}{#1}            % Format for bibliography
\DeclareFieldFormat*{citetitle}{#1}        % Format for citations

% Paket, welches das Jahr zusätzlich in die Fußnote aufnehmen kann
\usepackage{xpatch}
\xapptobibmacro{cite}{\setunit{\nametitledelim}\printfield{year}}{}{}

% Paket und Konfiguration für Zahlen
\usepackage[locale=DE]{siunitx}
\sisetup{per-mode=fraction}

% Paket und Konfiguration für Schriftart
% Zur Verwendung von Arial, TeXstudio konfigurieren: Optionen -> TeXstudio konfigurieren -> Optionen -> Erzeugen -> "Standardcompiler" = "XeLaTeX" setzen!
% \usepackage{fontspec}
% \setmainfont{Arial}
% \setsansfont{Arial}

% Wichtig: Als letztes Paket laden!
\usepackage{hyperref}
\hypersetup{
	colorlinks=false,
	allbordercolors=white
}

% Neu-Definition der Namen
\renewcommand{\sectionautorefname}{Abschnitt}
\renewcommand{\subsectionautorefname}{Abschnitt}
\renewcommand{\subsubsectionautorefname}{Abschnitt}
\renewcommand{\figureautorefname}{Abb.}
\renewcommand{\tableautorefname}{Tab.}
\renewcommand{\listoflofentryname}{Abb.}

% Zeilenabstand einstellen
\renewcommand{\baselinestretch}{1.25}

\begin{document}
	% Deckblatt
	\begin{titlepage}
	\centering
	\includegraphics[width=0.8\textwidth]{"General/HDBW_Logo_large.png"}
	\vfill
	{\LARGE\bfseries Seminararbeit\par}
	\vspace{1cm}
	{\Large\bfseries \title\par}
	\vfill
	{\Large vorgelegt von:\par}
	{\Large\bfseries \author\par}
	{\Large (Matrikelnummer: \matriculationnumber)\par}
	\vfill
	\begin{tabular}{ll}
		vorgelegt am: & \submissiondate\\
	\end{tabular}
	\par
	\begin{tabular}{ll}
		Prüfer: & \jury\\
		Betreuung: & \supervisor\\
	\end{tabular}
\end{titlepage}

	\newpage

	% Kopf-/Fußzeilen ab 2. Seite
	\renewcommand*\sectionmarkformat{}% keine Nummerierung im Kopf
	\renewcommand*{\headfont}{\normalfont} % Nicht kursiv	
	\rohead*{\includegraphics{"General/HDBW_Logo_small.png"}}
	\lofoot*{\scriptsize \submissiondate}
	\rofoot*{\scriptsize Seite \thepage\ von \pageref{LastPage}}
	
	% Eidesstattliche Erklärung
	\lohead*{Eidesstattliche Erklärung}

\section*{Eidesstattliche Erklärung}
\label{sec:Eidesstattliche Erklärung}
Hiermit versichere ich, dass die vorliegende Arbeit ohne Hilfe Dritter und nur mit den angegebenen Quellen und Hilfsmitteln angefertigt wurde. Alle verwendeten Passagen wurden kenntlich gemacht. Diese Arbeit hat in gleicher oder ähnlicher Form noch keiner Prüfungsbehörde vorgelegen.
\newline
\newline
\newline
\newline
\author
\newline
\location, \submissiondate

	\newpage
	
	% Zusammenfassung
	\lohead*{Zusammenfassung}

\section*{Zusammenfassung}
\label{sec:Zusammenfassung}
Kurze \gls{test} Zusammenfassung der wichtigsten \gls{test} Inhalte

	\newpage
	
	% Kopf-/Fußzeilen ab 4. Seite
	\lohead*{\headmark}
	
	% Inhaltsverzeichnis
	\tableofcontents
	\newpage
	
	% Abkürzungsverzeichnis
	\printglossary[type=\acronymtype,title={Abbreviations},style=tree,nonumberlist]
	
	% Abbildungsverzeichnis
	\listoffigures
	
	% Tabellenverzeichnis
	%\listoftables
	
	% Quellcode-Verzeichnis
	%\lstlistoflistings
	\newpage
	
	% --- INHALT ---
	\section{Ein Beispiel-Hauptkapitel}
\label{sec:Ein Beispiel-Hauptkapitel}
% In diesem Kapitel zu sehen:
% - Anführungsstriche (siehe z.B. \quotes{Onlinevideo} wird zu "Onlinevideo")
% - Verbieten von Silbentrennung (siehe \mbox{Kopfzeile})
% - Zentrierter Text (siehe \begin{center} bzw. \end{center})
% - Kursiver Text (siehe \textit{Dies ist ein kursiver Text})
% - Fetter Text (siehe \textbf{Dies ist ein fetter Text})
% - Unterstrichener Text (siehe \underline{Dies ist ein unterstrichener Text})
Video bietet eine leistungsstarke Möglichkeit zur Unterstützung Ihres Standpunkts. Wenn Sie auf \quotes{Onlinevideo} klicken, können Sie den Einbettungscode für das Video einfügen, das hinzugefügt werden soll. Sie können auch ein Stichwort eingeben, um online nach dem Videoclip zu suchen, der optimal zu Ihrem Dokument passt. Damit Ihr Dokument ein professionelles Aussehen erhält, stellt Word einander ergänzende Designs für \mbox{Kopfzeile}, Fußzeile, Deckblatt und Textfelder zur Verfügung. Beispielsweise können Sie ein passendes Deckblatt mit \mbox{Kopfzeile} und Randleiste hinzufügen. Klicken Sie auf \quotes{Einfügen}, und wählen Sie dann die gewünschten Elemente aus den verschiedenen Katalogen aus. Designs und Formatvorlagen helfen auch dabei, die Elemente Ihres Dokuments aufeinander abzustimmen. Wenn Sie auf \quotes{Entwurf} klicken und ein neues Design auswählen, ändern sich die Grafiken, Diagramme und SmartArt-Grafiken so, dass sie dem neuen Design entsprechen. Wenn Sie Formatvorlagen anwenden, ändern sich die Überschriften passend zum neuen Design. Sparen Sie Zeit in Word dank neuer Schaltflächen, die angezeigt werden, wo Sie sie benötigen. 

\begin{center}
	Dies ist ein zentrierter Text
\end{center}

\textit{Dies ist ein kursiver Text}

\textbf{Dies ist ein fetter Text}

\underline{Dies ist ein unterstrichener Text}

\underline{Dies ist eine \textbf{Kombination} aus \textit{vielem}}

\subsection{Ein Unterkapitel}
\label{sec:Ein Unterkapitel}
% In diesem Kapitel zu sehen:
% - Zitate
%   Voraussetzung ist ein Eintrag mit dem jeweiligen Namen in der Datei "literatur.bib"
%   - Indirektes Zitieren (siehe \autocite[Vgl.][]{BavariaIpsum}), wodurch eine Fußnote und ein Eintrag im Literaturverzeichnis aufgeführt wird
%   - Direktes Zitieren (siehe \autocite{BavariaIpsum}), wodurch ebenfalls eine Fußnote und ein Eintrag im Literaturverzeichnis aufgeführt wird - jedoch ein anderer
%   - Indirektes Zitat mit Angabe von Seitenzahl 
% - Neue (eingerückte) Zeile direkt vor dem Zitat (\quotes{Einfach ausgedrückt, ist Bavaria ipsum eine neue Variante des Lorem-ipsum-Blindtextes.}\autocite{BavariaIpsum})
% - Neue (nicht eingerückte) Zeile (\noindent Der Algorithmus ...)
% - Eigene Silbentrennung (siehe Non-Profit-Or\-ga\-ni\-sa\-tion; wird an einem der \- getrennt - sofern erforderlich, wenn nicht erforderlich, wird keiner der \- angezeigt)
Aus’gschammta, Umstandskrama, Klaubauf, Jungfa, Schdehlratz, Vieh mit Haxn, Zwedschgarl, varreckta Deifi, Hampara, Schuggsn, Daamaluudscha, Aufgeiga, Goaspeterl, Stodara, Palmesel, Schrumsl, Fünferl, oide Schäsn, Sagglzemend, Grantlhuaba, Ruaßnosn, Gibskobf, mit deinen Badwandlfüaß, Broatarsch, Gscheidal, Blattada, Hornochs, Asphaltwanzn, du Ams’l, du bleede, schleich di, Klugscheissa, Sagglzemend, Gschpusi, oide Rudschn, Hampara, Pfundhamme, oida Schlawina, Aushuifsbaya, Dreegsau, damischa Saupreiß, Auftaklta, Luada, Kirchalicht, Rotzgloggn, oide Schäsn, Heislmo, Schdinkadores, glei fangst a boa, Beitlschneida, Ecknsteha, Hanswurst, hosd mi, oida Schlawina, Freindal, Freibialädschn, hoit’s Mei, Griasgram, Drottl, varreckter Hund, Bettwanzn, Griasgram, Schundnickl, depperta Doafdebb, Zwedschgnmanndl, Griasgram, oida Schlawina, eigschnabbda, Pfennigfuxa, Woibbadinga, greißlicha Uhu, Grattla, Eisackla, Klobürschdn, Affnasch, oide Bixn, Dreegsau, Beitlschneida, Heislschlaicha, Dipfalscheißa, Jungfa, Hallodri, Hosnscheissa, Pfundhamml, Betonschedl, Hemmadbiesla, Fünferl, bsuffas Wagscheidl, Hopfastanga, Hungaleida, Kniabisla, Charaktasau, Krampfhenna, Ecknsteha, damischa Saupreiß, Palmesel, Oasch, Radlfahra, schleich di, Betonschedl, Zigarettnbiaschal!\autocite[Vgl.][]{BavariaIpsum} 

\quotes{Einfach ausgedrückt, ist Bavaria ipsum eine neue Variante des Lorem-ipsum-Blindtextes.}\autocite{BavariaIpsum}

\noindent Der Algorithmus für die Verschlüsselung und etwas mehr wurde durch die Non-Profit-Or\-ga\-ni\-sa\-tion entwickelt.\autocite[Vgl.][73]{HeiseHarterWettbewerb}

\subsection{Noch ein Unterkapitel}
\label{sec:Noch ein Unterkapitel}
% In diesem Kapitel zu sehen:
% - Einfügen eines Bildes (alles zwischen \begin{figure} und \end{figure})
% - Referenz auf irgendetwas, hier ein Bild (\ref{fig:hdbw-logo})
% - Nutzung von Abkürzungen (bedingt einen Eintrag in der Datei "Abkürzungen.tex"; siehe \gls{owa})
%   -> WICHTIG: Das Abkürzungsverzeichnis wird (warum auch immer) erst nach dem zweiten Build aktualisiert. D.h. beim Abschluss oder wenn die Änderungen wirksam werden sollen, einfach zweimal den Play-Button drücken ;)
% - Beschreibungen (alles zwischen \begin{description} und \end{description})
% - Aufzählungen
%   - Allgemein (siehe \begin{itemize} bis \end{itemize})
%   - Nummerisch (siehe \begin{enumerate} bis \end{enumerate})
%   - Gemischt (ganz unten)
Hier ist ein schönes Bild mit einer Referenz auf Abbildung \ref{fig:hdbw-logo} zu sehen.\autocite[Vgl.][]{HDBWLogo} Außerdem ist es schön \gls{owa} zu nutzen, da \gls{owa} auch schöne Funktionen bietet.
\begin{figure}
	\centering
	\includegraphics[width=0.8\textwidth]{"General/HDBW_Logo_large.png"}
	\caption{Logo der HDBW-Hochschule (\citeauthor{HDBWLogo} \citeyear{HDBWLogo})}
	\label{fig:hdbw-logo}
\end{figure}

\subsubsection{Beschreibung}
\label{sec:Beschreibung}
\begin{description}
	\item [Trottel] Leute, die einfach doof sind
	\item [Idiot] Das selbe, nur anders
\end{description}

\subsubsection{Allgemeine Aufzählung}
\label{sec:Allgemeine Aufzählung}
\begin{itemize}
\item Erster Punkt
	\begin{itemize}
		\item 1. Ebene
	\end{itemize}
\item Zweiter Punkt
\end{itemize}

\subsubsection{Nummerische Aufzählung}
\label{sec:Nummerische Aufzählung}
\begin{enumerate} 
\item Erster Punkt
	\begin{itemize}
		\item 1. Ebene
	\end{itemize}
\item Zweiter Punkt
\end{enumerate}

\subsubsection{Gemischte Aufzählung}
\label{sec:Gemischte Aufzählung}
\begin{enumerate} 
	\item Erster Punkt
	\begin{enumerate}
		\item 1. Ebene
	\end{enumerate}
	\item Zweiter Punkt
\end{enumerate}

	
	
	% Quellenverzeichnis
	\newpage
	\printbibliography
	
\end{document}
