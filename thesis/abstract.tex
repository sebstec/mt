\section{Abstract}
Web application firewalls are employed in a defense-in-depth strategy to protect web applications.
Institutions deciding to employ a web application firewall likely ask the question how well the web application can protect a web application against targeted malicious requests.
After deciding to employ a web application firewall, knowing the answer to this question is vital to correctly assess the predicted standing of the protected web application against the threat landscape.
In order to evaluate a web application firewall against targeted malicious requests, a gray box evaluation system is proposed. The main idea is to simulate malicious requests against the web application using firewall evasion techniques, like a real attacker would, and simultaneously use the firewall log to increase the efficiency in finding bypassing payloads.
Found bypasses are subsequently used to propose ruleset additions in order to reduce the total amount of efficient firewall evasion techniques against the evaluated firewall configuration.
Experiments are conducted using the proposed system to evaluate the open-source OWASP ModSecurity firewall configured to use the OWASP \acrfull{crs} version 4.1 against targeted malicious \acrfull{xss} payloads.
The results reveal that 11 of 18 researched evasion techniques to obfuscate \acrshort{xss} payloads can be used in some combination to create multiple bypassing malicious payloads.
As response to found bypasses, proposals for ruleset additions to the \acrshort{crs} 4.1 could be given based on the developed system.
