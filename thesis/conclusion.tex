\section{Conclusion}
\label{sec:conclusion}
{\color{red}TODO
Firewall evasion is a tricky topic. A lot of influencing factors: Firewall, Configuration, Payload (Language, Type, Input Options), Blackbox or Whitebox. 

Hier nochmal zusammenfassung der ergebnisse und rotem faden
}


{\color{blue} formulierung verbessern und erweitern: }
Based on the proposed evaluation technique, the question is answered by X of Y tested malicious cross-site scripting requests have been filtered previously to the implementation of derived rules. This result increased by Z\% to A of B malicious requests after the implementation of derived rules. For a complete overview regarding a specific web application, all potentially malicious payloads based on the used web technologies powering the web application need to be investigated in a similar manner. Additionally, payload modifications by more applicable evasion techniques can be considered. Nevertheless, the results of evaluating only cross-site scripting payloads allow an estimation of the firewall performance against the focused cross-site scripting payloads. No matter if a complete evaluation investigating multiple potential weaknesses or investigating only one weakness is conducted, using the proposed evaluation technique allows to explore potential loopholes in the rules configuration in any case. This gives insight into filtering performace and subsequently allows to close those loopholes by implementing new rules.

{\color{red} TODO}

Possible improvements and further research into the evaluation approach shown in the context of this work are discussed in the following section, Section~\ref{sec:continuations}: Continuations. 


