\subsection{Related work}
In order to evaluate the performance of web application firewalls, Demetrio et al. \cite{rw/demetrio} propose WAF-A-MoLE, a tool following a guided mutational fuzz testing approach to bypass machine learning based web application firewalls. WAF-A-MoLE repeatedly transforms a failing test SQL-injection payload through random application of predefined mutational operators. Execution results of those payloads are ordered according to performance metrics. Subsequently, the process is repeated for the payloads with better execution results until a bypassing payload is found. The performance metrics are based on the confidence value generated by the web application firewall during the filtering of a payload. The confidence value represents the web application firewalls confidence in its classification of a payload as malicious. Once a certain confidence value threshold is undershot, the payload bypasses the firewall. It has been shown that WAF-A-MoLE successfully manages to degrade the confidence value of a machine learning based web application firewall to bypass a semantically equal SQL-injection payloads to \verb|admin ' OR 1=1#|.


Mohammadhossein et al. \cite{rw/mohammad} propose Reinforcement-Learning-Driven and Adaptive Testing (RAT), an automated black-box testing strategy to discover injection vulnerabilities in web application firewalls.
