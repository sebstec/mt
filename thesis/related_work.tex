\subsection{Related work}
In order to evaluate the performance of web application firewalls, Demetrio et al. \cite{rw/demetrio} propose WAF-A-MoLE, a tool following a guided mutational fuzz testing approach to bypass machine learning based web application firewalls. WAF-A-MoLE repeatedly transforms a failing test SQL-injection payload through random application of predefined mutational operators. Execution results of those payloads are ordered according to performance metrics. Subsequently, the process is repeated for the payloads with better execution results until a bypassing payload is found. The performance metrics are based on the confidence value generated by the web application firewall during the filtering of a payload. The confidence value represents the web application firewalls confidence in its classification of a payload as malicious. Once a certain confidence value threshold is undershot, the payload bypasses the firewall. It has been shown that WAF-A-MoLE successfully manages to degrade the confidence value of a machine learning based web application firewall to bypass a semantically equal SQL-injection payloads to \verb|admin ' OR 1=1#|.

Mohammadhossein et al. \cite{rw/mohammad} propose Reinforcement-Learning-Driven and Adaptive Testing (RAT), an automated black-box testing strategy to discover injection vulnerabilities in web application firewalls. They focus on SQL Injection and Cross-site scripting payloads. In order to improve efficiency in black-box testing, RAT clusters similar attack samples together. It then utilizes a reinforcement learning technique combined with a novel adaptive search algorithm to discover almost all bypassing attack patterns efficiently. RAT is compared with three state-of-the-art methods. RAT is supposed to performs significantly better than its counterparts in discovering the most possible bypassing payloads and reducing the number of attempts before finding the first bypassing payload when testing against well-configured WAFs.

Ke Li et al. \cite{rw/derLI} propose DaNuoYi, an automatic injection testing tool that simultaneously generates test inputs for multiple types of injection attacks on a Web application firewall. It's hypothesis is that test inputs from syntactically different types of injection attacks share certain latent semantic similarities that can be useful to generate sophisticated test inputs for each other. DaNuoYi assumes a context-free grammar (CFG) for each type of injection attack. This CFG is used to create test inputs to profile a Web application firewall. DaNuoYi trains a classifier to predict the likelihood of a test input bypassing the WAF. A multi-task injection translation paradigm to bridge the test input generation across different types of injection attack is developed. For any pair of injection types, a translation model is built that translates the test input from one type of injection
attack into a semantically related one. These models are trained using the data also generated by the CFG. Test inputs for multiple types of injection attacks are generated simultaneously by a multi-task evolutionary algorithm (MTEA). This MTEA shares promising test inputs generated for one type of injection attack across other types of injection attack by using the multi-task translation models.
At each generation, domain-specific mutation operators similar to evasion techniques are used to generate offspring inputs, which are then also tested against the Web application firewall.
It is shown that when employing DaNuoYi, both the multi-task translation and multi-task search are effective in handling and transferring the common semantic information for different injection types. This resulted in more bypassing test results that evaluated, single-task counterparts.

