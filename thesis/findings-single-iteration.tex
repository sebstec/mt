\subsection{Single Iteration Evaluation}
This chapter states the evaluation results of a single iteration of application of an evasion technique to the specified payloads. If a considerable (from the perspective of the author of this work) bypass was found, the result will be detailed. Evaluation results of blocked requests are referred to in attachment.


\subsubsection{Unicode in JSON}
\label{sec:unicodeinjsontest}
The ModSecurity firewall detects and blocks requests containing \verb|${| followed by a closing \verb|}| after an arbitrary number of characters in between.
Tests have shown that using Unicode encoding in JSON and substituting the \verb|$| in a JSON request with \verb|\u0024| evades the previously mentioned detection, but triggers another firewall rule: \verb|Possible Unicode character bypass detected|.
(see Listing~\ref{lst:jsonunicodetests})


\subsubsection{Percent encoding}
The ModSecurity firewall detects some percent-encoded payloads. Requests with plain as well as percent-encoded payloads were sent to the reverse proxy. All requests triggered the same rules. (See Listing~\ref{lst:urlencodedexampleblocked})


\subsection{Tagged Template Literals}
As stated in Listing~\ref{lst:alertXSSblocked}, the payload \verb|alert('XSS')| is being blocked by the tested firewall.
After substituting the function call using \verb|('XSS')| with a Tagged Template Literal in the form of \verb|alert`XSS`|, the payload successfully bypasses the firewall and reaches the web server. (Listing~\ref{lst:alertXSSbypass})

\begin{lstlisting}[style=ruleStyle, language=XML, caption=alert("XSS") blocked, label=lst:alertXSSblocked]
<payload>alert("XSS")</payload>
<file>"rules/REQUEST-941-APPLICATION-ATTACK-XSS.conf"</file>
<fileDetails>[line "714"] [id "941390"]<fileDetails>
<MatchedData>"alert("</MatchedData>
\end{lstlisting}

\begin{lstlisting}[style=ruleStyle, language=XML, caption=alert`XSS` bypass, label=lst:alertXSSbypass]
<payload>alert("XSS")</payload>
<bypass>alert`XSS`</payload>
\end{lstlisting}


