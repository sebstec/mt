\subsection{Overview and Scope}
To answer the afromentioned question, this work aims at evaluating web application firewalls using firewall evasion techniques. Firewall Evasion techniques are used by attackers specifically targeting a Web Application behind a Web Application Firewall. Attackers try to evade detection by using these techniques. 
% Firewall Evasion techniques will be detailed in the further course of this work. 

While malicious request containing payloads without obfuscation are certainly of importance, this work will focus on targeted attacks using payloads that were obscured using Firewall Evasion techniques.
% In Section~\ref{sec:fundamentals}, fundamentals regarding Web Applications, Web Application Security, Web Application Firewalls and Web Application Firewall Evasion will be detailed.
Targeted attacks come in many different forms, one being Cross-site scripting payloads.
With JavaScript being the most used programming language as of 2023 \cite{statista/mostusedlang} and Cross-Site Scripting being a notable weakness linked to Injection vulnerabilities \cite{OWASP/Injection21}, this work focuses on Cross-site Scripting payloads.

Following this, in Section~\ref{sec:firewallevasiontechniques}, Firewall Evasion techniques applicable to Cross-site scripting payloads will be discussed. 
An iterative approach will be developed to try and create bypassing payloads using those evasion techniques to evade the open-source ModSecurity firewall configured to use the OWASP CoreRuleSet in version 4.1. 

Section~\ref{sec:evaluation} and Section~\ref{sec:EvaluationResults} will describe the evaluation methodology and evaluation results. In this context, this work aims at investigating the development steps of bypassing payloads and discussing occuring payload limitations in the context of tested firewall configurations, properties of the Web application to be protected and used firewall evasion techniques. Section~\ref{sec:rulesproposal} will detail proposed rule additions based on found bypassing payloads and effective evasion techniques.

Subsequently, a generic evaluation proposal will be presented and assessed in Section~\ref{sec:proposal}. Finally, this work will be concluded (Section~\ref{sec:conclusion}) and ideas for further reasearch will be proposed (Section~\ref{sec:continuations}). 

In this work, no LLM-based or other machine learning approaches are pursued. Nonetheless, these approaches are promising and will be mentioned in Section~\ref{sec:continuations}.

Initially, in the following section, Section~\ref{sec:fundamentals}, fundamentals regarding Web Applications, Web Application Security, Web Application Firewalls and Web Application Firewall Evasion will be detailed.
