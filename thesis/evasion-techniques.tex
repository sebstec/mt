\section{Firewall evasion techniques}
\label{sec:firewallevasiontechniques}

This section describes various Web application firewall evasion techniques. These techniques are based on the authors research of freely available resources listing Web application firewall evasion techniques.

Evasion techniques are split into Section~\ref{sec:varioustech} and Section~\ref{sec:xsstech}. With JavaScript being the most used programming language as of 2023 \cite{statista/mostusedlang} and Cross-Site Scripting being a notable weakness linked to Injection vulnerabilities \cite{OWASP/Injection21}, techniques around \gls{xss} are dominantly represented in this section.

The validity of \gls{xss} payloads depends on the JavaScript version supported by the browser running the commands. The name of the JavaScript language is trademarked, therefore the official name of JavaScript is ECMAScript. ECMAScript is best known as the language embedded in web browsers. The standards organization Ecma manages the language. The current Ecma standard defines the ECMAScript 2023 language. It is the fourteenth edition of the ECMAScript Language Specification. Up to the release of the sixth edition of ECMAScript, the release cycle between versions spanned multiple years. After the release of the sixth edition of ECMAScript, Ecma adopted a new, yearly release cadence with the last release in June 2023. \cite{ecma/release,ecma/intro,explorejs/ecmascript}
Depending on the implementation of new ECMAScript features by web-browser maintainers and the browser version used, some browser installations might not support certain payloads consisting of ECMAScript statements. \cite{kangax/compattable}

All payloads have been tested for validity in FireFox browser version 124.0.1-1.

