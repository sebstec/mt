\section{Firewall evasion techniques}

Compatibility with browsers is not being regarded. {\color{red} TODO: check JavaScript versions (ECMA script 6) and write about browser compatibility}
Firewall evasion techniques covering XSS Payloads have been chosen.

All techniques are available {\color{red}(at the time of writing)}

\subsection{XSS Payloads}

\subsubsection{String passing}

\subsubsection{Tagged Template Literals}
Tagged Template Literals can be used in JavaScript payloads to avoid detection of JavaScript function calls in an http request. Dr. Alex Rauschmeyer explains Tagged Template Literals in ``Exploring ES6'':
\begin{quotation} Tagged Template Literals are function calls whose parameters are provided via template literals. [...]
	The following is a tagged template literal (short: tagged template):
	\begin{lstlisting}
tagFunction`Hello ${firstName} ${lastName}!`
\end{lstlisting}
	Putting a template literal after an expression triggers a function call, similar to how a parameter list (comma-separated values in parentheses) triggers a function call. The previous code is equivalent to the following function call (in reality, first parameter is more than just an Array, but that is explained later).
	\begin{lstlisting}
tagFunction(['Hello ', ' ', '!'], firstName, lastName)
\end{lstlisting}
	Thus, the name before the content in backticks is the name of a function to call, the tag function.
	\cite{exploringes6/templatelit}
\end{quotation}

For instance, as you can also see in Listing~\ref{lst:listing-alertXSSblocked}, the payload \verb|alert('XSS')| will be blocked by the ModSecurity firewall while using CRS 4.1.0.

\begin{lstlisting}[style=ruleStyle, language=XML, caption=alert("XSS") blocking example, label=lst:listing-alertXSSblocked]
<payload>alert("XSS")</payload>
<file>"rules/REQUEST-941-APPLICATION-ATTACK-XSS.conf"</file>
<fileDetails>[line "714"] [id "941390"]<fileDetails>
<MatchedData>"alert("</MatchedData>
\end{lstlisting}

As the function \verb|alert| can be called with an array as first parameter, it is possible to substitute the payload with \verb|alert`XSS`|. The substitute payload will pass through the ModSecurity firewall in the defined configuration. {\color{red}(insert ref to config here)}

Technique by \cite{onecons/wafbypass}.

\subsubsection{eval() function}
\label{sec:eval}

JavaScript payloads that are being passed in string format can be executed using the eval function. The \verb|eval()| function will evaluate the source string given as an argument and return its completion value. \cite{js/eval}
A possible usecase for \verb|eval()| in the quest to bypass a \gls{waf} is the splitting of function names that would cause a HTTP request to get rejected by the firewall.
Looking at Listing~\ref{lst:listing-alertXSSblocked}, it is clear that the function call \verb|alert('XSS')| is being blocked by the ModSecurity firewall using CRS 4.1.0.
To avoid a match on \verb|alert(|, it is possible to split the function call into parts \verb|al|, \verb|e| and \verb|rt('XSS')| by using string concatenation in the form of \verb|`al` + `e` + `rt('XSS')`|.
Sending the payload in this form will not cause the the desired effect, as a JavaScript interpreter will interpret the payload as a string instead of a function call to the \verb|alert()| function.
Using the \verb|eval()| function in combination with the splitted string solves this problem. Passing the string \verb|`al` + `e` + `rt('XSS')`| in the form of \verb|eval(`al` + `e` + `rt('XSS')`)| as an argument to the \verb|eval()| function will cause the string to be evaluated and interpreted. Technique inspired by \cite{onecons/wafbypass}.
The ModSecurity firewall using CRS 4.1.0. checks for usage of the function \verb|eval()| and tries to block requests containing it. An example is listed in Listing~\ref{lst:listing-evalalertXSSblocked}.

\begin{lstlisting}[style=ruleStyle, language=XML, caption=eval(`al` + `e` + `rt('XSS')`) blocking example, label={lst:listing-evalalertXSSblocked}]
<payload>eval(`al` + `e` + `rt('XSS')`)</payload>
<file>"rules/REQUEST-933-APPLICATION-ATTACK-PHP.conf"</file>
<fileDetails>[line "331"] [id "933160"]<fileDetails>
<MatchedData>"eval(`al`   `e`   `rt('XSS')"</MatchedData>

<file>"rules/REQUEST-934-APPLICATION-ATTACK-GENERIC.conf"</file>
<fileDetails>[line "52"] [id "934100"]<fileDetails>
<MatchedData>"eval("</MatchedData>

<file>"rules/REQUEST-941-APPLICATION-ATTACK-XSS.conf"</file>
<fileDetails>[line "714"] [id "941390"]<fileDetails>
<MatchedData>"eval("</MatchedData>
\end{lstlisting}

\subsubsection{function constructor}

When using the function \verb|eval()| (see Section~\ref{sec:eval}) causes a request to be blocked , a substitute in the form of \verb|[].constructor.constructor(`alert('XSS')`)()| is available in JavaScript. 
Calling \verb|[]| will create an empty array, which is a special kind of object.
Any kind of object in JavaScript, with the exception of null prototype objects, will have a constructor property on its prototype. \cite{js/object}
The Array objects constructor can be accessed by calling \verb|.constructor| on an instance of Array, like \verb|[].constructor|. \cite{js/array} 
Constructors are technically regular functions. \cite{js/constructor} 
As such, they are Function objects. Function objects have a constructor themselfes, which again is called using the \verb|.constructor| notation.
Using \verb|[].constructor.constructor| will yield access to the Function() constructor.
Calling the Function() constructor directly can create functions dynamically, similar to using \verb|eval()|.
The difference being that the Function() constructor creates functions that execute in the global scope only.
The last argument given to the Function() constructor is expected to be a string containing the JavaScript statements compromising the function definition. \cite{js/function} 
Therefore, the afromentioned substitute for \verb|eval()| -  \verb|[].constructor.constructor(`alert('XSS')`)()| - will create a function with the function body \verb|alert('XSS')| and call it directly.
In other words: It will call a function that calls \verb|alert('XSS')|.


Technique by \cite{onecons/wafbypass}.

\subsubsection{square bracket notation}
JavaScript provides a \quotes{square bracket notation} to access properties of an Object. It is used to access multiword properties as well as provide a way of obtaining property names as the result of an expression. \cite{js/brackets}
This \quotes{square bracket notation} allows to ...

\verb|[][`constructor`][`constructor`](`alert('XSS')`)()|


\subsubsection{string replace}

Some characters like () can be replaced by [open+[]][0][13] [open+[]][0][13]

Technique by \cite{onecons/wafbypass}.
