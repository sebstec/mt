\section{Firewall evasion techniques}
\label{sec:Firewall evasion techniques}

Firewall evasion techniques covering XSS Payloads have been chosen.


\definecolor{codegreen}{rgb}{0,0.6,0}
\definecolor{codegray}{rgb}{0.5,0.5,0.5}
\definecolor{codepurple}{rgb}{0.58,0,0.82}
\definecolor{backcolour}{rgb}{0.95,0.95,0.92}
\lstdefinestyle{ruleStyle}{
	backgroundcolor=\color{backcolour},
	commentstyle=\color{codegreen},
	keywordstyle=\color{magenta},
	numberstyle=\tiny\color{codegray},
	stringstyle=\color{codepurple},
	identifierstyle=\color{blue},
	rulecolor=\color{black},
	basicstyle=\ttfamily\footnotesize,
	breakatwhitespace=false,
	breaklines=true,
	captionpos=b,
	keepspaces=true,
	numbers=left,
	showspaces=false,
	showstringspaces=false,
	showtabs=false,
	tabsize=2,
	frame=single,
	morekeywords={test, do, <file>},
	aboveskip=1\baselineskip,
	escapeinside=``
}

\subsection{XSS Payloads}

\subsubsection{Tagged Template Literals}
Tagged Template Literals can be used in javascript payloads to avoid detection of javascript function calls in an http request. Tagged Template Literals are ...
\cite{exploringes6/templatelit}

For instance, as you can also see in Listing~\ref{lst:listing-alertXSS}, the payload \verb|alert('XSS')| will be blocked by the ModSecurity firewall while using CRS 4.1.0.

\begin{lstlisting}[style=ruleStyle, language=XML, caption=alert("XSS") blocking example]
`\label{lst:listing-alertXSS}`<payload>alert("XSS")</payload>
<file>"rules/REQUEST-941-APPLICATION-ATTACK-XSS.conf"</file>
<fileDetails>[line "714"] [id "941390"]<fileDetails>
<MatchedData>"alert("</MatchedData>
\end{lstlisting}

Technique by \cite{onecons/wafbypass}.

\label{sec:XSS Payloads}
