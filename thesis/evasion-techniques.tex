\section{Firewall evasion techniques}

Compatibility with browsers is not being regarded. {\color{red} TODO: check JavaScript versions (ECMA script 6) and write about browser compatibility}
Naturally, when discussing web application firewall evasion, most chosen firewall evasion techniques are covering XSS Payloads.

All techniques are available {\color{red}(at the time of writing)}

\subsection{Various Payloads}

\subsubsection{Payload size}

\subsubsection{Unicode encoding in JSON}
Detected by modsecurity, checked with \u0024 and \u0028

\subsubsection{Case alternation}
In order to evade regex filerting by the WAF, the case of a payload can be alternated. \cite{medium/allypetitt}
Modern regex flavors allow the application of modifiers to parts of the regular expression.
One such modifier is \verb|(?i)|. It makes the regex case insensitive. \cite{regex/jan} Only when this modifier is not used, can a payload evade regex filtering using case alternation.

Taking the XSS payload \verb|<script>alert('XSS')</script>| as an example, after applying case alternation, it might result in a payload in the form of: \verb|<sCrIpT>alert('XSS')</sCriPt>|.
Another example is file access on a wrongfully public file.
The Windows file system trests file and directory names as case-insensitive by default. \cite{windows/casesensitive}
On a web server hosted on Windows that exposes a .env file with stored secrets, both urls: \\ \verb|http://127.0.0.1:8000/.env| and \verb|http://127.0.0.1:8000/.enV| \\
are treated equally. {\color{blue} verified by the author using the python webserver module on windows 10 [should this be here?]}


\subsubsection{Comment interference}
In some context, comments can be used to break up statements. Regarding SQL, the Oracle Database SQL Reference states:
\begin{quote}
	A comment can appear between any keywords, parameters, or punctuation marks in a statement. You can include a comment in a statement in two ways:
	\begin{itemize}
		\item Begin the comment with a slash and an asterisk (/*). Proceed with the text of the comment. This text can span multiple lines. End the comment with an asterisk and a slash (*/). The opening and terminating characters need not be separated from the text by a space or a line break.
		\item Begin the comment with -- (two hyphens). Proceed with the text of the comment. This text cannot extend to a new line. End the comment with a line break.
	\end{itemize}
	\cite{oracle/sqlcomments}
\end{quote}
Ally Petitt suggests using comments inside SQL statements to break up SQL keywords: \verb|?id=1+un/**/ion+sel/**/ect+1,2,3--| \cite{medium/allypetitt}


\subsubsection{Percent encoding}


\subsubsection{Charset alternation}
To indicate the original media type prior to any applied content encoding, the HTTP \verb|Content-Type| representation header is used.
It can be used in requests by the client to tell the server what type of data is actually sent.
\verb|charset| is a possible directive, that specifies the character encoding standard.
It can be supplied with the \verb|Content-Type| header. \cite{http/contenttype}
If a web server is supporting requests in different encoding standards, but the WAF is not configured to parse certain encoding standards, the WAF may not recognize such encoded request as malicious.

The payload \verb|'<script>alert("xss")</script>'| becomes \\
\verb|'L%A2%83%99%89%97%A3n%81%93%85%99%A3M%7F%A7%A2%A2%7F%5DLa%A2%83%99%89%97%A3n'|
after encoding the payload to the charset \verb|IBM037| followed by percent encoding. \\
This differs from the same payload in \verb|UTF-8| followed by percent encoding: \\
\verb|'%3Cscript%3Ealert%28%22xss%22%29%3C%2Fscript%3E'| \\
\verb|UTF-8| encoding is the standard for URIs according to RFC-3986. \cite{rfc3986} \\
Technique by \cite{medium/allypetitt}.


\subsection{XSS Payloads}

\subsubsection{Tag modification}
In a case, where regex filters are configured to look for html tags, modifying html tags might make a payload evade the filter. Possible modifications include:
\begin{itemize}
	\item prepending the tag with an additional \verb|<|: \verb|<<script>alert('XSS')</script>|
	\item omitting the closing tag: \verb|<script>alert('XSS')|
	\item using double open angle brackets: \verb|<iframe src=javascript:alert('XSS') <|
	\item using uncommon tags: \\ \verb|<STYLE>.classname{background-image:url("javascript:alert(XSS)");}</STYLE>|
\end{itemize}
Technique by \cite{medium/allypetitt}

\subsubsection{Space replace}
A regex filter that is expecting a space inside a html tag at certain positions can be evaded by replacing the space with a \verb|/|.
For instance, \verb|<img src="1" onerror="alert('XSS')">| becomes \verb|<img/src="1"/onerror="alert('XSS')">|.
Technique by \cite{medium/allypetitt}

\subsubsection{HTML encoding}
see the <a> thing

\subsubsection{From charcode}
String.fromCharCode(0x4c,105,0x6e,117,120) \cite{asecsite/jsobf1}

\subsubsection{String passing}

\subsubsection{Tagged Template Literals}
Tagged Template Literals can be used in JavaScript payloads to avoid detection of JavaScript function calls in an http request. Dr. Alex Rauschmeyer explains Tagged Template Literals in ``Exploring ES6'':
\begin{quotation} Tagged Template Literals are function calls whose parameters are provided via template literals. [...]
	The following is a tagged template literal (short: tagged template):
	\begin{lstlisting}
tagFunction`Hello ${firstName} ${lastName}!`
\end{lstlisting}
	Putting a template literal after an expression triggers a function call, similar to how a parameter list (comma-separated values in parentheses) triggers a function call. The previous code is equivalent to the following function call (in reality, first parameter is more than just an Array, but that is explained later).
	\begin{lstlisting}
tagFunction(['Hello ', ' ', '!'], firstName, lastName)
\end{lstlisting}
	Thus, the name before the content in backticks is the name of a function to call, the tag function.
	\cite{exploringes6/templatelit}
\end{quotation}

As the function \verb|alert| can be called with an array as first parameter, it is possible to substitute the payload with \verb|alert`XSS`|. The substitute payload might evade regex filtering that looks for a function name followed by an opening \verb|(|.
Technique by \cite{onecons/wafbypass}.


\subsubsection{eval() function}
\label{sec:eval}

JavaScript payloads that are being passed in string format can be executed using the eval function. The \verb|eval()| function will evaluate the source string given as an argument and return its completion value. \cite{js/eval}
A possible usecase for \verb|eval()| in the quest to bypass a \gls{waf} is the splitting of function names that would cause a HTTP request to get rejected by the firewall.
Looking at Listing~\ref{lst:alertXSSblocked}, it is clear that the function call \verb|alert('XSS')| is being blocked by the ModSecurity firewall using CRS 4.1.0.
To avoid a match on \verb|alert(|, it is possible to split the function call into parts \verb|al|, \verb|e| and \verb|rt('XSS')| by using string concatenation in the form of \verb|`al` + `e` + `rt('XSS')`|.
Sending the payload in this form will not cause the the desired effect, as a JavaScript interpreter will interpret the payload as a string instead of a function call to the \verb|alert()| function.
Using the \verb|eval()| function in combination with the splitted string solves this problem. Passing the string \verb|`al` + `e` + `rt('XSS')`| in the form of \verb|eval(`al` + `e` + `rt('XSS')`)| as an argument to the \verb|eval()| function will cause the string to be evaluated and interpreted. Technique inspired by \cite{onecons/wafbypass}.
The ModSecurity firewall using CRS 4.1.0. checks for usage of the function \verb|eval()| and tries to block requests containing it. An example is listed in Listing~\ref{lst:evalalertXSSblocked}.

\begin{lstlisting}[style=ruleStyle, language=XML, caption=eval(`al` + `e` + `rt('XSS')`) blocking example, label={lst:evalalertXSSblocked}]
<payload>eval(`al` + `e` + `rt('XSS')`)</payload>
<file>"rules/REQUEST-933-APPLICATION-ATTACK-PHP.conf"</file>
<fileDetails>[line "331"] [id "933160"]<fileDetails>
<MatchedData>"eval(`al`   `e`   `rt('XSS')"</MatchedData>

<file>"rules/REQUEST-934-APPLICATION-ATTACK-GENERIC.conf"</file>
<fileDetails>[line "52"] [id "934100"]<fileDetails>
<MatchedData>"eval("</MatchedData>

<file>"rules/REQUEST-941-APPLICATION-ATTACK-XSS.conf"</file>
<fileDetails>[line "714"] [id "941390"]<fileDetails>
<MatchedData>"eval("</MatchedData>
\end{lstlisting}

\subsubsection{function constructor}
\label{sec:functionconstructor}

When using the function \verb|eval()| (see Section~\ref{sec:eval}) causes a request to be blocked, a substitute in the form of \verb|[].constructor.constructor(`alert('XSS')`)()| is available in JavaScript.

Calling \verb|[]| will create an empty array, which is a special kind of object.
Any kind of object in JavaScript, with the exception of null prototype objects, will have a constructor property on its prototype. \cite{js/object}

The Array objects constructor can be accessed by calling \verb|.constructor| on an instance of Array, like \verb|[].constructor|. \cite{js/array}
Constructors are technically regular functions. \cite{js/constructor}
As such, they are Function objects. Function objects have a constructor themselfes, which again is called using the \verb|.constructor| notation.

Using \verb|[].constructor.constructor| will yield access to the Function() constructor.
Calling the Function() constructor directly can create functions dynamically, similar to using \verb|eval()|.
The difference being that the Function() constructor creates functions that execute in the global scope only.

The Function() constructor takes a variable count of arguments. The first $n - 1$ arguments are the \quotes{names to be used by the function as formal argument names. Each must be a string that corresponds to a valid JavaScript parameter [...]} \cite{js/function}
The last argument given to the Function() constructor is expected to be a string containing the JavaScript statements compromising the function definition. \cite{js/function}

Therefore, the afromentioned substitute for \verb|eval()| - \\ \verb|[].constructor.constructor(`alert('XSS')`)()| - \\ will create a function with the function body \verb|alert('XSS')| and call it directly.
In other words: It will call a function that calls \verb|alert('XSS')|.
Technique by \cite{onecons/wafbypass}.

A more straightforward and less obscure way of accessing the Function() contructor is to call it directly:
\begin{quote}
	\verb|Function()| can be called with or without \verb|new|. Both create a new \verb|Function| instance. \cite{js/function}
\end{quote}
Depending on the firewall configuration, a WAF might use regex filtering to detect usage of \verb|Function()|.
In this case, accessing the Function() constructor via an array object can be an alternative.

\subsubsection{square bracket notation}
JavaScript provides a \quotes{square bracket notation} to access properties of an Object. It is used to access multiword properties as well as provide a way of obtaining property names as the result of an expression. \cite{js/brackets}

This \quotes{square bracket notation} allows to substitute the dot notation in a case where using the dot notation is not feasible or blocked by firewall rules.
For instance, in a payload composed using the function constructor as stated in \ref{sec:functionconstructor}, the part \\ \verb|[].constructor.constructor(`alert('XSS')`)()| can be substituted with \\ \verb|[][`constructor`][`constructor`](`alert('XSS')`)()|.


\subsubsection{string replace}
\label{sec:stringreplace}
When passing certain characters is not feasible, there is a chance they can be substituted in a string replacement strategy using the \verb|+| operator.
A part of the description of the \verb|+| operator in JavaScript states:
\begin{quote}
	The + operator is overloaded for two distinct operations: numeric addition and string concatenation. When evaluating, it first coerces both operands to primitives. ... \cite{js/+}
\end{quote}
In the process of primitive coercion, objects are converted to primitives by calling their \verb|[@@toPrimitive]()|, \verb|valueOf()| and \verb|toString()| methods in the given order. Date and Symbol objects are the only built-in objects that override the \verb|[@@toPrimitive]()| method.
Objects without an override for the \verb|[@@toPrimitive]()| method inherit \verb|valueOf| from \verb|Object.prototype.valueOf|, which returns the object itself.
Since the return value is an object, it is ignored and \verb|toString()| is called instead. \cite{js/primitiveCoercion}

In the code \verb|open + []|, \verb|open| is a built-in Function object, \verb|[]| is a built-in Array object.
In the process of primitive coercion, for both, their \verb|toString()| method is being called.
When they are joined by the \verb|+| operator, it results in a String in the form of \verb|open.toString()| concatenated with \verb|[].toString()|. The result is equal to \verb|open.toString()| (see Listing~\ref{lst:opentostring}) as \verb|[].toString()| returns an empty String.

\begin{lstlisting}[style=basicStyle, caption=open.toString() in JavaScript, label={lst:opentostring}]
function open() {
    [native code]
}
\end{lstlisting}

In JavaScript, array-like access to String objects is possible. \cite{js/stringbrackets} For example, \verb|[open+[]][0][13]| returns the character \verb|(|. \verb|[open+[]][0][14]| returns the character \verb|)|. Listing~\ref{lst:opentostringindices} visualizes open.toString() with the corresponding indices.

\begin{lstlisting}[style=basicStyle, caption=open.toString() with indices in JavaScript, label={lst:opentostringindices}]
f u n c t i o n   o p  e  n  (  )     {  \n
0 1 2 3 4 5 6 7 8 9 10 11 12 13 14 15 16 17 18 19 20 21

[  n  a  t  i  v  e     c  o  d  e  ]  \n }
22 23 24 25 26 27 28 29 30 31 32 33 34 35 36
\end{lstlisting}


\subsubsection{JSFuck}


\subsubsection{Aurebesh}



