\section{Firewall evasion techniques}
\label{sec:Firewall evasion techniques}

Firewall evasion techniques covering XSS Payloads have been chosen.


\definecolor{codegreen}{rgb}{0,0.6,0}
\definecolor{codegray}{rgb}{0.5,0.5,0.5}
\definecolor{codepurple}{rgb}{0.58,0,0.82}
\definecolor{backcolour}{rgb}{0.95,0.95,0.92}
\lstdefinestyle{ruleStyle}{
	backgroundcolor=\color{backcolour},
	commentstyle=\color{codegreen},
	keywordstyle=\color{magenta},
	numberstyle=\tiny\color{codegray},
	stringstyle=\color{codepurple},
	identifierstyle=\color{blue},
	rulecolor=\color{black},
	basicstyle=\ttfamily\footnotesize,
	breakatwhitespace=false,
	breaklines=true,
	captionpos=b,
	keepspaces=true,
	numbers=left,
	showspaces=false,
	showstringspaces=false,
	showtabs=false,
	tabsize=2,
	frame=single,
	morekeywords={test, do, <file>},
	aboveskip=1\baselineskip,
	escapeinside=``
}

\subsection{XSS Payloads}

\subsubsection{String passing}

\subsubsection{Tagged Template Literals}
Tagged Template Literals can be used in javascript payloads to avoid detection of javascript function calls in an http request. Dr. Alex Rauschmeyer explains Tagged Template Literals in ``Exploring ES6'':
\begin{quotation} Tagged Template Literals are function calls whose parameters are provided via template literals. [...]
	The following is a tagged template literal (short: tagged template):
	\begin{lstlisting}
tagFunction`Hello ${firstName} ${lastName}!`
\end{lstlisting}
	Putting a template literal after an expression triggers a function call, similar to how a parameter list (comma-separated values in parentheses) triggers a function call. The previous code is equivalent to the following function call (in reality, first parameter is more than just an Array, but that is explained later).
	\begin{lstlisting}
tagFunction(['Hello ', ' ', '!'], firstName, lastName)
\end{lstlisting}
	Thus, the name before the content in backticks is the name of a function to call, the tag function.
	\cite{exploringes6/templatelit}
\end{quotation}

For instance, as you can also see in Listing~\ref{lst:listing-alertXSSblocked}, the payload \verb|alert('XSS')| will be blocked by the ModSecurity firewall while using CRS 4.1.0.

\begin{lstlisting}[style=ruleStyle, language=XML, caption=alert("XSS") blocking example]
`\label{lst:listing-alertXSSblocked}`<payload>alert("XSS")</payload>
<file>"rules/REQUEST-941-APPLICATION-ATTACK-XSS.conf"</file>
<fileDetails>[line "714"] [id "941390"]<fileDetails>
<MatchedData>"alert("</MatchedData>
\end{lstlisting}

As the function \verb|alert| can be called with an array as first parameter, it is possible to substitute the payload with \verb|alert`XSS`|. The substitute payload will pass through the ModSecurity firewall in the defined configuration. {\color{red}(insert ref to config here)}

Technique by \cite{onecons/wafbypass}.

\subsubsection{Function constructor}
Javascript payloads that are being passed in string format can be executed using the eval function. The \verb|eval()| function will evaluate the source string given as an argument and return its completion value. \cite{js/eval}
A possible usecase for \verb|eval()| in the quest to bypass a \gls{waf} is the splitting of function names that would cause a HTTP request to get rejected by the firewall. Looking at Listing~\ref{lst:listing-alertXSSblocked}, it is clear that the function call \verb|alert('XSS')| is being blocked by the ModSecurity firewall using CRS 4.1.0. To avoid a match on \verb|alert(|, it is possible to split the function call into parts \verb|al|, \verb|e| and \verb|rt('XSS')| by using string concatenation in the form of \verb|`al` + `e` + `rt('XSS')`|. Sending the payload in this form will not cause the the desired effect, as a javascript interpreter will interpret the payload as a string instead of a function call to the \verb|alert()| function. Using the \verb|eval()| function in combination with the splitted string solves this problem. Passing the string \verb|`al` + `e` + `rt('XSS')`| in the form of \verb|eval(`al` + `e` + `rt('XSS')`)| as an argument to the \verb|eval()| function will cause the string to be evaluated and interpreted. {\color{red}unfortunately Firewalls check for eval( too ...}

\label{sec:XSS Payloads}
