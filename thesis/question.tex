\subsection{Relevance of this work}
Why does one ask this question:
\begin{quote} "How well can web application firewalls in their generic standard configuration protect a web application against targeted malicious requests?" 
\end{quote}
The author of this work believes that this question is always asked by security minded staff in the context of deploying a Web Application. 
For a variety of reasons, such as legal implications or turnover loss, generally institutions deploying applications are considering the possibility of security weaknesses built into their applications. These security weaknesses could allow malicious-minded actors to inflict damage to the application itself or to the systems that are part of the applications network. Weaknesses should be considered when deploying any kind of application. Depending on the implementation details, the attack surface of an application will vary greatly. The attack surface of an isolated, embedded application controlling an electric motor is different to the attack surface of a Web Application that is open to traffic from the Internet. As a result of allowing such traffic, the attack surface of a Web Application is only limited to the ingenuity of an attacker who knows to handle the protocols the Web Application is built upon. Taking a Web Application that provides some services as response to HTTP requests as an example, the response to each request depends solely on the definition the application's engineer provided with the applications source code. If this engineer does not consider every possible request modification, which there are usually many, a possible security weakness could have been put in the applications source code. Attackers actively try to figure out and exploit such weaknesses. More details on Web Applications and security risks will be laid out in the following chapters.

The risk of security weaknesses is inherent to deploying (Web-) applications which are open to traffic from the Internet. The National Institute of Standards and Technology by the U.S. Department of Commerce uses the term "Defense-in-depth" in a multitude of their information security standards. 
Following the idea of defense-in-depth, multiple layers of security measures are applied in a layered manner to achieve security objectives. Attacks that are missed by one of those layers should be caught by another layer. \cite{nist/did}
Under this paradigm, institutions deploying a Web Application are considering deploying a Web Application Firewall as another security layer. As there is not neccessarily staff and budget for an extensive configuration available, to estimate the value in terms of added security while keeping configuration times at a minimum, the question
\begin{quote} "How well can web application firewalls in their generic standard configuration protect a web application against targeted malicious requests?" 
\end{quote}
will be asked.
