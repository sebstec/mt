% \subsection{Ideas for Future Work}
During the evaluations laid out under Section~\ref{sec:singleiterationeva}: Single Iteration Evaluation and Section~\ref{sec:multiiteration}: Multi Iteration Evaluation, it became apparent that programmatic automation to create and modify the payloads would improve the process. To achieve this, conventional as well as machine learning approaches to developing an application for this use case are conceivable.

In the conventional approach, initially implementing functions to intelligently parse the firewall log file would be required. 
This would be done with the goal of making the application understand the part of the payload that got detected.
As a next step, the application needs to select a fitting evasion technique to apply. For this, the application needs to understand the payload class. In the context of payload modification, Cross-Site Scripting payloads need to be treated differently to SQL-injection payloads. 
Each known evasion technique needs to be defined to the application with its incurred limitations and characteristics. 
Once a fitting evasion technique has been selected, the evasion technique needs to be programmatically applied to the payload. 
This requires the application to deeply understand the payload effect and intention as well as know different possibilities to apply an evasion technique depending on the chosen payload. Once the payload has been obscured, it should be tested for validity. 
The application needs to be able to interpret and evalutate the effects of a payload. 
Only after these fundamentals are covered, can the proposed evaluation technique be implemented into the application. 
With this, the afromentioned functions could be applied iteratively to the combination of (obscured) payload and firewall log. 
The steps following what is described under Section~\ref{sec:evaluation}: Evaluation respectively Section~\ref{sec:proposal}: Rule Proposal could thus be automated by such an application.

The author of this work considers the machine learning approach to automating the proposed evaluation methodology similar to the described conventional approach.
Tokenization and context-awareness of Large Language Models could potentially assist in achieving a working implementation of the functions to understand the log messages, understand the payload and understand limitations imposed by certain evasion techniques. 
To investigate the substance behind this statement, research into this idea seems promising. 
