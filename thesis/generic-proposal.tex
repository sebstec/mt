\section{Generic Evaluation Proposal}
\label{sec:proposal}
As seen in Section~\ref{sec:EvaluationResults}, when evaluating web application firewalls with firewall evasion techniques, an iterative approach to developing payloads seems to be a good solution. If the evaluation is done from the perspective of an firewall administrator, access to logs is possible. Using the approach from Section~\ref{sec:evaluation}, bypassing payloads can be developed quickly if the log message and specific matched part of the payload is known. Once a bypassing payload has been successfully developed and tested against the firewall that is to be evaluated, counter measures can be implemented. As stated in Section~\ref{sec:evalinterpretation}, even if there is no immediate possibility to implement rules based on the discovered bypass, the knowledge gained can be valuable. Based on what was described in Section~\ref{sec:evaluation}: Evaluation and the results thereof (Section~\ref{sec:EvaluationResults}), the following generic evaluation approach is proposed.

\subsection{Proposed step 1: Gathering evasion techniques}
As a first step, research into specific evasion techniques covering the used technologies by a web application that is to be protected by the evaluated firewall should be conducted. If the web application uses a NoSQL database, evasion techniques specific to obscuring SQL-injection payloads should be ignored in favor of evasion techniques specific to obscuring NoSQL-injection payloads. As detailed in Section~\ref{sec:varioustech}, some evasion techniques are not specific to any kind of (injection-) attack and could be considered more broadly.

\subsection{Proposed step 2: Single iteration evaluation}
After evasasion techniques to be evaluated against the tested firewall have been chosen, they should be used to create proof-of-concept payloads in a seperate fashion. Every evasion technique should be evaluated against the firewall configuration by itself. The crafted payloads are subsequently tested against the firewall while analyzing the firewall log to take note on which payloads show potential to be used within a multi iteration approach. Potential can be classified in 3 different classes: unsuable, semi-usable and full potential. Only semi-usable and full potential evasion techniques should be used within the following multi iteration approach.

Taking the results from Section~\ref{sec:casealternationevaluation} as an example: The log file created by the ModSecurity firewall shows that this evasion technique is to be classified as unusable. When sending a request with a proof-of-concept payload, the log reveals that the regex rule used to filter out the payload is configured to ignore case. As such, this techniques has no effect on any payload sent.

Taking the results from Section~\ref{sec:taggedtemplateliteralsevaluation} as an example: While the evasion technique could be used to send a bypassing proof-of-concept payload in the form of \verb|alert`XSS`| towards the web application, by the way Tagged Template Literals are implemented in JavaScript, some payload limitations are incurred. In the example, because of using Tagged Template Literals, the function \verb|alert()| can no longer be called with a variable given as a parameter. Payload limitations incurred by using Tagged Template Literals are further discussed in Section~\ref{sec:payloadlimitations}. Considering incurred payload limitations, using Tagged Template Literals as an evasion technique can be classified semi-usable.

Taking the results from Section~\ref{sec:aurebeshevaluation} as an example: Using the Aurebesh evasion technique, the proof-of-concept payload \verb|prompt(1,2)| could be sent without any incurring limitations. As such, using Aurebesh as an evasion technique can be classified with full potential.

\subsection{Proposed step 3: Multi iteration evaluation}
Once evasion techniques showing potential have been discovered by using the proof-of-concept payloads crafted in Step 2, multi iteration evaluation can begin. Initially, payloads of interest should be researched or crafted. To keep it relevant, these payloads should have possible malicious influence on the web application that is to be protected by the evaluated firewall. As in the proposed step 1, the technologies used by the web application should be considered when deciding on payload variants. Taking a banking web application using JavaScript in the Frontend as an example, Cross-site-scripting payloads are of interest. Once a payload that could negatively influence the web application is found, iteratively applying the noted evasion techniques from proposed step 2 can begin. Multi iteration evaluation is started by sending the plain, not-yet-obscured payload towards the web application and subsequently analyzing the firewall log. As stated in Section~\ref{sec:evaluation}, each iteration begins with analyzing the firewall log of filtered/rejected requests for matched parts of the payload. Once these parts have been identified, an evasion technique classified as usable in proposed step 2 is used to obscure these parts of the payload. During application of an evasion technique to the payload, the payload needs to be kept valid. The validity of the payload should be tested at each iteration. Once the payload has been obscured, it is sent towards to web application. Subsequently checking the firewall logs marks the beginning of the next iteration. If at any iteration the payload passes the filter of the firewall, a new bypassing payload has been discovered.

\subsection{Proposed step 4: Implementing new firewall rules}
\label{sec:genericproposalstep4}
With the found bypassing payload from proposed step 3 and the knowledge about the evasion techniques used, new rules to extend the firewall configuration can be established. The crafting of new rules at this point is based on the gained knowledge. Knowing the bypassing payload as well as all applied evasion techniques, it should be able to craft a rule that detects at least one of the applied evasion techniques and therefore could cause the firewall to block the obscured payload. 
Taking the evaluation result of Section~\ref{sec:funconstrconbypass} as an example: A rule could be implemented that would detect the usage of \verb|]["constructor"]|. This would block the request as it is no longer possible to call the function constructor using the square bracket notation if this rule is implemented. This specific example if further detailed in Section~\ref{sec:rulespropfunctionconstructor}.

Once rules for a specific payload have been developed and implemented, these rules are reflected on the used evasion techniques. Some evasion techniques allow for mutiple vectors to create bypassing payloads. The coverage provided by the newly implemented rules is assessed and potentially additionally found vectors for bypassing payloads are noted down to be used in the next evaluation step. 
\subsection{Continuing the Evaluation}
Finally, new rules should be tested against the found bypassing payloads to verify their detection. After verifying the rules, either the evaluation comes to an end or should be continued at the initial phase of proposed step 3. If additional vectors for bypassing payloads have been found during the previous step, these are handled first. Handling these first allows to focus on one evasion technique at a time.
