\subsection{Web Application}
In the scope of Software, a web application is software that runs in a web browser. 
They allow to access complex functionality without installing or configuring software. 
Web application provide a means for businesses and other institutions or individuals to share information and deliver services remotely and securely. 
Website features like shopping carts, product search and instant messaging are web applications by design. \cite{aws/webapp}
Amazon, providing multiple web applications themselves, describes the typical architecture of a web application as follows: 
\begin{quote}
Web applications have a client-server architecture. 
Their code is divided into two components—client-side scripts and server-side scripts.  


\textbf{Client-side architecture}


The client-side script deals with user interface functionality like buttons and drop-down boxes.
When the end user clicks on the web app link, the web browser loads the client-side script and renders the graphic elements and text for user interaction.
For example, the user can read content, watch videos, or fill out details on a contact form.
Actions like clicking the submit button go to the server as a client request.


\textbf{Server-side architecture}


The server-side script deals with data processing.
The web application server processes the client requests and sends back a response.
The requests are usually for more data or to edit or save new data.
For example, if the user clicks on the Read More button, the web application server will send content back to the user.
If the user clicks the Submit button, the application server will save the user data in the database.
In some cases, the server completes the data request and sends the complete HTML page back to the client.
This is called server side rendering. \cite{aws/webapp}
\end{quote}
{\color{red} Http requests \\ explanation}


With recent developments leading to increased availability and usage of the Internet, the way businesses and institutions are run has been influcenced.
This has led to the widespread adoption of web applications by companies.
Web applications allow businesses to streamline their processes and increase efficiency. \cite{stackpath/webapp}

