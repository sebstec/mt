\subsection{Web Application}
With recent developments leading to increased availability and usage of the Internet, the way businesses and institutions are run has been influcenced.
This has led to the widespread adoption of web applications by companies. \cite{stackpath/webapp}
Businesses hope to streamline their processes and increase efficiency by employing web applications. 

In the scope of Software, a web application is software that runs in a web browser.
They allow to access complex functionality without installing or configuring software.
Web application provide a means for businesses and other institutions or individuals to share information and deliver services remotely and securely.
Website features like shopping carts, product search and instant messaging are web applications by design. \cite{aws/webapp}

Amazon, providing multiple web applications themselves, describes the typical architecture of a web application as follows:
\begin{quote}
	Web applications have a client-server architecture.
	Their code is divided into two components—client-side scripts and server-side scripts.


	\textbf{Client-side architecture}


	The client-side script deals with user interface functionality like buttons and drop-down boxes.
	When the end user clicks on the web app link, the web browser loads the client-side script and renders the graphic elements and text for user interaction.
	For example, the user can read content, watch videos, or fill out details on a contact form.
	Actions like clicking the submit button go to the server as a client request.


	\textbf{Server-side architecture}


	The server-side script deals with data processing.
	The web application server processes the client requests and sends back a response.
	The requests are usually for more data or to edit or save new data.
	For example, if the user clicks on the Read More button, the web application server will send content back to the user.
	If the user clicks the Submit button, the application server will save the user data in the database.
	In some cases, the server completes the data request and sends the complete HTML page back to the client.
	This is called server side rendering. \cite{aws/webapp}
\end{quote}
The foundation of any data exchange on the Web is the HTTP protocol.
It is a client-server protocol on the application layer in which clients and servers communicate by exchanging individual messages.
The messages sent by the client are called requests, the messages sent by the server are called responses. Responses are sent as an answer to requests.

The actor on the client-side is the user-agent.
A user-agent is any tool that acts on behalf of the user, such as web browsers used by users visiting web pages.
User-agents are always the entity that initiates a request.
To dispay a web page, a web browser sends a request to fetch the data that represents the page.
Additional requests are made during the display or the interaction with a web page.
These are triggered corresponding to execution scripts.

Responses are sent by the server, which is the actor on the server-side.
A server serves the data of a web page as requested by the client.

Unless there is a direct physical connection between the server and the client, numerous machines from the Web stack are used in between to relay the HTTP messages.
Most of these operate at the transport, network or physical layer.
Those operating at the application layer are generally called proxies.
Proxies exist in message altering or transparent form, in which case they forward the message without alteration. \cite{mdn/http}

Web application firewalls can be filtering proxies that filter out HTTP messages indentified as malicious.


