\subsection{Interpretation of evaluation results}
Evaluation results show that there are multiple Firewall Evasion techniques effective against the tested firewall configuration. {\color{red}X of Y} tested Firewall Evasion techniques have proven effective during the evaluation. Using the techniques described in Section~\ref{sec:unicodenormalization}: Unicode Normalization, Section~\ref{sec:taggedtemplateliterals}: Tagged Template Literals, Section~\ref{sec:functionconstructor}: Function constructor in combination with Section~\ref{sec:sbn}, Section~\ref{sec:stringreplace} and Section~\ref{sec:aurebesh}, multiple payloads can be constructed to bypass the tested firewall configuration. Section~\ref{sec:multiiteration} states examples including passing malicious html elements (Section~\ref{sec:htmlencjsesc}) as well as directly executable JavaScript code (Section~\ref{sec:funconstrconbypass}. 
Further possible payloads include payloads that could be used to circumvent a more stringent ruleset.
Section~\ref{sec:avoidingbypassA} states a proof of concept of a valid payload that avoids passing the characters \verb|()| as part of the payload.
Similarly, Section~\ref{sec:avoidingbypassB} states a proof of concept of a valid payload that avoids passing the characters \verb|{}| as part of the payload. 

While the evaluation results state some possible evading payloads as is, it seems natural that there are more evading payloads possible using a combination of the stated techniques under Section~\ref{sec:firewallevasiontechniques}.

Using these results, the administrator of the tested firewall can implement new filtering rules. Depending on the implementation context of the firewall, such as the language used to configure the firewall rules, some limitations might apply. Using the ModSecurity firewall configured to use the CoreRuleSet as an example, the firewall rules can be extended following the guide provided in their documentation. By extending the ruleset, it is possible to blacklist the bypassing payloads. Created and proposed rules based on the evaluation results mentioned under Section~\ref{sec:singleiterationeva} and Section~\ref{sec:multiiteration} are detailed under Section~\ref{sec:rulesproposal}. Depending on the specific Web Application that is protected by the Web Application firewall, adding rules that completely block some Evasion Techniques that were detected to be effective might be possible. If that is not the case, it could be possible to blacklist payloads that were discovered during the evaluation. At least, some discovered bypassing payloads are known after the evaluation. With this knowledge, the Web Application can be fortified using a different methodology to adding firewall filtering rules. If nothing is to be done about the findings, still the knowledge gained about bypassing payloads can be used to estimate the risk posed to the Web Application. 

Gathered evaluation results can be used to gain knowledge about payload limitations that incur by applying Firewall Evasion techniques to a payload. (Using this knowledge, more defensive measures based on exploiting the incurred payload limitations can theoretically be deployed to render a specific evasion technique useless.)
Incurring payload limitations will be discussed in the following section.



... 
