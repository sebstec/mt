\subsection{Interpretation of evaluation results}
\label{sec:evalinterpretation}
Evaluation results show that there are multiple firewall evasion techniques effective against the evaluated firewall configuration. 11 of 18 Firewall Evasion techniques have proven effective during the single iteration evaluation, as stated under Section~\ref{sec:singleiterationeva}.
Using the techniques described in Section~\ref{sec:unicodenormalization}, Section~\ref{sec:htmlcharreftech}, Section~\ref{sec:fromcharcodetech}, Section~\ref{sec:stringconc}, Section~\ref{sec:jscommentsub}, Section~\ref{sec:jsescape}, Section~\ref{sec:taggedtemplateliterals}, Section~\ref{sec:functionconstructor}, Section~\ref{sec:sbn}, Section~\ref{sec:stringreplace} and Section~\ref{sec:aurebesh}, multiple payloads can be constructed to bypass the ModSecurity firewall configured to use the \acrshort{crs} in version 4.1.

Section~\ref{sec:multiiteration} states some of the possible bypasses for the evaluated firewall configuration constructed in multiple iterations of evasion technique application:
\begin{itemize}
	\item Section~\ref{sec:jsencpercenc} states a valid bypass using JavaScript character escaping in cases where the data is percent encoded.

	\item In Section~\ref{sec:jsescapemultiiter}, JavaScript character escaping is combined with string concatenation to create a valid bypass in cases where the data is percent encoded.

	\item Section~\ref{sec:funconstrconbypass} states a valid bypass created using the \verb|Function()| constructor, string concatenation and JavaScript comment substitution.

	\item In Section~\ref{sec:charcodemultiiter} the \verb|Function()| constructor is combined with string concatenation and the \verb|fromCharCode()| function to create a valid bypass.

	\item Section~\ref{sec:avoidingbypassA} states a proof of concept of a valid payload that avoids passing the characters \verb|()| as part of the payload.

	\item Section~\ref{sec:avoidingbypassB} states a proof of concept of a valid payload that avoids passing the characters \verb|{}| as part of the payload.

	\item In Section~\ref{sec:forcedunicodenorm}, the \verb|Function()| constructor is used to create a payload that enforces Unicode normalization on itself. This enables the payload to be obscured using uncommon Unicode characters.

	\item In Section~\ref{sec:doublehtmlcharref}, \acrshort{html} character references are used in multiple iteration to create a valid payload.

	\item In Section~\ref{sec:htmlencjsesc}, \acrshort{html} character references are used in combination with JavaScript normal character escape to create a valid payload aiming at exploiting a \acrshort{xss} vulnerability to exfiltrate secrets.
\end{itemize}

While the evaluation results state some possible evading payloads as is, it seems natural that there are more evading payloads possible using a combination of the techniques stated under Section~\ref{sec:firewallevasiontechniques} which have proven effective.

Through these results, the administrator of the evaluated firewall configuration has gained insight into which of the payloads are blocked by the evaluated web application firewall configuration and which evasion techniques have proven successful in bypassing the ruleset. Using these results, the administrator can implement new filtering rules to cover the discovered loopholes in the ruleset.

Depending on the implementation context of the firewall, such as the language used to configure the firewall rules, some limitations regarding the extension of the ruleset might apply. Using the ModSecurity firewall configured to use the CoreRuleSet as an example, there are no limitations and the firewall rules can be extended following the guide provided in their documentation. By extending the ruleset, it is possible to blacklist the bypassing payloads. Created and proposed rules based on the evaluation results are detailed under Section~\ref{sec:rulesproposal}.

Gathered evaluation results can be used to gain knowledge about payload limitations that incur by applying firewall evasion techniques to a payload. Using this knowledge, workarounds for payloads initially considered ineffective or invalid might be found. Incurring payload limitations will be discussed in the following section.
